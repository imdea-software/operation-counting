%!TEX root = draft.tex
\section{Approximating History Inclusion}
\label{sec:counting}

By the equivalences of Section~\ref{sec:lin}, checking whether a given history
$H(e)$ is included in a fixed set $H(L)$ of library histories is equivalent to
checking whether $H(e)$ is linearizable with respect to $L$, for a fixed atomic
library $L$. It follows that deciding $H(e) \in H(L)$ is NP-hard for an
arbitrary, but fixed, library $L$~\cite{journals/siamcomp/GibbonsK97}.
Generally speaking, the only known algorithms to decide $H(e) \in H(L)$ must
check whether each possible linearization of the partially-ordered history
$H(e)$ is equivalent to some sequential execution of operations according to
$L$, backtracking to try alternate linearizations on each failed attempt.
Recent work implies that the more general problem of checking whether \emph{all
histories} $H(L_1)$ of a given library $L_1$ are included in the set of histories $H(L_2)$
of a fixed library $L_2$ is undecidable, since it is equivalent to checking whether $L_1$ is
linearizable w.r.t. $L_2$~\cite{conf/esop/BouajjaniEEH13}.

These complexity obstacles suggest investigating approximations to the history
inclusion problems --- i.e.,~both $h \in H(L)$ and its more general variation
$H(L_1) \subseteq H(L_2)$ --- in order to devise tractable algorithms.

In this work, we focus on parameterized \emph{under approximations} for
detecting violations to observational refinement, achieving increasing accuracy
with decreasing efficiency. For this, we design a notion of parameterized
history-weakening approximation functions $A_k$ which map any history $h$ to a
weaker history $A_k(h) \preceq h$, and which have the following properties:
\begin{description}

  \item[Strength-increasing:]
  $A_0(h) \preceq A_1(h) \preceq .. \preceq A_k(h) \preceq h$.
  
  \item[Completeness:]
  there exists $k \in \<Nats>$ such that $h \preceq A_k(h)$.
  
  \item[Tractable inclusion:]
  $A_k(h) \in H(L)$ is decidable in polynomial time when $k$ is fixed. %and $L$ are fixed.

\end{description}
This weakening-based approximation is convenient since whenever $A_k(h)$ is not
included in $H(L)$, then neither is $h$, since $H(L)$ is closed under
weakening; if $h$ were to belong to $H(L)$, then any weakening, and in
particular $A_k(h)$, would also belong to $H(L)$. While completeness means that
increasing $k$ increases the ability to detect observational refinement
violations, this must incur a decrease in efficiency since the inclusion
problem $A_k(h) \in H(L)$ is NP-hard when $k$ is not fixed. By design, the
approximation function $A_k$ allows us to solve the approximate history
inclusion problem $A_k(h) \in H(L)$ in polynomial time for fixed $k$. For the
more general problem of refinement between libraries $L_1$ and $L_2$, our
approximation asks whether $A_k(h) \in H(L_1) \setminus H(L_2)$, and becomes
decidable for fixed $k$, so long as the set $\set{ A_k(h) : h \in H(L_1) }$ is
computable. Completeness of $A_k$ ensures overall completeness, i.e.,~that for
any $h \in H(L_1) \setminus H(L_2)$ there is some $k \in \<Nats>$ such that
$A_k(h) \in H(L_1) \setminus H(L_2)$.

Our key challenge is to develop approximation functions $A_k$ for which history
inclusion can be computed in polynomial time for fixed $k$, and for which
observational refinement violations surface with small $k$. We demonstrate the
latter in Sections~\ref{sec:containers}--\ref{sec:exp}.

In this section we develop a schema of approximation functions for which the
approximate history inclusion problem is polynomial-time computable. Our
development exploits structural aspects of the history inclusion problem; in
particular, we exploit the fact that histories are \emph{interval orders}, with a natural measure of
complexity, i.e.,~the interval order \emph{length}~\cite{phd/Greenough76}.
Leveraging this notion of length, we abstract each history $h$ to a weaker
history $A_k(h)$ whose length is bounded by $k$, and represent the set $H(L)$
of histories by a formula $@Y_k$ over bounded interval length, against which
$A_k(h)$ can be evaluated in polynomial time~(\S\ref{sec:counting:logic}).
Finally, we exhibit a program monitoring scheme which can be used to decide our
approximate observational refinement problem $\exists h.\ A_k(h) \in H(L_1)
\setminus H(L_2)$, or as a general purpose runtime
montior~(\S\ref{sec:counting:monitor}).

\subsection{Bounded-Interval-Length History Inclusion}
\label{sec:counting:logic}

The \emph{past} of an element $o \in O$ of a poset $\tup{O,<}$ is the set
\begin{align*}
  \<past>(o) = \set{ o' \in O : o' < o }
\end{align*}
of elements ordered before $o$.

\begin{example}

For the history $h$ in Figure~\ref{fig:stacks}(b), the $\<pop>=>3$ and $\<pop>=>1$
operations have the same past, namely the operation $\<push>(1)$, while the past of the $\<push>(3)$
operation consists of the $\<push>(1)$, $\<pop>=>1$, and $\<push>(2)$ operations.
  
\end{example}

This notion of operations' pasts induces a linear notion of time into execution
histories due to the following fact.

\begin{lemma}[Rabinovitch~\cite{Rabinovitch197850}]

  The set $\set{ \<past>(o) : o \in O }$ of pasts of an interval order
  $\tup{O,<}$ is linearly ordered by set inclusion.

\end{lemma}

\noindent
Furthermore, this linear notion of time has an associated notion of
\emph{length}, which corresponds to the length of the linear order on
operation's pasts.

\begin{definition}[Greenough~\cite{phd/Greenough76}]
  \label{lemma:len}
  
  The \emph{length} of an interval order $\tup{O,<}$ is one less than the
  number of its distinct pasts.

\end{definition}
We denote the length of the interval order $\tup{O,<}$ underlying a history
$h = \tup{O,f,<}$ as $\len h$.

Our history-weakening approximation functions $A_k$ map histories to weaker
histories whose corresponding interval orders have length at most $k$. While
there are various ways to define such a function $A_k$, any such function
enables the polynomial-time inclusion check $A_k(h) \in H(L)$ which we
demonstrate in the following; Section~\ref{sec:counting:monitor} describes a
polynomial-time computable instantiation of $A_k$ which we have found useful in
practice.

Interval orders have canonical representations which associate to operations
integer-bounded intervals on the real number line; their canonical
representations minimize the interval bounds to the interval-order length.

\begin{lemma}[Greenough~\cite{phd/Greenough76}]
  \label{lem:representation}
  
  An interval order $\tup{O,<}$ of length $n \in \<Nats>$ has a canonical
  representation $I : O -> [n]^2$ mapping each $o \in O$ to the interval $I(o)
  = [i,j] \subseteq [0,n]$, with
  \begin{align*}
    i = |\set{\past{o'} : o'<o}| \text{ and }
    j = |\set{\past{o'} : \neg o<o'}| \text{.}
  \end{align*}
  such that $o_1 < o_2$ if{f} $\sup I(o_1) < \inf I(o_2)$.
\end{lemma}

\noindent
The canonical representation thus associates the interval $[i,j]$ to an
operation $o$ which succeeds the $i$th past, and precedes the $(j\!+\!1)$st
past. Note that the interval of an operation can be determined in polynomial
time by counting the distinct pasts among operations.

\begin{example}

  Figure~\ref{fig:stacks}(e) pictures the canonical representation of the history $h$ in
  Figure~\ref{fig:stacks}(b). Note that the length of $h$ is 4.

\end{example}

We define the \emph{counting representation} $\Pi(h)$ of a history $h =
\tup{O,f,<}$, whose underlying interval order $\tup{O,<}$ has the canonical
representation $I : O -> [n]^2$, as the multiset
\begin{align*}
  \Pi(h) = \mset{ \tup{f(o),I(o)} : o \in O }
\end{align*}
of label-and-interval pairs; defining $\Pi(L) = \set{ \Pi(h) : h \in
H(L) }$ yields a criterion equivalent to history inclusion based on counting
representations (which follows  from Lemmas~\ref{lem:intervals} and~\ref{lem:representation}).

\begin{lemma}

  $H(e) \in H(L)$ if{f} $\Pi(H(e)) \in \Pi(L)$.

\end{lemma}

%\begin{proof}
%
%  Follows from Lemmas~\ref{lem:intervals} and~\ref{lem:representation}.
%\end{proof}

\begin{example}

 The counting representation of the history $h$ in Figure~\ref{fig:stacks}(b) is the multiset
 \begin{align*}
 \{\!\{&\tup{\<push>(1),[0,0]}, \tup{\<pop>=>1,[1,1]},\tup{\<pop>=>3,[1,3]},\\
 &\tup{\<push>(2),[2,2]}, \tup{\<push>(3),[3,3]},\tup{\<pop>=>{\tt Empty},[4,4]}\}\!\}.
 \end{align*}
% while the counting representation of the history $h'$ in Figure~\ref{fig:stacks}(c) is
% \begin{align*}
% \{\!\{&\tup{\<push>(1),[0,0]},\tup{\<push>(2),[0,0]},\tup{\<pop>=>1,[0,0]}, \\
% &\tup{\<push>(3),[0,0]},\tup{\<pop>=>{\tt Empty},[1,1]},\tup{\<pop>=>\bot,[0,1]},\\
% &\tup{\<pop>=>\bot,[0,1]}\}\!\}
% \end{align*}
\end{example}

This counting representation leads to an effective logical characterization of
history sets using arithmetic operations, inequalities, and a counting function
$\#$. Formally, \emph{operation counting logic} is the first-order logic whose
syntax is listed in Figure~\ref{fig:logic}. The $\#$ function is interpreted
over a history $h = \tup{O,f,<}$ as
\begin{align*}
  \db{\#(@l,i,j)}_h = |\set{ o \in O : f(o) = @l \text{ and } I(o) = [i,j] }|
\end{align*}
where $I$ is the canonical representation of $\tup{O,<}$.

\begin{figure}
  \begin{align*}
    i,j & \in \<Nats>
      \qquad \text{integer constants} \\
    @l & \in \<Labels>
      \qquad \text{operation-label constants} \\
    x & : \<Labels>
      \qquad \text{operation-label variables} \\
    X & ::= @l \mid x \\
    T & ::= i \mid \#(X, i, j) \mid T + T \\
    F & ::= T \leq T \mid P(X,.,X) \mid \neg F \mid F \land F \mid \exists x.\ F
  \end{align*}
  \caption{The syntax of Operation Counting Logic.}
  \label{fig:logic}
  \vspace{-2mm}
\end{figure}

We allow predicates $P$ of arbitrary arity over operation labels, so long as
they are evaluated in polynomial time. Furthermore, operation-label variables
are quantified only over the operation labels which occur in the history over
which a formula is evaluated. The satisfaction relation $|=$ for quantified
formulas is defined by:
\begin{align*}
  h |= \exists x.\ F
  \quad \text{if{f}} \quad
  \exists o \in O.\ h |= F[x <- f(o)].
\end{align*}
where $h = \tup{O,f,<}$. The \emph{quantifier rank} of an operation counting
formula $@Y$ is the maximum number of nested quantifiers in $@Y$. An operation
counting formula $@Y$ \emph{represents a library $L$ up to $k$} when $h \in
H(L)$ if{f} $h |= @Y$ for every history $h$ of length at most $k$.

\begin{lemma}\label{lemma:satisfaction}

  Checking if a history $h$ satisfies an operation counting formula
  $@Y$ of fixed quantifier rank is decidable in polynomial time.

\end{lemma}

\begin{proof}\let\qed\relax

  This follows from the facts that: (1) the canonical representation of $h$ is polynomial-time computable (see Lemma~\ref{lem:representation}),
  (2) functions and predicates are polynomial-time computable, and (3) quantifiers are only instantiated over labels occurring in $h$.
  The latter implies that quantifiers can be replaced by a disjunction over the
  labels occurring in $h = \tup{O,f,<}$:
  \begin{align*}
    h |= \exists x.\ F
    \quad \text{if{f}} \quad
    h |= \bigvee_{o \in O} F[x <- f(o)].
  \end{align*}
\end{proof}

\subsection{Monitoring Bounded-Interval-Length Histories}
\label{sec:counting:monitor}

Though there are numerous ways to define a function $A_k$ which approximates
histories to weaker $k$-interval-length histories, the natural solution we
consider here is an $A_k$ which maintains the last $k$ interval bounds
precisely, abstracting all previous interval bounds with equality. Formally,
given a history $h = \tup{O,<,f}$ such that $\len h = n$, and $k \in \<Nats>$,
we define $A_k(h) = \tup{O,<',f}$ by
\begin{align*}
  o_1 <' o_2 & \text{ if{f} } o_1 < o_2 \text{ and } n - k < \inf I(o_2)
\end{align*}
where $I$ is the interval map of $h$. Intuitively, $A_k$ remembers only the the
ordering between ``recent'' operations which have started after interval $n-k$.
It follows by definition that $A_k(h) \preceq h$, since ordering constraints
are only removed from $<$. Note that $A_k(h) = h$ for any history $h$ of length
$n\leq k$.

\begin{lemma}
  \label{lemma:abstraction}

  Let $I$ and $I_k$ be the interval maps of $h$ and $A_k(h)$.
  For each $o \in O$ with $I(o) = [i,j]$:
  \begin{align*}
    I_k(o) = [ \max(i-n+k,0), \max(j-n+k,0) ] \text{.}
  \end{align*}

\end{lemma}

We reduce the online maintenance of $k$-interval-length histories to the
maintenance of integer counters. As the first step in establishing the link
between history and counter maintenance, we define the $\oplus$ operator:
For a given history $h = \tup{O,f,<}$ let
\begin{align*}
  h \oplus m(u)_o & =
    \tup{O \u \set{o}, f[o |-> m(u) => \bot], <'} \\
  h \oplus \<ret>(v)_o & =
    \tup{O, f[o |-> m(u) => v], <} \text{ s.t. } f(o) = m(u) => \bot
\end{align*}
where $<'$ is the transitive closure of $\mathbin{<} \cup \{ \tup{o',o} :
f(o') = \_(\_) => v\neq \bot\}$ relating all completed operations to $o$. As the
following lemma shows, the $\oplus$ operator allows us to manipulate our
$k$-approximations $A_k(H(e))$ directly, without having to maintain the precise
history $H(e)$ of an execution $e$. Note that besides maintaining $A_k(H(e)$,
this formulation requires us to compute $A_{k-1}(H(e))$ occasionally as well.

\begin{lemma}
  \label{lem:step}

  Let $e' = e \cdot m(u)_o$.
  \begin{itemize}

    \item $A_k(H(e')) = A_k(H(e)) \oplus m(u)_o$ if $\len H(e') = \len H(e)$.

    \item $A_k(H(e')) = A_{k-1}(H(e)) \oplus m(u)_o$ if $\len H(e') > \len H(e)$.

  \end{itemize}
  Additionally,
  \begin{itemize}

    \item $A_k(H(e \cdot \<ret>(v)_o)) = A_{k}(H(e)) \oplus \<ret>(v)_o$.

  \end{itemize}

\end{lemma}

The second step in establishing the link between history and counter
maintenance is given by the following lemma, which implies that history
extension, i.e.,~execution step, corresponds to either a single counter
increment, or a single decrement and a single increment.

\begin{lemma}
  \label{lem:incr}

  Let $\len h' = k$. If $h' = h \oplus m(u)_o$ then
  \begin{itemize}

    \item $\Pi(h') = \Pi(h) \cup \set{\tup{m(u) => \bot, [k,k]}}$.

  \end{itemize}
  If $h' = h \oplus \<ret>(v)_0$ then
  \begin{itemize}

    \item $\Pi(h') = \Pi(h) \cup \set{\tup{m(u) => v, [i,k]}}
    \setminus \set{\tup{m(u) => \bot, [i,k]}}$,

  \end{itemize}
  where $I$ is the interval map of $h'$ and $I(o) = [i,k]$.

\end{lemma}

As noted before Lemma~\ref{lem:step}, besides incrementing/decrementing
counters as operations begin and complete, we must occasionally compute
$A_{k-1}(H(e))$ as well, thus merging the least-recent interval bounds, in
order to maintain a bounded-length history approximation. We achieve this with
an interval shifting operation $\overleftarrow{\pi}$ applied to a counting
representation $\pi$. Formally, for all $@l \in \<Labels>$, and $0 \le i \le j
\le k$, if $@l$ is completed then
\begin{align*}
  \overleftarrow{\pi}(@l,[i,j])
  & = @n(\pi(@l,[i+1,j+1]), i<k \land j<k) \\
  & + @n(\pi(@l,[i,j+1]), i=0 \land j<k) \\
%  & + @n(\pi(@l,[i+1,j]), i<k \land j=0) \\
  & + @n(\pi(@l,[i,j]), i=0 \land j=0),
\end{align*}
and if $@l$ is pending then
\begin{align*}
  \overleftarrow{\pi}(@l,[i,j])
  = @n(\pi(@l,[i+1,k]), i<k) + @n(\pi(@l,[i,k]), i=0),
\end{align*}
where the conditional function $@n : \<Nats> \x \<Bools> -> \<Nats>$ is defined by
\begin{align*}
  @n(n,@p) = \left\{
  \begin{array}{ll}
    n & \text{ if } @p \text{ is valid} \\
    0 & \text{ otherwise.}
  \end{array}
  \right.
\end{align*}
The following lemma establishes the intended effect of $\overleftarrow{\pi}$.

\begin{lemma}
  \label{lem:shift}

  If $\len h = k$ and $\pi = \Pi(h)$, then $\overleftarrow{\pi} = \Pi(A_{k-1}(h))$.

\end{lemma}

\begin{algorithm}[t]
  \DontPrintSemicolon
  \SetKwInOut{Input}{Input}
  \SetKwInOut{Output}{Output}
  \SetKw{Init}{initially}
  \SetKw{Yield}{yield}
  \SetKw{True}{true}
  \SetKw{False}{false}
  \SetKw{Increment}{incr}
  \SetKw{Decrement}{decr}
  
  \KwData{Interval length $k \in \<Nats>$}
  \KwData{Stream $e \in (C\u R)^{@w}$ of call/return actions}
  \KwResult{counting representation $\Pi(A_k(H(e)))$}
  \Init{$n = 0$, $s = @0$, $\pi = @0$} \;
  \Switch{input action}{
    \If{call action $m(u)_o$}{
      \If{previous action was return}{
        \Increment $n$ \;
        \lIf{ $n > k$ }{ $\pi <- \overleftarrow{\pi}$ }
      }
      $s(o) <- n$ \;
      \Increment $\pi(m(u) => \bot, [\min(n,k),k])$ \;
    }
    \Case{return action $ \<ret>(v)_o$}{
      $i = s(o) - \max(n-k,0)$ \;
      \Decrement $\pi(m(u) => \bot, [i,k])$ \;
      \Increment $\pi(m(u) => v, [i,\min(n,k)])$ \;
    }
  }
  \Yield $\pi$ \;
  \caption{An online operation algorithm for computing the approximation
    $\Pi(A_k(H(e)))$ of a given history $H(e)$.}
  \label{alg:counting}
\end{algorithm}

Algorithm~\ref{alg:counting} computes the counting representation of a given
history incrementally, according to the increment/decrement and shift
operations of Lemmas~\ref{lem:incr} and~\ref{lem:shift}. The variable $n$
maintains the interval length of the input execution's history $H(e)$, while
the variable $s$ maintains the infimum $s(o)$ of the interval of each operation
$o$ in $H(e)$. The corresponding infimum in $A_k(H(e))$ when $o$ is invoked is
given by $\min(n,k)$, as the interval bounds used in $A_k(H(e))$ may not exceed
$k$. When $o$ completes, it is possible that some number of shift operations
have translated its interval in $A_k(H(e))$. As the number of performed shift
operations is equal to $\min(n-k,0)$, $o$'s infimum in $A_k(H(e))$ when
completing is given by $s(o) - \min(n-k,0)$. It follows by
Lemmas~\ref{lem:incr} and~\ref{lem:shift} that Algorithm~\ref{alg:counting}
computes $A_k(H(e))$.

\begin{theorem}
  \label{thm:counting:alg}

  Algorithm~\ref{alg:counting} computes the history approximation
  $\Pi(A_k(H(e)))$ of an execution $e$ in $O(|\<Labels>| \cdot \<width>(e))$
  space\footnote{The space complexity of Algorithm~\ref{alg:counting} is
  constant in the number of operations when the size of integers is fixed, as
  in the case of modern computer architectures; otherwise, the space complexity
  is logarithmic in the number of operations, i.e.,~in execution length.} and
  $O(|e|)$ time, where $\<width>(e)$ is the maximum number of overlapping operations
  in $e$.

\end{theorem}

Note that, in the worst case, $\<width>(e)=|e|$. However, in practice, the number of
overlapping operations is usally fixed and it doesn't influence the space complexity.

\begin{corollary}

  The approximate inclusion problem $A_k(h) \in H(L)$ is decidable in
  polynomial time, for fixed $k$, given an operation counting formula
  $@Y$ for $L$ up to $k$, of fixed quantifier rank.

\end{corollary}

\begin{proof}
  
  Follows from the fact that deciding $A_k(h) \in H(L)$ is equivalent to checking $A_k(h) |= @Y$, since $A_k(h)$
  has length at most $k$, Theorem~\ref{thm:counting:alg}, and 
  Lemma~\ref{lemma:satisfaction}.
\end{proof}


As an online monitor program $P_k$ for history approximation,
Algorithm~\ref{alg:counting} effectively reduces the approximate refinement
checking problem between $L_1$ and $L_2$ to a reachability problem on the
composition $P_k \x L_1$: as $P_k$ tracks the counting representation $\pi =
\Pi(A_k(H(e)))$ of the current execution $e$'s approximation, we must only
verify whether $\pi$ satisfies the operation-counting formula $@Y$ for $L_2$ up
to $k$.

\begin{corollary}

  Given $@Y$ an operation counting formula for $L_2$ up to $k$, the approximate
  inclusion $\exists h.\ A_k(h) \in H(L_1) \setminus H(L_2)$ is equivalent to
  the safety verification problem
  \begin{align*}
    (P_k \x L_1) |= \Box @Y
  \end{align*}
  where $@Y$ is interpreted over $P_k$'s counting representation $\pi$.

\end{corollary}
