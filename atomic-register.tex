%!TEX root = appendix.tex

\section{Atomic Registers}
\label{sec:registers}

\begin{figure*}[h]
{\small
\begin{align*}
\pred{wellFormed}(x_0,\ldots,x_{k},{i_{0}},\ldots,{i_{k'}}) =\ 
  &\exists k'' \in [k].\ \exists {j_{0}}<\ldots<{j_{k''}}.\\
  &\exists u_{j_{0}},v_{j_{0}},\ldots,
    u_{j_{k''}},v_{j_{k''}}.\\
 &\exists z_{i_{0}},t_{i_{0}},\ldots,z_{i_{k'}},t_{i_{k'}}
   \in [k].\\
& \forall i,j \in [k].\ 
  \#({x_a},i,j) \geq 
  |\set{m : (z_{i_{m}},t_{i_{m}}) = (i,j)}| + \\
&  \qquad\qquad|\set{m : 
    \pred{SameVal}({x_a},x_{j_m}) \land
    (u_{j_{m}},v_{j_{m}}) = (i,j) }| \land\\
& \forall m \in [k''].\ {j_{m}} \in [u_{j_m},v_{j_m}]
\land\\
& \forall m \in [k'].\ 
  {i_{m}} \in [z_{i_{m}},t_{i_{m}}] \land\\
& \forall i \in [k].\ ((\forall m.\ i != j_m) \implies 
\exists m .\ \pred{SameVal}(x_{j_m},i) \land j_m < i \land 
\pred{totalWrites}(j_{m+1},i) = 0)
\end{align*}

\begin{align*}
\pred{covered}_{x_a}(x_0,\ldots,x_{k},{i_{0}},\ldots,{i_{k'}}) =\ 
  & 
    \forall r,i,j.\ \pred{Read}(x) \land \pred{SameVal}({x_a},x) \land \#(r,i,j) > 0 
    \implies\\
  & \exists m \in [i-1,j].\ \pred{SameVal}(r,x_{m}) \lor\\
  & \exists m \in [1,k'].\ {i_{m}} \in [i,j]
\end{align*}

\begin{align*}
@Y_\mathsf{reg}^k =\ 
  &\exists x_0,\ldots,x_{k}.\ 
    \bigwedge_{i \in [k]} \pred{Write}(x_i) \land\\
  &\forall {x_a}.\ \pred{Write}({x_a}) \implies
  \exists k' \leq k.\ 
  \exists {i_{0}}<\ldots<{i_{k'}} \in [k].\\ 
  &\pred{wellFormed}(x_0,\ldots,x_{k},{i_{0}},\ldots,{i_{k'}}) \land\\
  &\pred{covered}_{x_a}(x_0,\ldots,x_{k},{i_{0}},\ldots,{i_{k'}})
\end{align*}
}
\caption{The formula $@Y_\mathsf{reg}^k$ representing $L_\mathsf{reg}$ up to $k$. 
The names of the predicates over operation labels are capitalized while the 
names of the sub-formulas of $@Y_\mathsf{reg}^k$ start with lower case. The 
predicates are defined as follows: 
(1) ${\sf Read}(x)$ holds for any $x={\sf read}=>\_$  
(2) ${\sf Write}(x)$ holds for any $x={\sf write}(\_)=>\_$  
(3) ${\sf SameVal}(x,y)$ holds for read or writes reading/writing the same 
value. Moreover, we denote by ${\sf totalWrites}(i,j)$ the total number of 
writes which start at or after $i$, and finish at or before $j$.
}
\label{fig:register}
\end{figure*}


We define the OCL formula $@Y_\mathsf{reg}^k$ representing the atomic register
up to some bound $k$. The atomic register, denoted by $L_\mathsf{reg}$, is
defined as usual: the methods are $\<read>$ and $\<write>(\cdot)$, and the
kernel consists of all sequential executions where every read returns the last
written value (or the initial value $0$ if it is not preceded by a write).
Formally, $\ker E(L_\mathsf{reg})$ is the set of all sequential executions with
only completed operations such that the return value of any $\<read>$
invocation $o$ is the value $v$, where $\<write>(v)$ is the last $\<write>$
invocation before $o$, or $0$, if $o$ is not preceded by a $\<write>$
invocation. Thus, a history $h$ belongs to $H(L_\mathsf{reg})$ iff it is weaker
than the history of a sequential execution $e$ of $L_\mathsf{reg}$, i.e.,
$h\preceq H(e)$.

In the following, we give a non-deterministic procedure for deciding if a 
history $h$ belongs to $H(L_\mathsf{reg})$, which is used to define 
$@Y_\mathsf{reg}^k$\footnote{\citet{journals/siamcomp/GibbonsK97} proves that this 
problem is NP-complete but using a different non-deterministic procedure for 
showing membership to NP.}.

% Note that $H(L_\mathsf{reg})$ contains all histories where every completed $\<read>=>a$ 
% operation\footnote{Pending $\<read>$ operations can be ignored: a history $h$ 
% belongs to  iff the history $h'$ obtained from $h$ by deleting pending read 
% operations belongs to $H(L_\mathsf{reg})$} 
% can be mapped injectively to an overlapping $\<write>(a)=>\_$ operation. 
% Indeed, for any total ordering on the $\<write>$ operations of such a history 
% $h$ that doesn't contradict the ordering constraints in $h$, one can insert 
% every completed $\<read>=>a$ operation right after the $\<write>(a)=>\_$ 
% operation given by the injective mapping. We then obtain a sequential execution
% $e\in E(L_\mathsf{reg})$ with $h\preceq H(e)$. 

Let $h$ be a history of $H(L_\mathsf{reg})$ and $I:O->[n]^2$ be its canonical 
representation. The procedure for deciding if a history $h$ belongs to 
$H(L_\mathsf{reg})$ works as follows. First, for each $i \in [n]$, it guesses a 
value $x_i$ which is going to be the last value written before the end 
of interval $[0,i]$.
Then, for each value $a$, it guesses which intervals are going to contain
linearization points of $\<write>(a)=>\_$ operations. Finally, it checks
that every $\<read>=>a$ can either read from one of the \emph{last write}
or from one of the $\<write>(a)=>\_$ operations just mentioned.

More formally:
\begin{enumerate}
  \item 
    for each $i \in [n]$, guess a value $x_i$ corresponding to 
    the last written value before the end of the interval $[0,i]$
  \item
    then, for each value $a$, guess $n' \leq n$ and
    ${i_{0}},\ldots,{i_{n'}}$ representing the intervals containing
    linearization points of $\<write>(a)=>\_$ operations
  \item
    finally, check that for each completed $\<read>=>a$ operation $o$ with 
    $I(o)=[i,j]$, one of the following holds:
  \begin{itemize}
    \item 
      one of the last written value $x_{i-1},\ldots,x_j$ is $a$
    \item 
      one of ${i_{m}}$ is in $[i,j]$
  \end{itemize}
\end{enumerate}

The formula $@Y_\mathsf{reg}^k$ representing the atomic register up to $k$ is given 
in Figure~\ref{fig:register}. It is a translation of the above decision 
procedure in OCL. 
To simplify the formula, we use some notations such as the set comprehension 
and the quantification over bounded integers, which can easily be translated 
to conjunctions and disjunctions. 
Without loss of generality, we assume that our history always starts 
with a $\<write>(0)=>\_$ operation (before all other operations) to avoid
edge cases.

The predicate $\pred{wellFormed}(x_0,\ldots,x_{k},{i_{0}},\ldots,{i_{k'}})$ is the 
most difficult part of the formula. Informally, it simply states that there are 
enough writes in the history with the same value as ${x_a}$ in order to satisfy 
everything which was guessed so far. When translating this idea into a formula, 
we get a fairly long formula which, in english, goes as follows:
There exist $k''$ writes (filling the role of last written value for each 
interval) such that:
\begin{itemize}
\item 
  the number of writes in the trace is greater than the number of writes needed
\item
  the writes are well-placed
\item
  (last line of the formula) if, for some interval $i$, we didn't choose a 
  write (among the $k''$ writes) to fill the role of last written value, it 
  means we used the same write as a previous interval (namely, the one used for
  the existentially quantified $x_{j_{m}}$), and that there must not exist
  a write in between.
\end{itemize}

The predicate $\pred{covered}_{x_a}(x_0,\ldots,x_{k},{i_{0}},\ldots,{i_{k'}})$ is a
simple translation of the last item of the procedure, and ensured that all 
read's have a write to read from.
