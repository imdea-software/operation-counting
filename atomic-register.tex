%!TEX root = draft.tex

\section{Atomic Registers}

We construct a formula $\Psi_{reg}(K)$ characterizing  
the counting representations in $\Pi_K(\alert{L_{reg}})$. 
This formula is based on a characterization of canonical representations of 
histories of $\alert{L_{reg}}$ described hereafter.

Let $h=\tup{O,<,f}$ be a history of $\alert{L_{reg}}$ and $I$ its canonical representation. By definition,
$h$ is weaker than a sequential history $h'$, i.e., there exists a total order $h'$ 
on the operations in $h$ consistent with $h$'s order relation such that the last write before 
a $\<read>(a)$ operation is $\<write>(a)$. 
%Deciding if $h$ is a history of $\alert{L_{reg}}$
%is an NP-complete problem~\cite{journals/siamcomp/GibbonsK97}. Intuitively, this holds because
The difficulty in characterizing such histories $h$ (or their canonical representations) 
is that for every read operation $\<read>(a)$, there may exist several $\<write>(a)$ operations 
finishing before or overlapping with $\<read>(a)$ in $h$, that could be chosen to be the last 
$\<write>$ operation before $\<read>(a)$ in $h'$. \todo{For example,}

We define an equivalence relation on read operations of $h$ such that $h$ is weaker than a sequential history $h'$, where 
the last writes before two equivalent reads are always mapped by $I$ to the same interval. Roughly, this equivalence relation 
relates two reads mapped by $I$ to the same interval for which the operations that write the same value are again 
mapped by $I$ to the same intervals. \todo{For example,}

Formally, for every $\<read>(a)$ operation $o$ of $h$, let $\mathcal{I}_{\<w>}(o)$ be the set of intervals associated by $I$ to
$\<write>(a)$ operations, that finish before or overlap with $o$, i.e.,
\[
\mathcal{I}_{\<w>}(o)=\set{I(o'):\mbox{$o'$ a $\<write>(a)$ operation and $\lnot o< o' $}}.
\]

\todo{For example,}
The write operations whose intervals are in $\mathcal{I}_{\<w>}(o)$
are the only possible candidates for being the last write before $o$ in a sequential history stronger than $h$.
Let $\sim$ be an equivalence relation on read operations such that
\[
o_1\sim o_2\mbox{ iff }I(o_1)=I(o_2)\mbox{ and }\mathcal{I}_{\<w>}(o_1)=\mathcal{I}_{\<w>}(o_2).
\]

We prove that $h$ is a history of $\alert{L_{reg}}$ iff the history $h_1$ obtained from $h$ by 
keeping only one representative for each equivalence class of $\sim$ is a history of $\alert{L_{reg}}$.
Clearly, the former implies the latter. For the other direction, one can use a sequential history $h_1'$ stronger than $h_1$
to build a sequential history $h'$ stronger than $h$. Let $o_1$ be a $\<read>(a)$ operation of $h_1$ and $o_1'$ the 
$\<write>(a)$ operation,which is the last write operation before $o_1$ in $h_1'$. Now, consider a $\<read>(b)$ operation $o_2$ of $h_1$,
which is equivalent to $o_1$, i.e., $I(o_1)=I(o_2)$ and there exists a $\<write>(b)$ operation $o_2'$ of $h_1$ such that $I(o_2')=I(o_1')$.
Then, one can build a history $h'=\tup{O,<',f}$ stronger than $h$ where $o_1'<o_1<o_2'<o_2$.

This result is stated in the next lemma. The history $h_1'$ is encoded by a mapping $f$ that associates to each equivalence 
class $e$ of $\sim$ an interval in the image of $I$. Thus, $f(e)$ is the interval of the last write
operation occurring before the representative of $e$ in $h_1'$.

For every equivalence class $e$ of $\sim$, let $I_{\<r>}(e)$ be the interval associated by $I$ to the
read operations in $e$ (which by definition is unique) and $\mathcal{I}_{\<w>}(e)=\mathcal{I}_{\<w>}(o)$, for some $o\in e$. 
Also, let $EC(\sim)$ be the set of non-empty equivalence classes of $\sim$.


\begin{lemma}\label{lemma:register}
Let $h$ be a history and $I$ its canonical representation. Then, 
$h$ is a history of $\alert{L_{reg}}$ iff the following formula holds 
\[
\begin{array}{l}
\exists f:EC(\sim) -> img(I)\big( \\
\hspace{2.7cm}\forall e.\ f(e)\in \mathcal{I}_{\<w>}(e)\ \land \\
\hspace{2.7cm}\forall e\neq e'.\ \big(\neg (f(e) < f(e') < I_{\<r>}(e))\ \land  \\
\hspace{4.05cm} \neg (f(e) < I_{\<r>}(e') < I_{\<r>}(e)) \big) \\
\hspace{2.7cm}\big)
\end{array}
\]
\end{lemma}

\begin{proof}

\todo{}

\end{proof}

In the formula above, the second row states that $f(e)$ is the interval of a write operation,  
which writes a value read by one of the read operations in $e$. The last two rows state that 
it is not possible to separate $f(e)$ from the interval of the reads in $e$ using $f(e')$ or $I_{\<r>}(e')$, for some
other equivalence class $e'$. Otherwise, in the history $h_1'$ encoded by $f$, the write operation represented
by $f(e)$ is not the last write operation before the representative of $e$.

Based on Lemma~\ref{lemma:register}, we prove that there exists an effectively computable formula 
$\Psi_{reg}(K)$ characterizing the counting representations in $\Pi_K(\alert{L_{reg}})$.
This result is based on the fact that for every history $h$ of $\alert{L_{reg}}$,
the non-empty equivalence classes of $\sim$ can be determined from the counting representation of $h$.
The formula $\Psi_{reg}(K)$ contains several quantifiers but the satisfaction problem can still be solved
in polynomial time.

\begin{theorem}\label{th:register}

There exists an effectively computable formula 
$\Psi_{reg}(K)$ such that $[\Psi_{reg}(K)]=\Pi_K(\alert{L_{reg}})$.
The satisfaction problem for $\Psi_{reg}(K)$ can be solved in polynomial time.

\end{theorem}

\begin{proof}

\todo{}

\end{proof}

Theorem~\ref{th:register} implies that deciding if a history of length at most $K$ belongs to $\alert{L_{reg}}$
is in polynomial time. Note that the same problem is NP-complete for arbitrary histories~\cite{journals/siamcomp/GibbonsK97}.

%Deciding if a history of a concurrent register, of bounded length,
%is correct w.r.t. the atomic register, i.e., it is also a history of the atomic
%register, is decidable in polynomial time. The same problem for arbitrary
%histories is NP-complete since it is equivalent to deciding if an execution of
%the concurrent register is linearizable w.r.t. a sequential register (by
%Proposition~\ref{prop:atom}).
%
%\begin{theorem}
%
%  Let $h$ be a history of a concurrent register, of bounded length. Deciding if
%  $h\in At[Reg]$ is polynomial time.
%
%\end{theorem}
%
%\begin{proof}
%
%  Let $h$ be a history of a concurrent register of length $K$.
%
%\end{proof}