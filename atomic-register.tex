%!TEX root = draft.tex

\section{Atomic Registers}
\label{sec:registers}

\begin{figure*}[t]
{\small
\begin{align*}
\<total>(x,i,j) &= \hspace{-3mm}\sum_{i\le i'\le j'\le j} \#(x,i',j'), \quad
\<isWrite>(x,\_,i) = \<Write>(x)\land \sum_{i'\le i} \#(x,i',i) > 0, \quad
\<noWrites>(i,j) =\forall x.\ \<Write>(x)\implies \<total>(x,i,j) = 0, \
\end{align*}
\begin{align*}
\<amongLastWr>(x_i,i)& = \bigvee_{i'\in [i-1]} \big( \<isWrite>(x_i,\_,i')\land \<noWrites>(i'+1,i-1)\big) % \land \neg \<noWrites>(i'+1,i)\big),
\end{align*}
\begin{align*}
\<isRead>(y,i,j) = \<Read>(y)\land \<Completed>(y)\land \#(y,i,j) > 0, \quad\quad\quad
\<overlaps>(z,i,j) = \sum_{i'\leq j,j'\geq i} \#(z,i',j') > 0,\\
\<noWrites>_{\neq 0}(i,j) =\forall x.\ \big(\<Write>(x)\land \neg \<Write>_0(x)\big)\implies \<total>(x,i,j) = 0\quad\quad\quad 
\<init>(y,i) = \<Read>_0(y)\land \<noWrites>_{\neq 0}(0,i-1)
\end{align*}

\begin{align*}
\regform{k} = \exists x_0,\ldots,x_{k-1}. \bigwedge_{i\in [k-1]} \<amongLastWr>(x_i,i)\land\ \  \forall y. \bigwedge_{i,j\in [k]}\<isRead>(y,i,j) \implies
\big( \<init>(y,i)\vee \<Rf>(y,x_{i-1}) \vee \exists z.\ (\<Rf>(y,z) \land \<overlaps>(z,i,j))\big)
\end{align*}
}
\caption{The formula ${\tt register}[k]$ representing $L_{\tt reg}$ up to $k$. The names of the predicates over operation labels are capitalized while the names of the sub-formulas of ${\tt register}[k]$ start with lower case. The predicates are defined as follows: (1) ${\tt Read}(x)$ holds for any $x={\tt read}=>\_$  (2) ${\tt Write}(x)$ holds for any $x={\tt write}(\_)=>\_$  (3) ${\tt Completed}(x)$ holds for any $x={\tt write}(\_)=>v$ or $x={\tt read}(\_)=>v$ with $v\neq\bottom$
(4) ${\tt Read}_0(x)$ holds for $x={\tt read}=>0$ (5) ${\tt Write}_0(x)$ holds for any $x={\tt write}(0)=>\_$ and (6) ${\tt Rf}(x,y)$ holds for any pair $x={\tt read}=>v$ and $y={\tt write}(v)=>\_$.
}
\label{fig:register}
\end{figure*}

We define the $\ocl$ formula $\regform{k}$ representing the atomic register up to some bound $k$. The atomic register, denoted by $\lreg$, is defined as usual: the methods are $\<read>$ and $\<write>(\cdot)$, and the kernel consists of all sequential executions where every read returns the last written value (or the initial value $0$ if it is not preceded by a write). 
%Formally, $\ker E(L)$ is the set of all sequential executions with only completed operations such that the return value of any $\<read>$ invocation $o$ is the value $v$, where $\<write>(v)$ is the last $\<write>$ invocation before $o$, or $0$, if $o$ is not preceded by a $\<write>$ invocation.
Thus, a history $h$ belongs to $H(\lreg)$ iff it is weaker than the history of a sequential execution $e$ of $\lreg$, i.e.,  $h\preceq H(e)$.

In the following, we give a non-deterministic procedure for deciding if a history $h$ belongs to $H(\lreg)$, which is used to define $\regform{k}$\footnote{\citet{journals/siamcomp/GibbonsK97} proves that this problem is NP-complete but using a different non-deterministic procedure for showing membership to NP.}.

Note that $H(\lreg)$ contains all histories where every completed $\<read>=>a$ operation\footnote{Pending $\<read>$ operations can be ignored: a history $h$ belongs to  iff the history $h'$ obtained from $h$ by deleting pending read operations belongs to $H(\lreg)$} 
overlaps with a $\<write>(a)=>\_$ operation. Indeed, for any total ordering on the $\<write>$ operations of such a history $h$, that doesn't contradict the ordering constraints in $h$, one can insert every completed $\<read>=>a$ operation right after a $\<write>(a)=>\_$ operation in order to obtain a history of a sequential execution $e\in E(\lreg)$ with $h\preceq H(e)$. Therefore, a history $h$ may not belong to $H(\lreg)$ only if it contains completed $\<read>=>a$ operations with no overlapping $\<write>(a)=>\_$ operations. 

Now, let $h$ be a history of $H(\lreg)$ %and $I:O->[n]^2$ its canonical representation. 
and $o$ a completed $\<read>=>a$ operation with no overlapping $\<write>(a)=>\_$ operations. Then, one of the following must be true: 
\begin{itemize}
	\item $a=0$ and there are no $\<write>$ operations, that finish before $o$ and write a different value, 
	\item there exists a $\<write>(a)=>\_$ operation $o'$ finishing before $o$, such that no other $\<write>$ operation starts after $o'$ and finishes before $o$.
\end{itemize}
If we consider two $\<read>$ operations $o_1$ and $o_2$ in $h$, that have no overlapping writes (on the corresponding value), then the $\<write>$ operations $o_1'$ and $o_2'$ finishing before $o_1$, resp., $o_2$, may be the same. However, $o_1'$ and $o_2'$ must be the same when $o_1$ and $o_2$ have the same past in $h$. Indeed, the latter implies that the set of $\<write>$ operations occurring before $o_1$ and resp., $o_2$ in some sequential execution $e$ with $h\preceq H(e)$ is the same and consequently, $o_1$ and $o_2$ return a value written by the same $\<write>$ operation (the last $\<write>$ operation before $o_1$ and $o_2$ in $e$).

The procedure for deciding if a history $h$ belongs to $H(\lreg)$ works as follows. Roughly, for each past $p$ in $h$, it guesses a $\<write>$ operation $w_p$, that should write the value returned by all reads $o$ with $\<past>(o)=p$ and no overlapping writes, and then, it checks if this is indeed the case. Formally, the procedure works on the canonical representation $I:O->[n]^2$ of $h$, where a past in $h$ is represented by an index $i\in [n]$: 
%and if read operations have overlapping writes.
\begin{enumerate}
	\item for each $i\in [n]$, guess a $\<write>$ operation $w_i$ such that: 
	\begin{itemize}
%		\item $I(w_i)=[\_,i-1]$ or 
		\item $w_i$ finishes before every $\<read>$ operation $o_r$ with $I(o_r)=[i,\_]$ and no other $\<write>$ operation starts after $w_i$ and finishes before such an $o_r$, i.e., $I(w_i)=[\_,i']$ and %$i'$ is the maximum value such that 
		there exists no $\<write>$ operation $o$ with $I(o)\subseteq [i'+1,i-1]$ 
	\end{itemize}
	\item check that for each completed $\<read>=>a$ operation $o$ with $I(o)=[i,j]$, one of the following holds:
	\begin{itemize}
		\item $a=0$ and no $\<write>(b)$ operation with $b\neq 0$ finishes before $o$, i.e., there exists no $\<write>(b)$ operation $o'$ with $I(o')\subseteq [0,i-1]$,
		\item $a$ is the value written by $w_{i-1}$,  
		\item there exists a $\<write>(a)$ operation overlapping with $o$, i.e., there exists a $\<write>(a)$ operation $o'$ with $I(o')\cap [i,j]\neq\emptyset$.
	\end{itemize}
\end{enumerate}

The formula $\regform{k}$ representing the atomic register up to $k$ is given in Figure~\ref{fig:register}. It is a mere translation of the above decision procedure in $\ocl$. The non-deterministic guessing is translated to a bounded number of existential quantifiers (since it describes histories of bounded interval length).

