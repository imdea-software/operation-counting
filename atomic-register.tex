%!TEX root = draft.tex

\section{Atomic Registers}

We show that there exists an effectively computable formula $\Psi_{reg}[K]$ representing
the counting representations in $\Pi_K(\alert{L_{reg}})$.

Thus, let $h$ be a history of $\alert{L_{reg}}$. The definition of $\alert{L_{reg}}$ implies that
$h$ is weaker than a sequential history $h'$, i.e., that there exists a total order
on the operations in $h$ consistent with $h$'s order relation such that the last write before 
every read operation $\<read>(a)$ is $\<write>(a)$. In the history $h$, this $\<write>(a)$ operation finishes
before or it is overlapping with 


 In the following, we give a property 
of the canonical representation of $h$ which is equivalent to this fact.


deciding if a history of a concurrent register, of bounded length,
is correct w.r.t. the atomic register, i.e., it is also a history of the atomic
register, is decidable in polynomial time. The same problem for arbitrary
histories is NP-complete since it is equivalent to deciding if an execution of
the concurrent register is linearizable w.r.t. a sequential register (by
Proposition~\ref{prop:atom}).

\begin{theorem}

  Let $h$ be a history of a concurrent register, of bounded length. Deciding if
  $h\in At[Reg]$ is polynomial time.

\end{theorem}

\begin{proof}

  Let $h$ be a history of a concurrent register of length $K$.

\end{proof}