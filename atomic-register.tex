%!TEX root = draft.tex

\section{Atomic Registers}

We construct a formula $\Psi_{reg}(K)$ characterizing  
the counting representations in $\Pi_K(\alert{L_{reg}})$. 
This formula is based on a characterization of canonical representations of 
histories of $\alert{L_{reg}}$ described hereafter.

Let $h=\tup{O,<,f}$ be a history of $\alert{L_{reg}}$ and $I$ its canonical representation. By definition,
$h$ is weaker than a sequential history $h'$, i.e., there exists a total order $h'$ 
on the operations in $h$ consistent with $h$'s order relation such that the last write before 
a $\<read>(a)$ operation is $\<write>(a)$. 
%Deciding if $h$ is a history of $\alert{L_{reg}}$
%is an NP-complete problem~\cite{journals/siamcomp/GibbonsK97}. Intuitively, this holds because
The difficulty in characterizing such histories $h$ (or their canonical representations) 
is that for every read operation $\<read>(a)$, there may exist several $\<write>(a)$ operations 
finishing before or overlapping with $\<read>(a)$ in $h$, that could be chosen to be the last 
$\<write>$ operation before $\<read>(a)$ in $h'$. \todo{For example,}

We define an equivalence relation on read operations of $h$ such that $h$ is weaker than a sequential history $h'$, where 
the last writes before two equivalent reads are always mapped by $I$ to the same interval. Roughly, this equivalence relation 
relates two reads mapped by $I$ to the same interval for which the operations that write the same value are again 
mapped by $I$ to the same intervals. \todo{For example,}

Formally, for every $\<read>(a)$ operation $o$ of $h$, let $I_{\<w>}(o)$ be the set of intervals associated by $I$ to
$\<write>(a)$ operations, that finish before or overlap with $o$, i.e.,
\[
I_{\<w>}(o)=\set{I(o'):\mbox{$o'$ a $\<write>(a)$ operation and $\lnot o< o' $}}.
\]

\todo{For example,}
The write operations whose intervals are in $I_{\<w>}(o)$
are the only possible candidates for being the last write before $o$ in a sequential history stronger than $h$.
Let $\sim$ be an equivalence relation on read operations such that
\[
o_1\sim o_2\mbox{ iff }I(o_1)=I(o_2)\mbox{ and }I_{\<w>}(o_1)=I_{\<w>}(o_2).
\]

We prove that $h$ is a history of $\alert{L_{reg}}$ iff for every completed $\<read>(a)$ operation $o$ there exists
a $\<write>(a)$ operation which finishes before or overlaps with $o$ and the history $h_1$ obtained from $h$ by 
keeping only one representative for each equivalence class of $\sim$ is a history of $\alert{L_{reg}}$.
Clearly, the former implies the latter. For the other direction, one can use a sequential history $h_1'$ stronger than $h_1$
to build a sequential history $h'$ stronger than $h$. Let $o_1$ be a $\<read>(a)$ operation of $h_1$ and $o_1'$ the 
$\<write>(a)$ operation,which is the last write operation before $o_1$ in $h_1'$. Now, consider a $\<read>(b)$ operation $o_2$ of $h_1$,
which is equivalent to $o_1$, i.e., $I(o_1)=I(o_2)$ and there exists a $\<write>(b)$ operation $o_2'$ of $h_1$ such that $I(o_2')=I(o_1')$.
Then, one can build a history $h'=\tup{O,<',f}$ stronger than $h$ where $o_1'<o_1<o_2'<o_2$.

This result is stated in the next lemma. The history $h_1'$ is encoded by a mapping $f$ that associates to each equivalence 
class $e$ of $\sim$ an interval in the image of $I$. Thus, $f(e)$ is the interval of the last write
operation occurring before the representative of $e$ in $h_1'$.

For every equivalence class $e$ of $\sim$, let $\mathcal{I}_{\<r>}(e)$ be the interval associated by $I$ to the
read operations in $e$ (which by definition is unique) and $\mathcal{I}_{\<w>}(e)=I_{\<w>}(o)$, for some $o\in e$. 
Also, let $E_\sim$ be the set of non-empty equivalence classes of $\sim$.


\begin{lemma}\label{lemma:register}
Let $h=\tup{O,<,f}$ be a history and $I$ its canonical representation. Then, 
$h$ is a history of $\alert{L_{reg}}$ iff the following two formulas hold:
\[
\begin{array}{l}
\bullet\ \forall a\,\forall o\in O.\ f(o)=(\<read>(a),\<true>) \\[.5mm]
\hspace{1.7cm}\Rightarrow \exists o'\in O.\  f(o')=(\<write>(a),\_)\land \neg I(o) < I(o')\\[1mm]
\bullet\ \exists\ f:E_\sim -> img(I)\big( \\[.5mm]
\hspace{2.3cm}\forall\ e\in E_\sim.\ f(e)\in \mathcal{I}_{\<w>}(e)\ \land \\[.5mm]
\hspace{2.3cm}\forall\ e\neq e'\in E_\sim.\ \big(\neg (f(e) < f(e') < \mathcal{I}_{\<r>}(e))\ \land  \\[.5mm]
\hspace{4.55cm} \neg (f(e) < \mathcal{I}_{\<r>}(e') < \mathcal{I}_{\<r>}(e)) \big) \\
\hspace{2.7cm}\big)
\end{array}
\]
\end{lemma}

\begin{proof}

\todo{}

\end{proof}

The first formula states that for every completed $\<read>(a)$ operation $o$ there exists
a $\<write>(a)$ operation whose interval is either before or overlaps with $I(o)$.
In the second formula, the second row states that $f(e)$ is the interval of a write operation,  
which writes a value read by one of the read operations in $e$. The last two rows state that 
it is not possible to separate $f(e)$ from the interval of the reads in $e$ using $f(e')$ or $I_{\<r>}(e')$, for some
other equivalence class $e'$. Otherwise, in the history $h_1'$ encoded by $f$, the write operation represented
by $f(e)$ is not the last write operation before the representative of $e$.

Based on Lemma~\ref{lemma:register}, we prove that there exists an effectively computable formula 
$\Psi_{reg}(K)$ characterizing the counting representations in $\Pi_K(\alert{L_{reg}})$.
This result is based on the fact that (1) the first formula in Lemma~\ref{lemma:register} can be rewritten
in terms of counting representations and (2) for every history $h$, the non-empty equivalence classes of 
$\sim$ can be determined from the counting representation of $h$.
The formula $\Psi_{reg}(K)$ contains several quantifiers but the satisfaction problem can still be solved
in polynomial time.

\begin{theorem}\label{th:register}

There exists an effectively computable formula 
$\Psi_{reg}(K)$ such that $[\Psi_{reg}(K)]=\Pi_K(\alert{L_{reg}})$.
The satisfaction problem for $\Psi_{reg}(K)$ can be solved in polynomial time.

\end{theorem}

\begin{proof}

Basically, $\Psi_{reg}(K)$ is obtained by rewriting the first formula in Lemma~\ref{lemma:register} in terms
of counting representations and by adding a disjunction in front of the second formula 
(called $\Psi$ in the following), over all equivalence relations $\sim$. Actually, $\Psi$ characterizes only the intervals
associated to some non-empty equivalence class $e$, i.e., $\mathcal{I}_{\<r>}(e)$ and $\mathcal{I}_{\<w>}(e)$. Hence, 
the disjunction ranges over sets of pairs interval -- sets of intervals and since the goal is to characterize 
histories of length at most $K$ the limits of the intervals are bounded by $K$.

%Note that the canonical representation of a history of length at most $K$ uses intervals whose limits are bounded by $K$. 
Thus, let $\mathcal{I}_{K}$ be the set of intervals $[i,j]$ with $0\leq i\leq j\leq K$ and 
$\mathcal{E}_K=\mathcal{I}_{K}\times 2^{\mathcal{I}_{K}}$ the set of pairs interval -- sets of intervals.

The formula $\Psi_{reg}(K)$ is defined by:
\[
\begin{array}{l}
\forall a.\ (\exists \iota\in \mathcal{I}_{K}.\ \counting{\<read>(a)}{\iota}>0)\Rightarrow \\[.5mm]
\hspace{2cm}\big(\exists \iota'\in \mathcal{I}_{K}.\ (\counting{\<write>(a)}{\iota'} > 0 \land \neg \iota < \iota')\big)\\[1mm]
\land \\[-3mm]
\end{array}
\]
\[
\bigvee_{E\subseteq \mathcal{E}_K} \Psi[E_\sim\mapsto E]\land \forall e\in \mathcal{E}_K.\ e\in E \Leftrightarrow NonEmpty(e),
\]
where $\Psi[E_\sim\mapsto E]$ denotes the formula $\Psi$ where $E_\sim$ is substituted by $E$ and 
$NonEmpty(e)$ states that the equivalence class $e$ is non-empty, i.e., there exists a read operation $o$ with $I(o)=e|_1$
and $I_{\<w>}(o)=e|_2$ ($e|_1$, resp., $e|_2$, is the projection of $e$ on the first, resp., second, component).
Formally, $NonEmpty(e)$ is defined by
\[
\begin{array}{l}
\exists\ a.\ \big(\ \counting{\<read>(a)}{e|_1} > 0\ \land \\[.5mm]
\hspace{.9cm}\forall\ \iota\in \mathcal{I}_{K}.\ \iota\in e|_2 \Leftrightarrow \counting{\<write>(a)}{\iota} > 0\ \land \\[.5mm]
\hspace{.9cm}\exists\ \iota \in e|_2\land \forall\ \iota\in \mathcal{I}_{K}.\ \iota\in e|_2 \Rightarrow \neg e|_1 < \iota \big)
\end{array}
\]

The first two rows state that there exists a $\<read>(a)$ operation whose interval is exactly the first component in $e$
and one $\<write>(a)$ operation in each interval from the second component of $e$. The last row states that the second
component of $e$ is non-empty (i.e., there exists a write operation  the intervals
in the second component of $e$ (corresponding to write operations) are not ordered after the first component of $e$
(this is to say that the intervals of the write operations are before or overlapping with the interval of the read operations).

The fact that $\Psi_{reg}(K)$ characterizes precisely the counting representations in $\Pi_K(\alert{L_{reg}})$ is 
a direct consequence of Lemma~\ref{lemma:register}. The satisfaction problem for this formula is in polynomial time
because %(1) the nesting of quantifiers is bounded and (2) 
the quantifiers over values ($\forall a$ and $\exists a$) need to be instantiated only over
values $a$ for which there exists at least one instantiation of $\<read>(a)$ or $\<write>(a)$ in the input 
counting representation. Note that the range of the other quantifiers (e.g., $\forall\ \iota\in \mathcal{I}_{K}$ and 
$\forall e\in \mathcal{E}_K$) is a domain of constant size.\hfill $\Box$
\end{proof}

Theorem~\ref{th:register} implies that deciding if a history of length at most $K$ belongs to $\alert{L_{reg}}$
is in polynomial time. Note that the same problem is NP-complete for arbitrary histories~\cite{journals/siamcomp/GibbonsK97}.

%Deciding if a history of a concurrent register, of bounded length,
%is correct w.r.t. the atomic register, i.e., it is also a history of the atomic
%register, is decidable in polynomial time. The same problem for arbitrary
%histories is NP-complete since it is equivalent to deciding if an execution of
%the concurrent register is linearizable w.r.t. a sequential register (by
%Proposition~\ref{prop:atom}).
%
%\begin{theorem}
%
%  Let $h$ be a history of a concurrent register, of bounded length. Deciding if
%  $h\in At[Reg]$ is polynomial time.
%
%\end{theorem}
%
%\begin{proof}
%
%  Let $h$ be a history of a concurrent register of length $K$.
%
%\end{proof}