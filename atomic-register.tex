%!TEX root = draft.tex

\section{Atomic Registers}

We prove that there exists an effectively computable $\ocl$ formula $\Psi_{\<reg>}[k]$ representing the atomic register $L_{\<reg>}$   
up to $k$, for every $k$. 
This formula is based on a characterization of $L_{\<reg>}$ histories described hereafter.

Let $h=\tup{O,<,f}$ be a history of $L_{\<reg>}$ and $I$ its canonical representation. By definition,
$h$ is weaker than a sequential history $h'$, i.e., there exists a total order $h'$ 
on the operations in $h$ consistent with $h$'s order relation such that the last write before 
a $\<read>(a)$ operation is a $\<write>(a)$ operation, for every value $a$. 
%Deciding if $h$ is a history of $L_{\<reg>}$
%is an NP-complete problem~\cite{journals/siamcomp/GibbonsK97}. Intuitively, this holds because
%The difficulty in characterizing such histories $h$ (or their canonical representations) is that 
In general, for a $\<read>(a)$ operation $o\in O$, there may exist several $\<write>(a)$ operations 
finishing before or overlapping with $o$ in $h$, that could play the role of the last 
write operation before $o$ in $h'$. \todo{For example,}

We define an equivalence relation on read operations of $h$ such that $h$ is weaker than a sequential history $h'$, where 
the last writes before any two equivalent reads are mapped by $I$ to the same interval. Roughly, two reads are equivalent 
if they are mapped by $I$ to the same interval and the operations that write the same value are also  
mapped by $I$ to the same intervals. \todo{For example,}

More precisely, for every $\<read>(a)$ operation $o\in O$, let $I_{\<w>}(o)$ be the set of intervals associated by $I$ to
$\<write>(a)$ operations, that finish before or overlap with $o$, i.e.,
\[
I_{\<w>}(o)=\set{I(o'):\mbox{$o'$ is a $\<write>(a)$ operation and $\lnot o< o' $}}.
\]

\todo{For example,}

Note that the write operations whose intervals are in $I_{\<w>}(o)$
are the only possible candidates for being the last write before $o$ in a sequential history stronger than $h$.
Let $\sim$ be an equivalence relation on read operations such that
\[
\begin{array}{ll}
o_1\sim o_2\mbox{ iff } & I(o_1)=I(o_2)=\tup{\<read>(a),true},\mbox{ for some $a$,}\\[.5mm]
&\mbox{and } I_{\<w>}(o_1)=I_{\<w>}(o_2).
\end{array}
\]

The equivalence relation $\sim$ ignores pending read operations because $h$ is a history of $L_{\<reg>}$
iff the history obtained from $h$ by removing pending read operations is.

We prove that $h$ is a history of $L_{\<reg>}$ iff for every completed $\<read>(a)$ operation $o$ there exists
a $\<write>(a)$ operation which finishes before or overlaps with $o$ and the history $h_1$ obtained from $h$ by 
keeping only one representative from each equivalence class of $\sim$ is a history of $L_{\<reg>}$.
Clearly, the former implies the latter. For the other direction, one can use a sequential history $h_1'$ stronger than $h_1$
to build a sequential history $h'$ stronger than $h$. Let $o_1$ be a $\<read>(a)$ operation of $h_1$ and $o_1'$ the 
$\<write>(a)$ operation,which is the last write operation before $o_1$ in $h_1'$. Now, consider a $\<read>(b)$ operation $o_2$ of $h_1$,
which is equivalent to $o_1$, i.e., $I(o_1)=I(o_2)$ and there exists a $\<write>(b)$ operation $o_2'$ of $h_1$ such that $I(o_2')=I(o_1')$.
Then, one can build a history $h'=\tup{O,<',f}$ stronger than $h$ where $o_1'<o_1<o_2'<o_2$.

This result is stated in the next lemma. The existence of the sequential history $h_1'$ stronger than $h_1$ is 
equivalent to the existence of a mapping $\rf$ that associates to each equivalence 
class $e$ of $\sim$ an interval in the image of $I$ satisfying several properties. The properties of $\rf$ ensure that 
$\rf(e)$ is the interval of the last write operation occurring before the representative of $e$ in $h_1'$.

For every equivalence class $e$ of $\sim$, let $\mathcal{I}_{\<r>}(e)$ be the interval associated by $I$ to the
read operations in $e$ (which by definition is unique) and $\mathcal{I}_{\<w>}(e)=I_{\<w>}(o)$, for some $o\in e$. 
Also, let $@O$ be the set of non-empty equivalence classes of $\sim$. We give a characterization of $L_{\<reg>}$
histories %in a logic describing their canonical representations.
in a logic that contains standard first-order terms and atomic formulas, quantifiers over values read or written 
by the operations of the register, quantifiers over operations, and quantifiers over functions from $@O$ to intervals
(the order relation between intervals is defined by: $I_1<I_2$ iff $\sup I_1 < \inf I_2$)

\begin{lemma}\label{lemma:register}

Let $h=\tup{O,<,f}$ be a history and $I : O -> [n]^2$ its canonical representation. Then, 
$h$ is a history of $L_{\<reg>}$ iff the following two formulas hold:
\[
\begin{array}{l}
\bullet\ \forall a\,\forall o\in O.\ f(o)=(\<read>(a),\<true>) \\[1mm]
\hspace{1.5cm}\Rightarrow \exists o'\in O.\  f(o')=(\<write>(a),\_)\land \neg I(o) < I(o')\\[2mm]
\bullet\ \exists\ \rf:@O -> [n]^2\ \big( \\[1mm]
\hspace{2.3cm}\forall\ e\in @O.\ \rf(e)\in \mathcal{I}_{\<w>}(e)\ \land \\[1mm]
\hspace{2.3cm}\forall\ e\neq e'\in @O.\ \big(\neg (\rf(e) < \rf(e') < \mathcal{I}_{\<r>}(e))\ \land  \\[1mm]
\hspace{4.35cm} \neg (\rf(e) < \mathcal{I}_{\<r>}(e') < \mathcal{I}_{\<r>}(e)) \big) \\
\hspace{2.4cm}\big)
\end{array}
\]
%\begin{enumerate}
%	\item for every $o\in O$, if $f(o)=(\<read>(a),\<true>)$, for some $a$, then there exists $o'\in O$
%such that $f(o')=(\<write>(a),\_)$ and $\neg I(o) < I(o')$,
%\end{enumerate}

\end{lemma}

\begin{proof}

\todo{}

\end{proof}

The first formula states that for every completed $\<read>(a)$ operation $o$ there exists
a $\<write>(a)$ operation $o'$ which finishes before or overlaps with $o$.
In the second formula, the second row states that $\rf(e)$ is the interval of a write operation,  
which writes a value read by one of the read operations in $e$. The last two rows state that 
it is not possible to separate $\rf(e)$ from the interval of the reads in $e$ using $\rf(e')$ or $I_{\<r>}(e')$, for some
other equivalence class $e'$. Otherwise, the write operation represented by $\rf(e)$
cannot be the last write operation before the representative of $e$.

%Otherwise, in the history $h_1'$ encoded by $f$, the write operation represented
%by $f(e)$ is not the last write operation before the representative of $e$.

Based on Lemma~\ref{lemma:register}, we prove that there exists an effectively computable $\ocl$ formula 
$\Psi_{\<reg>}[k]$ characterizing the $L_{\<reg>}$ histories of length at most $k$.
Essentially, this is a consequence of the fact that $\ocl$ is expressive enough to capture the constraints from 
the first formula in Lemma~\ref{lemma:register} and to characterize the non-empty equivalence classes of 
$\sim$, for every history $h$ of length at most $k$.
%as an $\ocl$ formula and (2) for every history $h$,  can be characterized using an $\ocl$ formula.
%The formula $\Psi_{\<reg>}[k]$ contains several quantifiers but the satisfaction problem can still be solved
%in polynomial time.

\begin{theorem}\label{th:register}

There exists an effectively computable $\ocl$ formula 
$\Psi_{\<reg>}[k]$ representing $L_{\<reg>}$ up to $k$, for every $k$.
%The satisfaction problem for $\Psi_{\<reg>}[k]$ can be solved in polynomial time.

\end{theorem}

\begin{proof}

Roughly, $\Psi_{\<reg>}[k]$ is obtained by rewriting the first formula in Lemma~\ref{lemma:register} and 
by adding a disjunction in front of the second formula 
(called $\Phi$ in the following), over all equivalence relations $\sim$. 
More precisely, since $\Phi$ characterizes only the intervals
associated to some non-empty equivalence class $e$, i.e., $\mathcal{I}_{\<r>}(e)$ and $\mathcal{I}_{\<w>}(e)$,  
the disjunction ranges over sets of pairs interval -- sets of intervals.
Furthermore, since the goal is to characterize histories of length at most $k$, the limits of the intervals 
are bounded by $k$.

%Note that the canonical representation of a history of length at most $K$ uses intervals whose limits are bounded by $K$. 
Thus, let $\mathcal{I}_{k}$ be the set of intervals $[i,j]$ with $0\leq i\leq j\leq k$ and 
$\mathcal{E}_k=\mathcal{I}_{k}\times 2^{\mathcal{I}_{k}}$ the set of pairs interval -- sets of intervals.

Let $\Psi_{\<reg>}[k]$ be the following $\ocl$ formula:
\[
\begin{array}{l}
\forall x.\ \big(\<read>(x)\land \exists \iota\in \mathcal{I}_{k}.\  \#(x,\<true>,\iota)>0)\big)\Rightarrow \\[1mm]
\hspace{.6cm}\big(\exists y.\ \mathsf{Rd\-Wr}(x,y) %\\[1mm]
%\hspace{2.8cm}
\land\ \exists \iota'\in \mathcal{I}_{k}.\ (\#(y,\_,\iota') > 0 \land \neg \iota < \iota')\big)\\[1mm]
\land \\[-3mm]
\end{array}
\]
%\[
%\begin{array}{l}
%\forall a.\ (\exists \iota\in \mathcal{I}_{K}.\ \counting{\<read>(a)}{\iota}>0)\Rightarrow \\[.5mm]
%\hspace{2cm}\big(\exists \iota'\in \mathcal{I}_{K}.\ (\counting{\<write>(a)}{\iota'} > 0 \land \neg \iota < \iota')\big)\\[1mm]
%\land \\[-3mm]
%\end{array}
%\]
\[
\bigvee_{\mathcal{E}\subseteq \mathcal{E}_k} \Phi[@O\gets \mathcal{E}]\land \bigwedge_{e\in \mathcal{E}} NonEmpty(e)
%\forall e\in \mathcal{E}_k.\ e\in E \Leftrightarrow NonEmpty(e),
\]
where $\Phi[@O\mapsto \mathcal{E}]$ denotes the formula $\Phi$ where $@O$ is substituted by $\mathcal{E}$ and 
$NonEmpty(e)$ states that $e$ represents a non-empty equivalence class of $\sim$, i.e., there exists a read operation $o$ with $I(o)=e|_1$
and $I_{\<w>}(o)=e|_2$ ($e|_1$, resp., $e|_2$, is the projection of $e$ on the first, resp., second, component).
The formula $\Psi_{\<reg>}[k]$ contains two predicates over methods, $\<read>(x)$, that holds for every method $x=\<read>(a)$, for some $a$,
and $\mathsf{Rd\-Wr}(x,y)$, that holds for every two methods $x=\<read>(a)$ and $y=\<write>(a)$, for some $a$. Also, for readability,
it contains quantifiers over intervals in $\mathcal{I}_{k}$. Since $k$ is fixed, the existential, resp., universal quantifier over intervals 
can be transformed in a disjunction, resp, conjunction, over all pairs of integer constants $0\leq i\leq j\leq k$.


Formally, $NonEmpty(e)$ is defined by
\[
\begin{array}{l}
\exists x.\ \big(\ \<read>(x)\land \#(x,\<true>,e|_1) > 0 \\[.5mm]
\hspace{.9cm}\land\ \forall\ \iota\in \mathcal{I}_{k}.\ \iota\in e|_2 \Leftrightarrow \big(\exists y.\ \mathsf{Rd\-Wr}(x,y)\land \#(y,\_,\iota) > 0\big)  \\[.5mm]
\hspace{.9cm}\land\ \exists\ \iota \in e|_2\land \forall\ \iota\in \mathcal{I}_{k}.\ \iota\in e|_2 \Rightarrow \neg e|_1 < \iota \big)
\end{array}
\]

The first two rows state that there exists a $\<read>(a)$ operation whose interval is exactly the first component in $e$
and one $\<write>(a)$ operation in each interval from the second component of $e$. The last row states that the second
component of $e$ is non-empty (i.e., there exists a write operation) the intervals
in the second component of $e$ (corresponding to write operations) are not ordered after the first component of $e$
(this is to say that the intervals of the write operations are before or overlapping with the interval of the read operations).

The fact that $\Psi_{\<reg>}[k]$ represents $L_{\<reg>}$ up to $k$ is 
a direct consequence of Lemma~\ref{lemma:register}. 
%The satisfaction problem for this formula is in polynomial time
%because %(1) the nesting of quantifiers is bounded and (2) 
%the quantifiers over values ($\forall a$ and $\exists a$) need to be instantiated only over
%values $a$ for which there exists at least one instantiation of $\<read>(a)$ or $\<write>(a)$ in the input 
%counting representation. Note that the range of the other quantifiers (e.g., $\forall\ \iota\in \mathcal{I}_{k}$ and 
%$\forall e\in \mathcal{E}_k$) is a domain of constant size.
\hfill $\Box$
\end{proof}

Theorem~\ref{th:register} implies that deciding if a history of length at most $k$ belongs to $L_{\<reg>}$
is in polynomial time. Note that the same problem is NP-complete for arbitrary histories~\cite{journals/siamcomp/GibbonsK97}.

%Deciding if a history of a concurrent register, of bounded length,
%is correct w.r.t. the atomic register, i.e., it is also a history of the atomic
%register, is decidable in polynomial time. The same problem for arbitrary
%histories is NP-complete since it is equivalent to deciding if an execution of
%the concurrent register is linearizable w.r.t. a sequential register (by
%Proposition~\ref{prop:atom}).
%
%\begin{theorem}
%
%  Let $h$ be a history of a concurrent register, of bounded length. Deciding if
%  $h\in At[Reg]$ is polynomial time.
%
%\end{theorem}
%
%\begin{proof}
%
%  Let $h$ be a history of a concurrent register of length $K$.
%
%\end{proof}