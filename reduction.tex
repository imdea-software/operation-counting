%!TEX root = draft.tex

\section{Operation Counting Refinement vs Observational Refinement}
\label{sec:equiv}

We prove that observational refinement is equivalent to the trace-based version of operation counting refinement w.r.t. a class of programs, 
where each thread executes at most one operation. Such programs are called \emph{\simple}; formally, they are parallel compositions of 
sequential commands, where every sequential command contains at most one command of the form $y \colonequals \callCom{\meth}{x}$.
Let $\sprogs$ denote the class of {\simple} programs. 

\begin{theorem}\label{prop:count}
Let $L$ and $S$ be two thread-independent libraries. % such that $L$ observationally refines $S$. 
Then, $\lib$ observationally refines $\spec$ iff operation counting refinement holds from $\lib$ to $\spec$ w.r.t. $\sprogs$.
\end{theorem}

The proof of this result is done in two steps. Lemma~\ref{lemma:equiv1} shows that if we consider thread-independent libraries, 
then observational refinement w.r.t. the class of all programs is equivalent to observational refinement w.r.t. the class of canonical 
programs. It is based on the fact that that behavior of any thread that executes two operations $\theta_1$ and $\theta_2$, in this order,
is captured by two threads that execute $\theta_1$ and \resp $\theta_2$, which synchronize themselves such that the 
thread executing $\theta_2$ begins after the thread executing $\theta_1$ has finished. 

Lemma~\ref{lemma:equiv2} shows that observational refinement w.r.t. $\sprogs$ is equivalent to operation counting refinement w.r.t. $\sprogs$.
The fact that observational refinement implies operation counting refinement  is a direct consequence of the equivalence between 
operation counting refinement and its state-based version. For the other direction, we prove that if operation counting refinement
holds from $\lib$ to $\spec$, then for every executable trace $\trace$ in the composition of a program $\prog$ with $\lib$, %, that reaches a state $\state$,
there exists an \emph{equivalent} executable trace $\trace'$ in the composition of $\prog$ with $\spec$. The trace $\trace'$ has the
same operation count as $\trace$ and its existence is implied by the definition of operation counting refinement.
This relation between executable traces in the composition of $\prog$ with the two libraries implies that any reachable state
in the composition of $\prog$ with $\lib$ is also reachable in the composition of $\prog$ with $\spec$.

\begin{lemma}\label{lemma:equiv1}
Given two thread-independent libraries $\lib$ and $\spec$, $\lib$ observationally refines $\spec$ iff $\lib$ observationally refines $\spec$ w.r.t. $\sprogs$.
\end{lemma}
%\smallskip
\noindent
{\bf Proof sketch:} Assume that $\lib$ observationally refines $\spec$ w.r.t. the class of {\simple} programs.

Let $\prog$ be an arbitrary program. We define a {\simple} program $\prog'$ such that any trace of $\prog$ is also a trace of $\prog'$ modulo a renaming of thread ids, and vice-versa. 
Therefore, $\prog'$ is obtained by rewriting each sequential command $C$ of $\prog$, which contains at least two method calls, into 
\[
\begin{array}{ll}
&(\assume\ g_C=0; C_1;g_C:=1) \\
\| &(\assume\ g_C=1;C_2;g_C:=2) \\
&\ldots \\
\|&(\assume\ g_C=n-1;C_n)
\end{array}
\]
where $C_1$,$\ldots$,$C_n$ are sequential commands s.t. $C=C_1;\ldots;C_n$ and each sequential command $C_i$ with $1\leq i\leq n$ contains at most one command of the form $y \colonequals \callCom{\meth}{x}$.The program $\prog'$ uses a different variable $g_C$ for each rewritten command $C$.

For any library $\libq$, there exists a bijection $\bijection$ between $\traces{\prog}{\libq}$ and $\traces{\prog'}{\libq}$ (modulo thread id renaming), such that for every trace $\trace\in \traces{\prog}{\libq}$, $\bijection(\trace)$ consists in the same sequence of actions as $\trace$, except for the fact that, for every rewritten command $C=C_1;\ldots;C_n$ of $\prog$ and every $1\leq i\leq n$, the thread id of the actions in $\trace$ corresponding to $C_i$ is substituted by a new value in $\bijection(\trace)$. %The same holds for the composition of $\prog$ and \resp $\prog'$ with $\spec$. 
Therefore, 
%The fact that the traces of $\prog'$ and the ones of $\prog'$ (that they are equal up to thread-id renaming)
%We can prove that 
for any state $\state_0$,
\[
\begin{array}{c}
\reach{\prog}{\lib}{\state_0}\subseteq \reach{\prog}{\spec}{\state_0} %\\[.5mm]
\mbox{ iff } %\\[.5mm]
\reach{\prog'}{\lib}{\state_0}\subseteq \reach{\prog'}{\spec}{\state_0}.
\end{array}
\]

Since $\lib$ observationally refines $\spec$ w.r.t. the class of {\simple} programs, the latter inclusion holds, which implies that also the first inclusion holds. Consequently, $\lib$ observationally refines $\spec$.\hfill$\Box$


%\paragraph{The inclusion between states can be reduced to the inclusion between some counter valuations:}
%\subsection{Main result}

%Given a trace $\tau$ in the composition of a program $\prog$ with a library $\lib$, we define two mappings
%\[
%{\tt open}[\trace]:\meths{L}\times\<Dom>\times\<Nats>->\<Nats> \mbox{, } {\tt done}[\trace]:\meths{L}\times\<Dom>\times\<Dom>\times\<Nats>\times\<Nats>->\<Nats>,
%\]
%which count the number of method calls performed in $\trace$ as follows:
%\begin{itemize}
%	\item $\opent{\trace}{f,i}{k}$ stands for the number of unfinished calls to $f$ with input $i$, initiated in between the $k$th and $(k+1)$th {\shwrite}. By convention, the $0$th {\shwrite} is the beginning of $\trace$. Also, if $K$ is the total number of {\shwrites} in $\trace$, then the $(K+1)$th {\shwrite} is the end of $\trace$.
%	\item $\donet{\trace}{f,i,o}{k_1,k_2}$ stands for the number of completed calls to $f$ with input $i$ and return value $o$, which begin in between the $k_1$th and $(k_1+1)$th {\shwrite} and finish in between the $k_2$th and $(k_2+1)$th {\shwrite}.
%\end{itemize}
%
%The set of all valuations for ${\tt open}$ and ${\tt done}$ corresponding a trace in the composition of a program $\prog$ with a library $\lib$ is denoted by $\counters{\prog}{\lib}$, \ie
%\[
%\counters{\prog}{\lib}=\set{({\tt open}[\trace],{\tt done}[\trace])\mid \trace\in\traces{\prog}{\lib}}.
%\]


\begin{lemma}\label{lemma:equiv2}
Let $L$ and $S$ be two thread-independent libraries. % such that $L$ observationally refines $S$. 
Then, $\lib$ observationally refines $\spec$ w.r.t. $\sprogs$ iff operation counting refinement holds from $\lib$ to $\spec$ w.r.t. $\sprogs$.
%there exists a method counting refinement from $\lib$ to $\spec$, \ie
%for any program $\prog$ and state $\state_0$,
\end{lemma}
{\bf Proof:} 
($\Rightarrow$) 
By definition, if $\lib$ observationally refines $\spec$ w.r.t. $\sprogs$, then state-based operation counting refinement holds from $\lib$ to $\spec$ w.r.t. $\sprogs$. 
Consequently, by Proposition~\ref{prop:trace_state},  trace-based operation counting refinement w.r.t. $\sprogs$ also holds.

%let $\reach{\prog}{\lib}{\state}[{\tt open},{\tt done}]$
%be the projections of the sates in $\reach{\progC}{\lib}{\state}$ on the values of ${\tt open}$ and ${\tt done}$.

%Since $\lib$ observationally refines $\spec$, we have that 
%\[
%\forall \programs\in\sprogs,
%%\ \comp{\progC}{\lib}\subseteq \comp{\progC}{\spec},
%\ \state_0\in\States.\ \reach{\progC}{\lib}{\state_0}\subseteq \reach{\progC}{\spec}{\state_0},
%\]
%which, by (\ref{eq:counters1}), %the characterization of $\counters{\prog}{\lib}$, 
%implies that
%\[
%\forall \prog\in\sprogs.\ 
%%\comp{\prog}{\lib}\state_0\subseteq \comp{\prog}{\spec}\state_0\mbox{ iff }
%\counters{\prog}{\lib}\subseteq \counters{\prog}{\spec}
%\]
%and
%\[
%\bigcup_{\prog\in\sprogs} \counters{\prog}{\lib}\subseteq \bigcup_{\prog\in\sprogs} \counters{\prog}{\spec}.
%\]
%

($\Leftarrow$)
Assume that operation counting refinement holds, \ie
\begin{equation}\label{eq:count1}
%\forall \prog\in\sprogs.\ 
%\comp{\prog}{\lib}\state_0\subseteq \comp{\prog}{\spec}\state_0\mbox{ iff }
\bigcup_{\prog\in\sprogs}\counters{\prog}{\lib}\subseteq \bigcup_{\prog\in\sprogs}\counters{\prog}{\spec}.
\end{equation}

Let $\prog\in\sprogs$, $\state_0$ a state, and $\state\in \reach{\prog}{\lib}{\state_0}$.
The latter implies that there exists a trace $\tau$ in $\tracesExec{\prog}{\lib}$ such that 
$\state\in \eval{\trace}{\state_0}$. W.l.o.g. we assume that for every command $y \colonequals \callCom{\meth}{x}$ of 
$\prog$, $x$ and $y$ are local variables.

In the following, we prove that there exists a trace $\trace'\in\tracesExec{\prog}{\spec}$ such that $\trace'\equiv\trace$. 
This implies that $\state\in \eval{\trace'}{\state_0}$ and thus, $\state\in \reach{\prog}{\spec}{\state_0}$.

By (\ref{eq:count1}), there is a trace $\tracep$ in $\tracesExec{\prog'}{\spec}$, for some $\prog'\in\sprogs$, s.t. 
$\Pi(\trace)=\Pi(\tracep)$.
%\begin{equation}\label{eq:count2}
%{\tt open}[\trace]={\tt open}[\tracep]\mbox{ and }{\tt done}[\trace]={\tt done}[\tracep].
%\end{equation}
%By (\ref{eq:count2}), 
If we ignore thread ids, $\trace$ and $\tracep$ contain exactly the same set of library actions
between every two consecutive shared writes. Because $\Pi(\trace)=\Pi(\tracep)$ there exists 
a bijection $\gamma$ between the thread ids of $\trace$ and the thread ids of $\tracep$ such that:
%We define a new history $\histpp$ by substituting in $\histp$ a thread id $\tid$ by $\tid'$ whenever:
%\vspace{-1eX}
\begin{itemize}
	\item $\tid$ and $\gamma(\tid)$ execute an $\oper{\meth}{\cval}{*}$-operation between the $k$th and $(k+1)$th {\shwrite} of $\trace$, \resp $\tracep$, 
	for some $\meth$, $\cval$, $k$, 
	or
%	let $t$ be a thread in $\trace$ that executes an $\oper{f}{\cval}{*}$-operation between the $k$th and $(k+1)$th {\shwrite}. Because we have that $\opent{\trace}{\oper{f}{\cval}{*}}{k}=\opent{\tracep}{\oper{f}{\cval}{*}}{k}$, there exists a thread $t'$ in $\tracep$ that executes a similar operation (\ie an $\oper{f}{\cval}{*}$-operation between the $k$th and $(k+1)$th {\shwrite}). %We replace in $\histp$ the thread $t'$ by $t$.
%We define $\gamma(t)=t'$.
	\item $\tid$ and $\gamma(\tid)$ execute an $\oper{f}{\cval}{\rval}$-operation, which begins between the $k_1$th and $(k_1+1)$th {\shwrite} of $\trace$, \resp $\tracep$, and ends between the $k_2$th and $(k_2+1)$th {\shwrite} of $\trace$, \resp $\tracep$, for some $\meth$, $\cval$, $\rval$, $k_1$, $k_2$.
%	let $t$ be a thread in $\trace$ that executes an $\oper{f}{\cval}{\rval}$-operation, which begins between the $k_1$th and $(k_1+1)$th {\shwrite} and ends between the $k_2$th and $(k_2+1)$th {\shwrite}. Because $\donet{\trace}{\oper{f}{\cval}{\rval}}{k_1,k_2}=\donet{\tracep}{\oper{f}{\cval}{\rval}}{k_1,k_2}$, there is a thread $t'$ in $\tracep$ executing a similar operation. 
%	%We replace in $\histp$ the thread $t'$ by $t$.
%We define $\gamma(t)=t'$.
%\vspace{-1eX}
\end{itemize}

%Let $\histp=\getHist{\tracep}$. 
Since $\spec$ is thread-independent, the history $\histpp$ obtained from $\getHist{\tracep}$ by replacing every thread id $t$ with $\gamma^{-1}(t)$ is also in $\spec$.
% in order to build another history $\histpp\in\sem{\spec}$

If $K$ is the total number of {\shwrites} in $\trace$, then $\trace=\trace_0\cdot\ldots\cdot\trace_K$, where, for every $0\leq k\leq K$, $\trace_k$ contains all the actions starting from the $k$th {\shwrite} and until the $(k+1)$th {\shwrite}. 
%Let $\trace_k$ be the sub-sequence of $\trace$ containing all the actions in between the $k$th and $(k+1)$th {\shwrite}. %(these two {\shwrite} are not included in $\trace_k$). 
By definition, $\histpp$ can be written as $\histpp_0\cdot \ldots\cdot \histpp_K$ s.t.
%\begin{itemize}
%\item 
for all $0\leq k\leq K$, $\histpp_k$ contains exactly the same set of library actions as $\trace_k$, but not necessarily in the same order.
%\end{itemize} 

We define $\trace'=\trace'_0\cdot\ldots\cdot\trace'_K$, where every $\trace'_k$ is a reordering of $\trace_k$ such that $\getHist{\trace'_k}=\histpp_k$.
% as a reordering of $\trace$ such that $\tracep$ is equivalent to $\trace$ and $\getHist{\tracep}=\histpp$. Therefore, for each $0\leq k\leq K$, the actions in $\trace_k$ are reordered as follows:
Therefore, $\trace'_k$ is defined by a recursive procedure ${\tt reorder}(\trace_k,\histpp_k)$ that works as follows.
%\begin{itemize}
Let $l$ be the first library action of $\histpp_k$ and let $\tid$ be the thread-id of $l$. Let $\trace_k[l]$ be the sub-sequence of $\trace_k$ containing all the actions of thread $\tid$ appearing before and including $l$ and let $\trace_k[{\neg l}]$ be the sub-sequence of $\trace_k$ containing all the other actions. We define 
\vspace{-1eX}
\[
\trace'_k=\trace_k[l]\cdot {\tt reorder}(\trace_k[{\neg l}],{\tt tail}(\histpp_k)),
\vspace{-1eX}
\]
where ${\tt tail}(\histpp_k)$ is the sequence $\histpp_k$ without its first element.
	%The actions in $\trace_k$ are reordered such that all the actions of $\trace_k^l$ appear at the beginning, but in the same order. 
	The trace $\trace'_k$ is equivalent to $\trace_k$ because $\trace_k[l]$ contains only the library action $l$, client actions over the local variables, and client actions that don't modify the value of the shared variables. Thus, any two actions, one from $\trace_k[l]$ and one from $\trace_k[\neg l]$, are independent.\hfill$\Box$
%	\begin{itemize}
%		\item $\trace_k[l]$ contains the library action $l$, client actions over the local variables, and client reads over the shared variables.
%		\item the actions of $\trace$ placed between the first and the last action of $\trace_k^l$ (which are not kept in $\trace_k^l$) are library actions, client actions over the local variables,  or client reads over the shared variables.
%	\end{itemize}
%Thus, any action of $\trace_k^l$ is independent of any action of $\trace$ placed between the first and the last action of $\trace_k^l$.
%	\item we consider the next library action of $\hist''_k$ and apply the same transformation starting from the current reordering of $\trace$. This continues until all the library actions of $\hist''_k$ are considered.
%\end{itemize}


%The latter implies that there exists a trace $\trace$ in $\sem{P}$ such that $\state\in \eval{\trace}{\state_0}$.
%
%Let $\state^C$ be the valuation for ${\tt nbW}$, ${\tt open}$, and ${\tt done}$ satisfied in $\trace$.
%Clearly, $\state^C\in {\tt Count}(\prog,\lib,\state_0)$, which, by the hypothesis, implies that 
%$\state^C\in {\tt Count}(\prog,\spec,\state_0)$. The latter implies that there exists a history $\hist$ of $S$ that contains all the 
%method calls in the trace $\trace$ (with the same input/return values) and that satisfies the following order constraints
%defined from $\trace$:
%\begin{itemize}
%\item a method call on $\meth$ with input $\cval$ and output $\rval$ is before a method call on $\meth'$ with input $\cval'$ and output $\rval'$
%if $\trace$ contains a write on a shared variable between $\retAct{\tid}{\lib}{\meth}{\rval}$ and $\callAct{\tid}{\lib}{\meth}{\cval}$.
%\end{itemize}
%
%One can prove that there exists a trace $\trace'$ that consists of all the internal actions of $\trace$ (in the same order if they
%belong to the same thread or if they concern the shared memory) and the history $\hist$, which is executable in the composition 
%of $P$ and $S$. Here, we use the fact the $S$ is thread-independent. Thus, $\state\in \comp{\progC}{\spec}\state_0$. 

\section{Operation Counting Refinement as a Reachable-state Inclusion Problem}
\label{sec:client}

%\paragraph{The universal quantification over programs can be reduced to some property of a most general client that counts method calls:}
We show that operation counting refinement can be equivalently stated as an inclusion between the reachable counter valuations in two instrumented programs. 
This simplifies the reasoning about the infinite union over all program clients in the original definition. % of operation counting refinement.
Therefore, %, which by Proposition~\ref{prop:trace_state} is equivalent to the trace-based version, 
we define a most general client for a library $\lib$ that represents all the reachable operation counts in a canonical client of $\lib$.
This is a program with an unbounded number of threads, where each thread non-deterministically calls a method of $L$. In order
to represent reachable operation counts this program is instrumented by $\Xi$. 
This program contains constructs that were not introduced in Section~\ref{sec:prelim}, but which 
are commonly found in concrete programming languages. Therefore, it contains the operator $+$ for the non-deterministic choice 
between a set of sequential commands, the operator $^*$ for the iteration of a sequential command, and the operator 
$^{\shuffle}$ for the parallel composition of an unbounded number of instances of a sequential command. Extending
the notion of trace and reachable state to programs that contain such constructs is straightforward. 

Given a library $\lib$ with $\meths{\lib}=\{\meth_1,\ldots,\meth_n\}$, the most general client for $\lib$, denoted by $\mgc{\lib}$,
%we will define a program $\mgc{\lib}$ whose reachable states in the composition with $\lib$ is exactly the union of 
%all $\reachCount{\instr{\prog}}{\lib}{\state_0}$ with $\prog\in \sprogs$ and $\state_0\in\States$. 
% Also, it uses the variable ${\tt shWr}$ and the mappings ${\tt open}$ and ${\tt done}$ in a similar way
%to the instrumentation $\Xi$ defined in Section~\ref{sec:ct_ref}.
is defined by 
%If $\meths{\lib}=\{\meth_1,\ldots,\meth_n\}$, then 
\[
%\mgc{\lib}\coloncolonequals Init; \big(\ 
(sh\colonequals \star)^*\ \|\   (C_{\meth_1}+\ldots+C_{\meth_n})\,^{\shuffle}
\]
where $sh$ is a shared variable\footnote{In order to respect the definition of a shared variable in Section~\ref{sec:prelim}, one should also add some dummy reads of $sh$ in the commands $C_{f_i}$.}
%and $C_{\lib}$ is the non-deterministic choice $C_{\meth_1}+\ldots+C_{\meth_n}$, where
and each $C_{f_i}$ calls $f_i$ with an arbitrary value, \ie $C_{f_i}$ is the sequential command 
$x\colonequals \star;y \colonequals \callCom{\meth_i}{x}$ (here, $\star$ stands for non-deterministic assignment).
Also, the variables $x$ and $y$ are considered to be local to each of the threads executing a command $\instr{C_L}$.
%a loop that non-deterministically calls methods of $L$, updating also the maps ${\tt open}$ and ${\tt done}$, \ie
%%\[
%%\begin{array}{l}
%$
%C_{\lib}\coloncolonequals \big(\ \instr{C_{\meth_1}}+\ldots+\instr{C_{\meth_n}})^*,
%$
%%\end{array}
%%\]
%and for any method $f_i$, $C_{f_i}$ is the sequential command $x\colonequals \star;y \colonequals \callCom{\meth_i}{x}$
%($\star$ stands for non-deterministic assignment).
%\[
%	\begin{array}{l}
%	x\colonequals \star;\\
%	{\tt atomic}\ \{ \\
%	\hspace{3mm}\open{\meth_i,x}{shWr}\mbox{\tt ++;}\\
%	\hspace{3mm}{\tt shWr0_c}\colonequals {\tt shWr;}\\
%	\}\\[.5mm]
%	y \colonequals \callCom{\meth_i}{x}{\tt ;} \\[.5mm]
%	{\tt atomic}\ \{\\ 
%	\hspace{3mm}\open{\meth_i,x}{shWr}\mbox{\tt --;} \\
%	\hspace{3mm} \done{\meth_i,x,y}{shWr0_c,shWr}\mbox{\tt ++;}\\
%	\}\\
%	\end{array}
%\]
%%and for any method $f$, $C_{f}$ is the sequential command defined in (\ref{eq:instr}).
%%and $C_{f_i}$ is the statement
%%	\[
%%	\begin{array}{ll}
%%	C_{f_i}\coloncolonequals & x\colonequals \star;\\
%%	&\open{\meth_i,x}{nbW}{\tt ++;}\\
%%	&{\tt nbW0}\colonequals {\tt nbW;}\\
%%	&y \colonequals \callCom{\meth_i}{x}{\tt ;} \\
%%	&\done{\meth_i,x,y}{nbW0,nbW}{\tt ++;}\\
%%	\end{array}
%%	\]
%
%%Let $\states{\mgc{\lib}}{\lib}|_{{\tt open},{\tt done}}$ denote the set of states in $\states{\mgc{\lib}}{\lib}$ projected on the values of {\tt open} and {\tt done}.


The set of reachable states in the composition of $\instr{\mgc{\lib}}$ with $\lib$, projected over the values of {\tt open} and {\tt done}, doesn't depend on the initial state. Thus, we omit the initial state from the function {\tt St} when applied to $\instr{\mgc{\lib}}$ and under the scope of $\Pi$.
%The set of reachable states in this program composed with $L$, projected on the values of ${\tt open}$ and ${\tt done}$, is denoted by $\reachMGC{\lib}$.
%\[
%\reachMGC{\lib}=\projReach{\mgc{\lib}}{\lib}{\state_0}{{\tt open},{\tt done}}
%\]

\begin{proposition}\label{prop:mgc}
Let $\lib$ be a thread-independent library. Then, 
\[
\reachMGC{\lib} = \bigcup_{\prog\in\sprogs,\state_0\in\InitStates}\reachCount{\instr{\prog}}{\lib}{\state_0}.
\]%\in\InitStates
\end{proposition}

The following result states that (state-based) operation counting refinement from a library $\lib$ to a library $\spec$ 
is equivalent to checking the inclusion between the reachable counter valuations in the instrumentation by $\Xi$ of the most general client 
of $\lib$ and \resp $\spec$. 
%This simplifies the problem of verifying operation counting refinement 
%because it avoids reasoning about the infinite set of library clients. 
It is a direct consequence of 
Proposition~\ref{prop:trace_state} and Proposition~\ref{prop:mgc}.

\begin{corollary}\label{th:mgc_count}
Let $L$ and $S$ be two thread-independent libraries. % such that $\meths{\lib}=\meths{\spec}={\mathbb M}$.
Then, operation counting refinement holds from $\lib$ to $\spec$ w.r.t. $\sprogs$ iff
\[
\reachMGC{\lib}\subseteq 
\reachMGC{\spec}.
\]
\end{corollary}
%{\bf Proof sketch:} 












%which clearly implies that ${\tt Count}(\prog,\lib,\state)\subseteq {\tt Count}(\prog,\spec,\state)$.

%Since the actions on the variables of $P'$ 