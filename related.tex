%!TEX root = draft.tex
\section{Related Work}
\label{sec:related}

%~\cite{DBLP:conf/kbse/ZhangCW13} the same approach as Line-Up but for quasi-linearizability

Previous work on automated verification for concurrent objects focuses on the
\emph{linearizability} criterion~\citep{journals/toplas/HerlihyW90}. While
folklore has long held that linearizability implies observational refinement,
Filipovic et al.~\cite{journals/tcs/FilipovicORY10} recently proved this fact;
they also proved that the two criterion coincide when considering only library
executions in which all operations have completed. In general, as we show in
Section~\ref{sec:lin}, observational refinement can hold for non-linearizable
objects.

The theoretical limits of linearizability checking are well studied. Gibbons
and Korach~\cite{journals/siamcomp/GibbonsK97} demonstrate NP-completeness for
checking a single execution. Alur et al.~\cite{journals/iandc/AlurMP00} show
EXPSPACE membership for checking finite-state implementations against atomic
specifications, but only when the number of program threads is bounded.
Bouajjani et al.~\cite{conf/esop/BouajjaniEEH13} show the same problem is
undecidable with unbounded threads, and introduce a decidable variant for a
restricted class of executions; one can show that this class corresponds to
executions whose histories have bounded interval length.

Several semi-automated approaches for proving linearizability, and thus
observational refinement, have relied on annotating method bodies with
\emph{linearization points}~\cite{conf/cav/AmitRRSY07, conf/fm/LiuCLS09,
conf/podc/OHearnRVYY10, conf/cav/Vafeiadis10, conf/icse/Zhang11a,
conf/pldi/LiangF13, conf/cav/DragoiGH13}, to reduce the otherwise-exponential
space of possible history linearizations to one single linearization. These
methods often rely on programmer annotation, and do not admit conclusive
evidence of a violation in the case of a failed proof.

Automated approaches for detecting linearizability violations such as
Line-Up~\cite{conf/pldi/BurckhardtDMT10} enumerate the exponentially-many
possible history linearizations. This exponential cost effectively limits such
approaches to executions with few operations, as noted in
Section~\ref{sec:exp:dynamic}. Colt~\cite{conf/oopsla/ShachamBASVY11}'s
approach mitigates this cost with programmer-annotated linearization points, as
in the previously-mentioned approaches, and ultimately suffers from the same
problem: a failed proof only indicates incorrect annotation.
