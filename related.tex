\subsection{Related Work}
\label{sec:related}

Existing work on automated verification of concurrent library implementations
have focused on the criterion of
\emph{linearizability}~\citep{journals/toplas/HerlihyW90}, which has recently
been shown to coincide with observational
refinement~\citep{journals/tcs/FilipovicORY10}. By definition, an execution
involving concurrent operations is \emph{linearizable} with respect to a
sequential specification $S$ if there exists some linear order $\sigma$ of
those operations (consistent with the partial order defined by the concurrent execution), 
which is executable by $S$. This criterion leads to automation
either by determining the \emph{linearization
point}~\citep{journals/toplas/HerlihyW90} of each operation, thus determining
the linear order of operations by that in which their linearization points were
executed, or by exhausting every possible linearization. While, on the one
hand, considering linearization points has led to several semi-automated
approaches for proving observational refinement~\citep{DBLP:conf/cav/AmitRRSY07,conf/fm/LiuCLS09, conf/podc/OHearnRVYY10,
conf/cav/Vafeiadis10, conf/icse/Zhang11a, conf/pldi/LiangF13,
conf/cav/DragoiGH13}, manual human effort is often necessary in annotating
methods with their linearization point. On the other hand, while additional
automation can be had for detecting violations of observational refinement 
by exhaustively considering all possible linearizations,
for instance via the testing-based tools LineUp~\citep{conf/pldi/BurckhardtDMT10} and COLT~\citep{DBLP:conf/oopsla/ShachamBASVY11}, exponential
explosion in the number of linearizations severely limits the number of
executed operations for which exploration is possible.

Others have considered the theoretical limits in verifying linearizability.
\citet{journals/siamcomp/GibbonsK97} showed that the linearizability of a
single execution is NP-complete, while \citet{journals/iandc/AlurMP00} showed
that determining linearizability of a library implementation with respect to a
given regular sequential specification is in EXPSPACE, so long as the number of
concurrent operations is bounded. \citet{conf/esop/BouajjaniEEH13} has shown that the same
problem becomes undecidable once the number of concurrent operations becomes
unbounded. The same work has introduced a decidable variant of linearizability, which
considers only executions where the number of temporal separations between non-overlapping 
operations is bounded.
%
% The proof of this undecidability
%ultimately led us to a decidable variant, by leveraging operation counting,
%where the number of shared-memory writes is bounded. Whereas the aim of this
%previous work was to establish theoretical boundaries, the goal of the current
%work is to establish operation counting as an effective means of automating
%observational refinement.
