%!TEX root = draft.tex
\section{Related Work}
\label{sec:related}

%~\cite{DBLP:conf/kbse/ZhangCW13} the same approach as Line-Up but for quasi-linearizability

Existing work on automated verification of concurrent library implementations
have focused on the criterion of
\emph{linearizability}~\citep{journals/toplas/HerlihyW90}. 
\citet{journals/tcs/FilipovicORY10}
shows that linearizability is equivalent to refinement when restricting the set of executions in a library
such that they contain only completed operations. They also show that, in general, linearizability implies
observational refinement. 
By definition, an execution
involving concurrent operations is \emph{linearizable} with respect to an
atomic reference implementation $L$ if there exists some linear order $\sigma$ of
those operations (consistent with the partial order defined by the concurrent execution), 
which is executable by $L$ (in our framework, $\sigma$ represents an execution in
the kernel of $L$). 
This criterion leads to automation
either by determining the \emph{linearization
point}~\citep{journals/toplas/HerlihyW90} of each operation, 
thus determining the linear order of operations by that in which their linearization points were
executed, or by exhausting every possible linearization. While, on the one
hand, considering linearization points has led to several semi-automated
approaches for proving observational refinement~\citep{DBLP:conf/cav/AmitRRSY07,conf/fm/LiuCLS09, conf/podc/OHearnRVYY10,
conf/cav/Vafeiadis10, conf/icse/Zhang11a, conf/pldi/LiangF13,
conf/cav/DragoiGH13}, manual human effort is often necessary in annotating
methods with their linearization point. On the other hand, while additional
automation can be had for detecting violations of observational refinement 
by exhaustively considering all possible linearizations,
for instance via the testing-based tools LineUp~\citep{conf/pldi/BurckhardtDMT10} and 
COLT~\citep{DBLP:conf/oopsla/ShachamBASVY11}, exponential
explosion in the number of linearizations severely limits the number of
executed operations for which exploration is possible (as we show in \S\ref{sec:exp:dynamic}). 
COLT deals with the exponential explosion by checking a criterion stronger than linearizability, 
where the linearization points are fixed to some particular statements in the implementation.

Others have considered the theoretical limits in verifying linearizability.
\citet{journals/siamcomp/GibbonsK97} showed that the linearizability of a
single execution is NP-complete (they consider as reference implementation an atomic register), 
while \citet{journals/iandc/AlurMP00} showed
that determining linearizability of a library implementation with respect to a
given regular sequential specification is in EXPSPACE, so long as the number of
concurrent operations is bounded. \citet{conf/esop/BouajjaniEEH13} has shown that the same
problem becomes undecidable once the number of concurrent operations becomes
unbounded. The same work has introduced a decidable variant of linearizability, which
considers only executions where the number of temporal separations between non-overlapping 
operations is bounded.
%
% The proof of this undecidability
%ultimately led us to a decidable variant, by leveraging operation counting,
%where the number of shared-memory writes is bounded. Whereas the aim of this
%previous work was to establish theoretical boundaries, the goal of the current
%work is to establish operation counting as an effective means of automating
%observational refinement.