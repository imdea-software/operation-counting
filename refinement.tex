%!TEX root = draft.tex
\section{Observational Refinement}
\label{sec:refinement}

% Standard correctness criteria for software library implementations are their
% conformance to \emph{reference implementations}. The literature captures such
% conformance with the notion of \emph{observational refinement}: is every
% behavior of a program using a given library also possible when using instead
% its reference implementation?

We formalize the criterion of \emph{observational refinement} using a simple
yet universal model of computation, namely labeled transition systems (LTS).
This model captures shared-memory programs with an arbitrary number of threads,
abstracting away the details of any particular programming system irrelevant to
our development.

A \emph{labeled transition system} $A = (Q,@S,q_0,@d)$ over the
possibly-infinite alphabet $@S$ is a possibly-infinite set $Q$ of states with
initial state $q_0 \in Q$, and a transition relation $@d \subseteq Q \x @S \x
Q$. The $i$th symbol of a sequence $e \in @S^*$ is denoted $e_i$. An
\emph{execution} of $A$ is a sequence $e \in @S^*$ such that for some $q_1,
q_2, .., q_{|e|} \in Q$, we have $@d(q_i,e_i,q_{i+1})$ for each $i$ such that
$0 \le i < |e|$. The projection $e|@G$ is the maximum subsequence of $e$ over
alphabet $@G$. $E(A)$ denotes the set of $A$'s executions, and $E(A)|@G$ their
projections over $@G$ (note that, $E(A)$ is prefix closed). 
The \emph{synchronous product} $A_1 \x A_2$ of two LTSs
is defined as usual, respecting $E(A_1 \x A_2)|(@S_1 \n @S_2) = E(A_1)|@S_2 \n
E(A_2)|@S_1$.

%\begin{example}
%  \label{ex:threads:1}
%
%  A multithreaded system $\tup{S,L,@S,s_0,@l_0,n,->}$ of $n$ threads with
%  shared states $S$, local states $L$, actions $@S$, and per-thread transition
%  relation $-> \subseteq (S \x L) \x @S \x (S \x L)$ is modeled by the LTS
%  $\tup{Q,@S,q_0,@d}$ with states $Q = S \x L^n$ and initial state $q_0 =
%  \tup{s_0, @l_0, .., @l_0}$, which includes a transition
%  \begin{align*}
%    @d(\tup{s, @l_1, .., @l_n}, a, \tup{s', @l_1', .., @l_n'})
%  \end{align*}
%  iff $\tup{s,@l_i} \overset{a}{\rightarrow} \tup{s',@l_i'}$ for some $i$, and
%  $@l_j = @l_j'$ for $j \neq i$.
%  
%  \todo{actions should be associated to a thread}
%
%\end{example}

\subsection{Libraries}

Programs interact with libraries by calling named library \emph{methods}, which
receive \emph{parameter values} and yield \emph{return values} upon completion.
We fix arbitrary sets $\<Methods>$ and $\<Vals>$ of method names and
parameter/return values. 

\begin{example}
  \label{ex:methods}

  The method and value sets for the stack implementation in
  Figure~\ref{fig:treiber} are $\<Methods> = \set{ \<push>, \<pop> }$ and
  $\<Vals> = \<Nats> \u \set{ {\tt EMPTY} }$.

\end{example}

\noindent
We fix an arbitrary set $\<Ops>$ of operation identifiers, and for given sets
$\<Methods>$ and $\<Vals>$ of methods and values, we fix the sets
\begin{align*}
  & C = \set{ m(v)_o : m \in \<Methods>, v \in \<Vals>, o \in \<Ops> }
  \text{, and } \\
  & R = \set{ \<ret>(v)_o : v \in \<Vals>, o \in \<Ops> }  
\end{align*}
of \emph{call actions} and \emph{return actions}; each call action $m(v)_o$
combines a method $m \in \<Methods>$ and value $v \in \<Vals>$ with an
\emph{operation identifier} $o \in \<Ops>$. Operation identifiers are used to
pair call and return actions. We denote the operation identifier of a
call/return action $c$ by $\<op>(c)$. Call and return actions $c \in C$ and $r
\in R$ are \emph{matching}, written $c \match r$, when $\<op>(c) = \<op>(r)$. A
word $e \in @S^*$ over alphabet $@S$, such that $(C \u R) \subseteq @S$, is
\emph{well formed} when:
\begin{itemize}

  \item Each return is preceded by a matching call: \\
  $e_j \in R$ implies $e_i \match e_j$ for some $i < j$.

  \item Each operation identifier is used in at most one call/return: \\
  $\<op>(e_i) = \<op>(e_j)$ and $i < j$ implies $e_i \match e_j$.

\end{itemize}
We say that the well-formed word $e \in @S^*$ is \emph{sequential} when
\begin{itemize}

  \item Operations do not overlap: \\
  $e_i, e_k \in C$ and $i < k$ implies $e_i \match e_j$ for some $i < j < k$.

\end{itemize}
Well-formed words represent executions. We assume every set of well-formed
words is closed under isomorphic renaming of operation identifiers. For
notational convenience, we often associate $\<Ops>$ with $\<Nats>$,
e.g.,~writing $m(u)_1$ and $\<ret>(v)_2$ in place of $m(u)_{o_1}$ and
$\<ret>(v)_{o_2}$. An operation $o$ of an execution $e$ is \emph{completed}
when both call and return actions $m(u)_o$ and $\<ret>(v)_o$ of $o$ occur in
$e$, and is otherwise \emph{pending}.

\begin{example}
  \label{ex:executions}

  The well-formed words
  \scriptsize
  \begin{align*}
     \<push>(0)_1\ \<pop>_2\ \<pop>_3\ \<ret>_1\ \<ret>(0)_3\ \<ret>(0)_2 \\
    \text{\normalsize and } 
    \<push>(0)_1\ \<pop>_2\ \<pop>_3\ \<ret>_1\ \<ret>(0)_2
  \end{align*}
  \normalsize
  represent executions in which one call to the $\<push>(0)$ method overlaps
  with two calls to $\<pop>$. In the first execution both calls to $\<pop>$
  have matching return actions $\<ret>(0)$, i.e., the operations $2$ and $3$ are completed,
  while operation $3$ is pending in the second, it has no matching return.

\end{example}

Libraries dictate the execution of methods between their call and return
points. Accordingly, a library cannot prevent a method from being called,
though it can decide not to return. Furthermore, any library action performed
in the interval between call and return points can also be performed should the
call have been made earlier, and/or the return made later. Our technical
results rely on these properties. A library thus allows any sequence of
invocations to its methods made by \emph{some} program.

\begin{definition}
  \label{def:library}

  A \emph{library} $L$ is an LTS over alphabet $C \u R$ such that each
  execution $e \in E(L)$ is well formed, and
  \begin{itemize}

    \item Call actions $c \in C$ cannot be disabled: \\
    $e \cdot e' \in E(L)$ implies $e \cdot c \cdot e' \in E(L)$
    if $e \cdot c \cdot e'$ is well formed.
  
    \item Call actions $c \in C$ cannot disable other actions: \\
    $e \cdot a \cdot c \cdot e' \in E(L)$ implies $e \cdot c \cdot a \cdot e' \in E(L)$.
  
    \item Return actions $r \in R$ cannot enable other actions: \\
    $e \cdot r \cdot a \cdot e' \in E(L)$ implies $e \cdot a \cdot r \cdot e' \in E(L)$.
  
  \end{itemize}

\end{definition}

\noindent
We write $e_1 ~> e_2$ when $e_2$ can be derived form $e_1$ by applying zero or
more of the above rules. The \emph{closure} of a set $E$ of executions under
$~>$ is denoted $\overline{E}$.

Note that even a library that implements \emph{atomic methods}, e.g.,~by
guarding method bodies with a global-lock acquisition, admits executions in
which method calls and returns overlap. A library which accesses the client's
thread identifiers can be modeled by taking thread identifiers as method
parameters.

\begin{example}
  \label{ex:libraries}

  Any library which admits the execution
  \scriptsize
  \begin{align*}
    \<push>(0)_1\ \<ret>_1\ \<pop>_2\ \<ret>(0)_2
  \end{align*}
  \normalsize
  with sequential calls to $\<push>$ and $\<pop>$ must also admit
  \scriptsize
  \begin{align*}
    \<push>(0)_1\ \<pop>_2\ \<ret>_1\ \<ret>(0)_2
    \text{ \normalsize and }
    \<push>(0)_1\ \<pop>_2\ \<pop>_3\ \<ret>_1\ \<ret>(0)_2
    \text{\normalsize,}
  \end{align*}
  \normalsize
  among others, yet need not admit an execution
  \scriptsize
  \begin{align*}
    \<push>(0)_1\ \<pop>_2\ \<pop>_3\ \<ret>_1\ \<ret>(0)_3\ \<ret>(0)_2
  \end{align*}
  \normalsize
  with two completed $\<pop>$ operations returning $0$.
  
\end{example}

\begin{lemma}
  \label{lem:kernel}

  For any set $E(L)$ of library executions, there exists a unique minimal
  (w.r.t.~set inclusion) set $E_0$ such that $\overline{E}_0 = E(L)$.

\end{lemma}

\begin{proof}\let\qed\relax

  By contradiction, suppose that $E_1 \neq E_2$ are minimal such that
  $\overline{E}_1 = \overline{E}_2 = E(L)$. Let $E_1' = E_1 \setminus E_2$ and
  $E_2' = E_2 \setminus E_1$ be the elements unique to $E_1$ and $E_2$. Each
  element of $e_2' \in E_2'$ must be $~>$-derivable from some $e_1' \in E_1'$,
  and visa versa, since $e_2' \in \overline{E}_1$ and $e_2'$ is not derivable
  from $(E_1 \setminus E_1') = (E_2 \setminus E_2')$ --- otherwise $E_2$ would
  not be minimal. It follows that $|E_1'| = |E_2'|$, again by minimality, and
  that there must be a non-trivial cycle $e_1 ~> e_2 ~> .. ~> e_{2i} ~> e_1$ of
  elements alternating between $E_1'$ and $E_2'$; however it is easily seen
  that non-trivial $~>$-cycles do not exist.

\end{proof}

\noindent
This minimal $E_0$ of Lemma~\ref{lem:kernel} is called the \emph{kernel} of
$E(L)$, and is denoted by $\ker E(L)$. A library is called \emph{atomic} if its
kernel contains only sequential executions. Atomic libraries are typically used
as specifications for concurrent objects.

\begin{example}
  \label{ex:atomic_stack}

  A library $L$ is called an \emph{atomic stack} iff the kernel $\ker E(L)$
  consists of all sequential executions with only completed operations such
  that the return value of any $\<pop>$ invocation $o$ is:
  \begin{itemize}

  	\item {\tt EMPTY}, if the number of $\<push>$ invocations before $o$ equals
  	the number of $\<pop>$ invocations before $o$,

  	\item $v$, where $v$ is the input of the last $\<push>$ invocation,
  	otherwise.

  \end{itemize}
  The lock-based implementation mentioned in \S\ref{sec:motivation} is an atomic
  stack.

\end{example}

TODO CLARIFY AND LOCALIZE THIS COMMENT TO WHERE IT IS RELEVANT
By minimality, the kernel of an atomic library cannot contain sequential
executions with pending operations. Thus, if $\ker E(L)$ were to contain a
sequential execution $e = e' \cdot m(u)_o$, then it should also contain $e'$
since $E(L)$ is prefix-closed and $e \not\leadsto e'$. But, since $e' ~> e$, by
minimality, $\ker E(L)$ should contain only $e'$.

% a sequential execution $e$ with one pending operation at the end,

% The kernel contains only completed executions ... for a sequential execution
% $e$ with a pending operation at the end, the execution without this pending
% operation is also in the library because of prefix closure and it is a
% generator for $e$ (by applying the first closure rule)

\subsection{Refinement between Libraries}

Refinement between libraries is defined with respect to the observable actions
of programs which invoke library methods. Complementary to libraries, programs
control their execution outside of method call and return points. Accordingly,
any program action performed in the interval between call and return points can
also be performed should the call have been made later, and/or the return made
earlier. A program thus allows any sequence of matching returns generated by
\emph{some} implementation of the methods it invokes.

\begin{definition}
  \label{def:programs}

  A \emph{program} $P$ over actions $@S$ is an LTS over alphabet $(@S \uplus C
  \uplus R)$ where each execution $e \in E(P)$ is well formed, and
  \begin{itemize}

  	\item Call actions $c \in C$ cannot enable other actions: \\
    $e \cdot c \cdot a \cdot e' \in E(P)$ implies
    $c \match a$ or $e \cdot a \cdot c \cdot e' \in E(P)$.

    \item Return actions $r \in R$ cannot disable other actions: \\
    $e \cdot a \cdot r \cdot e' \in E(P)$ implies
    $a \match r$ or $e \cdot r \cdot a \cdot e \in E(P)$.

    \item Return actions $r \in R$ cannot be disabled: \\
    $e \cdot e' \in E(P)$ implies $e \cdot r \cdot e' \in E(L)$
    if $e \cdot r \cdot e'$ is well formed.

  \end{itemize}
\end{definition}

\begin{example}
  \label{ex:programs}

  Any program which admits the execution
  \scriptsize
    \begin{align*}
    \<push>(0)_1\ \<pop>_2\ \<ret>(0)_2\ \<pop>_3\ \ x = 0\ \<ret>_1
  \end{align*}
  \normalsize
  with overlapping calls to $\<push>$ and $\<pop>$ must also admit
  \scriptsize
  \begin{align*}
    &\<push>(0)_1\ \<ret>_1\ \<pop>_2\ \<ret>(0)_2\ \<pop>_3\ \ x = 0\text{\normalsize,}  \\
%    \end{align*}
%%    \scriptsize
%    \begin{align*}
    & \<push>(0)_1\ \<ret>_1\ \<pop>_2\ \<ret>(0)_2\ \<pop>_3\ \<ret>({\tt EMPTY})_3\ \ x = 0 % \text{ \normalsize and }
    \text{\normalsize,} 
  \end{align*}
  \normalsize
  among others, yet need not admit an execution
  \scriptsize
  \begin{align*}
    \<push>(0)_1\ \<ret>_1\ \<pop>_3\ \<pop>_2\ \<ret>(0)_2\  \ x = 0\text{\normalsize,}
  \end{align*}
  \normalsize
  where the second call to $\<pop>$ starts before the first one. In the second execution that must be admitted by the program, 
  $\<pop>_3$ could have been completed
  with any return action, even with $\<ret>(1)_3$ instead of $\<ret>({\tt EMPTY})_3$. The set of executions 
  admitted by a program considers any possible implementation of the methods.  
%  The sequence of actions program
%  is defined as the set of actions possible when composed with any library, not only correct implementations
%  of a concurrent stack.
  
\end{example}

%\begin{example}
%  \label{ex:threads:2}
%
%  A multithreaded program $P$ with $n$ threads, e.g.,~as defined in
%  Example~\ref{ex:threads:1}, only admits executions with at most $n$
%  overlapping method calls. A library which accesses $P$'s thread identifiers
%  can be modeled by defining the set $\<Methods>$ of methods to include thread
%  identifiers as method parameters.
%
%\end{example}

Refinement between libraries $L_1$ and $L_2$ means that any program execution
possible with $L_1$ is also possible with $L_2$.
\begin{definition}

  The library $L_1$ \emph{refines} $L_2$, written $L_1 \leq L_2$, iff
  \begin{align*}
    E(P \x L_1)|@S \ \subseteq \ E(P \x L_2)|@S
  \end{align*}
  for all programs $P$ over actions $@S$.

\end{definition}
As library and program alphabets only intersect on call and return actions
$C \cup R$, our formalization does not directly capture other means of
communication, e.g.,~through shared random-access memory.

\begin{example}

  As we have shown previously, the Treiber's stack implementation from Figure~\ref{fig:treiber}
  doesn't refine an atomic lock-based reference implementation. However, by removing the {\tt free} statement
  from the $\<pop>$ method and by considering a semantics for {\tt malloc} that doesn't return
  previously used memory addresses, we obtain a library, which refines
  the lock-based concurrent stack.
  
%  An $L_1$ that admits both executions from Example~\ref{ex:executions} does
%  not refine an $L_2$ which only admits the first, as witnessed by a program
%  which calls $\<push>(0)$ and $\<pop>$ twice, in parallel.

\end{example}
