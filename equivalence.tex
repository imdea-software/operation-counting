%!TEX root = draft.tex
\section{Observational refinement is equivalent to history inclusion}

To avoid an overload of notation, for any two sets of histories $H$ and $H'$, $H\subseteq H'$ 
denotes the fact that for every history $h\in H$ there exists a history $h'\in H'$ s.t. 
$h$ and $h'$ are equivalent.

\begin{theorem}

  $L_1 \leq L_2$ iff $H(L_1) \subseteq H(L_2)$.

\end{theorem}

\begin{proof}

  ($\Rightarrow$) Let $h\in H(L_1)$. As in Filipovic et al., construct a
  program $P_h$ (this LTS can be described as a program with shared variables)
  where we use shared variables to enforce the order constraints in $h$.
  Because $ReachStates_1(P_h\times L_1) \subseteq ReachStates_1(P_h\times
  L_2)$, there exists an execution $e$ of $P_h\times L_2$ s.t. $H(w)=h'$ and
  $h\preceq h'$. By the closure property on $L_2$, if $h'\in H(L_2)$ then $h\in
  H(L_2)$.

  ($\Leftarrow$) Let $P$ be a program and let $e$ be an execution in $P\times
  L_1$. By hypothesis, $h=H(e)\in H(L_2)$. Let $w'\in L_2$ s.t. $H(w')=h$. Note
  that the projection of $e$ on library actions may be different than $w'$. We
  must show that by the closure properties on $P$, $E(P)$ contains the
  interleaving of $w'$ and the projection of $e$ on client actions.

\end{proof}
