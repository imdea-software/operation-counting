%!TEX root = draft.tex
\section{Observational refinement is equivalent to history inclusion}\label{sec:equiv}

We show that the observational refinement between two libraries can be simply expressed as the inclusion between 
their sets of histories.This result reduces the burden of reasoning about the universal quantifier over programs and it is 
similar to the relation between observational refinement and linearizability~\cite{journals/tcs/FilipovicORY10}: linearizability implies
observational refinement and observational refinement implies linearizability but only for 
executions where every operation is completed 
(see the discussion below). To explain
the differences we first recall the notion of linearizability. Thus, a library execution $e_1$ is linearizable w.r.t. a library $L_2$
iff there exists an execution $e_1'$ obtained from $e_1$ by appending return actions (for unmatched call actions) or deleting call actions
and an execution $e_2$ of $L_2$ such that $e_1'$ doesn't contain pending operations, $e_1'$ and $e_2$ contain exactly the same set of actions and $e_2$ is a 
permutation of $e_1'$ that preserves the order between 
return and call actions (i.e., if a return action $r\in R$ occurs before a call action $c\in C$ in $e_1'$ then the same holds in $e_2$).
Then, a library $L_1$ is linearizable w.r.t. a library $L_2$ iff every execution $e_1\in L_1$ is linearizable w.r.t. $L_2$.
%

If we consider that libraries are defined as arbitrary sets of well-formed sequences of call and return actions with possibly 
pending operations, then observational refinement does not imply linearizability. 
For example, this holds for two libraries $L_1$ and $L_2$ containing a method $m$ that never finishes. 
Intuitively, a program can observe that it called the method $m$ with both libraries but the library execution of $L_1$
that allowed this observation is not linearizable. More precisely, let $L_1$ and $L_2$ be two libraries that contain 
a single execution $\tup{m,o}$ and $P$ a program with a single execution 
$\print{\tup{m,o}} \cdot \tup{m,o}$, where $\print{\tup{m,o}}$ is a program action. The observation of $P$ is $\print{\tup{m,o}}$
and it is possible when composing $P$ with both $L_1$ and $L_2$ but the library execution $\tup{m,o}$ is not linearizable 
w.r.t. $L_2$ because $L_2$ doesn't contain an execution where $m$ finishes.

When considering two real libraries $L_1$ and $L_2$, which must satisfy the closure 
properties in Section~\ref{sec:obsref}, we show that observational refinement between $L_1$ and $L_2$ is actually 
\emph{equivalent} to a plain inclusion between histories. Compared to the
result in \citet{journals/tcs/FilipovicORY10}, we give a property that is equivalent to observation refinement irrespectively of the
presence of pending operations, we consider a comparison between histories, which are more abstract 
than library executions (see Example~\ref{}), and we consider a universal model of computation for programs 
(the result in \citet{journals/tcs/FilipovicORY10} is proved only for programs with shared variables). 

Our result can be explained by the following relation between linearizability and the ``weaker than'' comparison between histories.

\begin{proposition}\label{prop:lin}
An execution $e_1$ is linearizable w.r.t. a library $L_2$ iff there exists an execution $e_2\in L_2$, where every operation is completed, 
such that $H(e_1)\preceq H(e_2)$.
\end{proposition}


%Then, linearizability between an execution $e_1$ and a library $L_2$ is equivalent to the fact that the history of $e_1$, where
%some pending operations may be deleted, is weaker than a history of $L_2$ with no pending operations. Linearizability
%between two libraries $L_1$ and $L_2$ means that every execution of $L_1$ is linearizable w.r.t. $L_2$.

Thus, if an execution $e_1$ is linearizable w.r.t. a library $L_2$, hence the comparison between their histories 
stated in Proposition~\ref{prop:lin} holds, then by Lemma~\ref{lemma:lib_closure}, 
there exists a history of $L_2$ which is actually equivalent to the history of $e_1$.

To avoid an overload of notation, for any two sets of histories $H$ and $H'$, $H\subseteq H'$ 
denotes the fact that for every history $h\in H$ there exists a history $h'\in H'$ s.t. 
$h$ and $h'$ are equivalent.

\begin{theorem}\label{th:equiv}

  $L_1 \leq L_2$ iff $H(L_1) \subseteq H(L_2)$.

\end{theorem}

\begin{proof}

  ($\Rightarrow$) Let $h\in H(L_1)$ be a history of the library $L_1$ and let  $e$ be an execution of the library $L_1$ such that $H(e)=h$.
  
  The annotation of $e$, denoted $\overline{e}$, is an execution obtained from $e$ by adding the following program actions:
  \begin{itemize}
  	\item $\print{\tup{m,o}}$ immediately before every call action $\tup{m,o}$,
	\item $\print{\tup{o,m}}$ immediately after every return action $\tup{o,m}$,
	\item $\sync{o}{o'}$ between every two actions $r=\tup{o,m}\in R$ and $c=\tup{m,o}\in C$ such that $r$ occurs before $c$ in $e$.
  \end{itemize}
  
  Let $P$ be a program such that $E(P)$ is the minimal set of executions that contains $\overline{e}$. %and satisfies all the closure properties required for a program. 
  The set of program actions in $P$, $\Sigma_P$, consists of symbols $\print{\tup{m,o}}$, $\print{\tup{o,m}}$, and $\sync{o}{o'}$ with $o,o'\in \<Ops>$ and $m\in\<Methods>$.
  
 Note that for any execution $e'$ obtained from $\overline{e}$ by applying the closure properties, $\sync{o}{o'}$ still occurs in between the return action $\tup{o,m}$ and the call action $\tup{m,o}$. This is because actions occurring after a return action $r$ cannot be moved to the left of $r$ and actions occurring before a call action $c$ cannot be moved to the right of $c$. Therefore, for any execution $e'\in E(P)$, that contains all the program actions in $\overline{e}$, $h\preceq H(e')$. Moving call actions to the right and return actions to the left may imply that the history $H(e')$ contains  more order constraints on operations than $h$ and adding return actions of the end may imply that the history $H(e')$ contains more completed operations than $h$. Also, prefix-closure implies that $h$ may have more pending operations.
  
%  Let 
%  \[
%  \overline{E}(L_1,h)=\{\overline{e}\mid e\in E(L_1,h)\}
%  \]
%  be the set of all the annotated executions in $E(L_1,h)$.
      
  Because $L_1 \leq L_2$ any sequence that contains all the program actions in $\overline{e}$ belongs to $E(P \x L_2)|@S_P$. Therefore, there exists 
  an execution $e'\in E(P \x L_2)$ such that $h\preceq H(e')$ with $H(e')\in H(L_2)$. 
  By Lemma~\ref{lemma:lib_closure}, the set of histories of $L_2$ is downward closed and consequently, $h\in H(L_2)$.
  
  The same proof holds also if we consider multi-threaded programs as in Example~\ref{ex:threads:1}. In this case, the $\<print>$ actions are considered to be local to the thread that executes the corresponding call or return action while $\sync{o}{o'}$ is replaced by two actions: $(t,x[o,o']=\<true>)$ immediately after the return action $r=\tup{o,m}$ and $(t',\<assume> (x[o,o'] = = \<true>))$ immediately before the call action $c=\tup{m,o}$, where $t$ is the thread executing $r$ and $t'$ is the thread executing $c$.
  
  ($\Leftarrow$) Let $P$ be a program and let $e$ be an execution in the composition of $P$ with $L_1$, $P\times L_1$. 
  We show that $e$ is also an execution of $P\times L_2$, which implies that the projection of $e$ on program actions $e|@S_P$ belongs to $E(P \x L_2)|@S_P$.

  By hypothesis, the history of $e$ is also a history of the library $L_2$, i.e., $H(e)\in H(L_2)$. 
  Let $l$ be the projection of $e$ on library actions. % and $l'$ an execution of $L_2$ whose history is $h$.   
  By Lemma~\ref{lemma:lib_exec}, $l$ is also an execution of $L_2$. Therefore, $e$ is also an execution of $P\times L_2$.
  \hfill $\Box$
%  Note
%  that the projection of $e$ on library actions may be different than $w'$. We
%  must show that by the closure properties on $P$, $E(P)$ contains the
%  interleaving of $w'$ and the projection of $e$ on client actions.

\end{proof}

\paragraph{Linearizability w.r.t. sequential specifications.} Concurrent implementations of data structures are usually considered to be correct if they are linearizable w.r.t. the sequential version of the same data structure, e.g., a concurrent stack should be linearizable w.r.t. the sequential stack. In our framework, we consider only concurrent libraries, therefore, this correctness condition translates to the fact that the concurrent implementation is an observational refinement of the atomic implementation, where, intuitively, method bodies execute in one single atomic step but, method calls and returns can still overlap. 

Given a \emph{sequential} library $L_{seq}$, i.e., whose executions are all sequential, the set of executions allowed by the \emph{atomic} concurrent library $L_{at}$ where method bodies are exactly the same as in $L_{seq}$ (modulo the fact that they are surrounded by synchronization actions, e.g., locks, in order to ensure that they are executed atomically) can be defined as the least set of executions that contains all the executions of $L_{seq}$ and that satisfies the closure properties in Section~\ref{sec:obsref}. Then, the following result is a direct consequence of the definitions.

\begin{proposition}\label{prop:atom}
An execution $e_1$ is linearizable w.r.t. a sequential library $L_{seq}$ iff $H(e_1)\in H(L_{at})$, where $L_{at}$ is the atomic concurrent library corresponding to $L_{seq}$.
\end{proposition}

The extension of Proposition~\ref{prop:atom} to libraries is straightforward, i.e., a library $L$ is linearizable w.r.t. a sequential library $L_{seq}$ iff $H(L)\subseteq H(L_{at})$, where $L_{at}$ is as above (or equivalently, by Theorem~\ref{th:equiv}, $L$ is an observational refinement of $L_{at}$).


