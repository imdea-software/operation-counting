%!TEX root = draft.tex
\section{Atomic Collections}
\label{sec:containers}

Collection-based data structures such as stacks and queues are among the most
heavily investigated concurrent objects~\cite{chapter/cds/MoirS07}. In this
section we demonstrate that our approximation is effective in uncovering
refinement violations for such atomic objects, in the sense that most
violations can be uncovered with coarse approximations, i.e.,~$k\le 3$,
depending on the data structure. To obtain these results we exhibit families
$@Y$ of operation counting formul\ae characterizing these structures with
``cutoff'' (or ``small-model'') properties: for any history $h$ violating $@Y$
there exists $h' \preceq h$ violating $@Y$ with $\len h' \le k$. While this
$h'$ may not correspond directly to $A_k(h)$, we can deduce there exists some
$h'$ whose $k$-approximation $A_k(h')$ does violate $@Y$. Our results build off
of previous characterizations of concurrent data
structures~\cite{conf/tacas/AbdullaHHJR13, conf/concur/HenzingerSV13} into a
small set of constituent properties. For ease of presentation, in the following
we consider only completed histories, i.e.,~in which no operation is pending.
This restriction comes without loss of generality for library implementations
which ensure that each operation can complete when uninterrupted by
others~\cite{conf/concur/HenzingerSV13}. In any case, our formul\ae can be
extended to handle pending operations as well.


\subsection{Small-Model Formulas for Stacks and Queues}
\label{sec:containers:small}

A \emph{formula family} $@Y$ is an indexed set $\set{
@Y_i : i \in \<Nats> }$. We say a history $h$ \emph{satisfies} $@Y$,
written $h |= @Y$, when $h |= @Y_i$ for $i = \len h$. We say that a family $@Y$
has a \emph{cutoff} $n \in \<Nats>$ when for each history $h |= @Y$ there
exists some $h' \preceq h$ such that $A_n(h')|= @Y$. Figure~\ref{fig:spec:ds}
defines four families of operation-counting formul\ae:
\begin{itemize}

  \item $@Y_\mathrm{rv}$ characterizes \emph{remove violations} in which a
  {\sf pop} operation returns a value for which there is no corresponding {\sf
  push}.

  \item $@Y_\mathrm{ev}$ characterizes \emph{empty violations} in which some
  {\sf pop} operation returns {\tt EMPTY}, yet whose span is covered by the
  presence of some {\sf push}ed element which has not (yet) been {\sf pop}ped.

  \item $@Y_\mathrm{fv}$ characterizes \emph{FIFO violations} in which some
  pair of {\sf pop}s occur in the \emph{opposite} order of their corresponding
  {\sf push}es.

  \item $@Y_\mathrm{lv}$ characterizes \emph{LIFO violations} in which some
  pair of {\sf pop}s occur in the \emph{same} order as their corresponding {\sf
  push}es, and the second push occurs before the first pop.

\end{itemize}
As in previous work~\cite{conf/tacas/AbdullaHHJR13, conf/concur/HenzingerSV13},
our arguments for the completeness of these properties relies on \emph{data
independence}~\cite{conf/popl/Wolper86}, i.e.,~that library executions are
closed under consistent renaming $\<Vals> -> \<Vals>$ of method call and
return values, and assume that each value is {\sf push}ed at most once. In
practice, collection-based data structures are data independent, and the second
condition can always be met by tagging each value with a unique identifier.
However, in order to achieve bounded counting representations, we may only
distinguish between these values up to equivalence relations with finite
quotients. Formally, we say a history $\tup{O,f,<}$ uses \emph{unique values}
when $f(o_1) = m_1(u) => v_1$ and $f(o_2) = m_2(u) => v_2$ implies $o_1 = o_2$.

\begin{figure}[t]
  \footnotesize
  \begin{align*}
    \pred{total}(x,i,j)
      & = \sum_{i\le i'\le j'\le j} \#(x,i',j')
    \\
    \pred{before}_k(x,y)
      & = \bigvee_{0 \le i < k} \left(
        \begin{array}{l}
          \pred{total}(x,0,i) > 0 \ /| \\
          \pred{total}(y,0,i) = 0 /| \pred{total}(x,i\!+\!1,k) = 0
        \end{array} \right)
    \\
    \pred{match}(x,y)
      & = \pred{Push}(x) /| \pred{Pop}(y) /| \pred{SameVal}(x,y)
  \end{align*}
  \begin{align*}
    @Y_\mathrm{rv}
      & = \exists x, y.\ \pred{match}(x,y) /| \pred{before}_k(y,x)
    \\
    @Y_\mathrm{ev}
      & = \exists x, y, z.\ \pred{match}(x,z) /| \pred{EmptyVal}(y) \\
      & \quad /| \pred{before}_k(x,y) /| \pred{before}_k(y,z)
    \\
    @Y_\mathrm{fv}
      & = \exists x_1, x_2, y_1, y_2.\ \pred{match}(x_1,y_1) /| \pred{match}(x_2,y_2) \\
      & \quad /| \pred{before}_k(x_1,x_2) /| \pred{before}_k(y_2,y_1)
    \\
    @Y_\mathrm{lv}
      & = \exists x_1, x_2, y_1, y_2.\ \pred{match}(x_1,y_1) /| \pred{match}(x_2,y_2) \\
      & \quad /| \pred{before}_k(x_1,x_2) /| \pred{before}_k(y_1,y_2) /| \pred{before}_k(x_2,y_1)
  \end{align*}
  \caption{Four families of operation-counting formul\ae characterizing atomic
    data structure violations, parameterized by the interval length $k \in
    \protect\<Nats>$. 
    The predicates $\pred{Push}(x)$, $\pred{Pop}(x)$, $\pred{EmptyVal}(x)$ hold
    when $x$ is the label of a push, pop, or empty-pop operation, respectively,
    and $\pred{SameVal}(x,y)$ holds when $x$ and $y$ are labels with the
    same value, either in the argument or return position. All formul\ae are
    of size polynomial in $k$, and have quantifier rank $1$.
  }
  \label{fig:spec:ds}
\end{figure}

% \begin{figure}
%   \begin{align*}
%     @Y_\mathrm{lv}^n
%       & = \exists x_1, .., x_n, y_1, .., y_n.\ \bigwedge_{i=1}^n \pred{match}(x_i,y_i) \\
%       & \quad /| \pred{before}_k(y_1,x_n) /| \bigwedge_{i=2}^n \pred{before}_k(x_i,y_{i-1})
%     \\
%     @Y_\mathrm{ev}^n
%       & = \exists x_1, .., x_n, y, z_1, .., z_n.\ \bigwedge_{i=1}^n \pred{match}(x_i,z_i) /| \pred{EmptyVal}(y) \\
%       & \quad /| \pred{before}_k(x_1,y) /| \bigwedge_{i=2}^{n} \pred{before}_k(x_i,z_{i-1}) /| \pred{before}_k(y,z_n)
%   \end{align*}
%   \caption{Parameterized formulas.}
% \end{figure}

\begin{theorem}
  \label{lem:cutoff}

  The families $@Y_\mathrm{rv}$, $@Y_\mathrm{ev}$, $@Y_\mathrm{fv}$, and
  $@Y_\mathrm{lv}$, of operation-counting formul\ae have cutoffs $0$, $2$, $2$,
  and $3$, respectively.

\end{theorem}

\begin{proof}

  Here we prove the theorem for $@Y_\mathrm{rv}$ and $@Y_\mathrm{fv}$; the
  others follow similarly. Since $@Y_\mathrm{rv}$ does not discriminate
  between intervals, $h = \tup{O,<,f} |= @Y_\mathrm{rv}$ if{f} $A_0(h) =
  \tup{O,@0,f} |= @Y_\mathrm{rv}$. % Thus $@Y_\mathrm{rv}$ has cutoff $0$.

  Next, let $h = \tup{O,<,f} |= @Y_\mathrm{fv}$ and let $@l_1$, $@l_2$,
  $@l_1'$, and $@l_2'$ be the instances of the variables $x_1$, $x_2$, $y_1$,
  and $y_2$, respectively. Let $@Y_{\mathrm{fv},k}$ denote the formula
  parametrized by $k$ in the family $@Y_\mathrm{fv}$. By definition $h |=
  @Y_{\mathrm{fv},k}$, where $k = \len h$. We consider only the case when
  $\pred{total}(y_1)>0$ holds (the other case is similar). Besides the constraints on
  operation labels, the formula $@Y_{\mathrm{fv},k}$ states that all $@l_1$
  operations end before an $@l_1'$ operation starts and all $@l_2$ operations
  end before an $@l_2'$ operation starts. Let $I:O->[n]^2$ be the canonical
  representation of $h$. Also, let $j_1$ (resp., $j_2$) be the maximum upper
  bound of an interval associated to an $@l_1$ (resp., $@l_3$) operation, i.e.,
  \begin{align*}
    j_1 = \max\set{j:\exists o\in O.\ I(o) = [i,j] /| f(o)=@l_1}, \\
    j_2 = \max\set{j:\exists o\in O.\ I(o) = [i,j] /| f(o)=@l_2}.
  \end{align*}
  We define a weaker history $h'\preceq h$, that contains exactly the same set
  of operations as $h$ but it preserves only the ordering constraints $o_1<o_2$
  s.t. $I(o_1)\subseteq [0,j_1]$ and $I(o_2)\subseteq [j_1,\infty]$, or
  $I(o_1)\subseteq [0,j_2]$ and $I(o_2)\subseteq [j_2,\infty]$. The length of
  $h'$ is at most 2
  % (it can be 1 when there is no operation that starts after the last finishing
  % $@l_1$ operation and before the last $@l_3$ operation finishes),
  and $h'|= @Y_{\mathrm{fv},2}$. Since $A_2(h') = h'$, $@Y_\mathrm{rv}$ has
  cutoff $2$.
\end{proof}

\begin{corollary}

  The families of operation-counting formul\ae,
  \begin{align*}
    & @Y_\mathrm{stack} = @Y_\mathrm{rv} |/ @Y_\mathrm{ev} |/ @Y_\mathrm{lv}
    \text{, and } \\
    & @Y_\mathrm{queue} = @Y_\mathrm{rv} |/ @Y_\mathrm{ev} |/ @Y_\mathrm{fv}
  \end{align*}
  have cutoffs $3$ and $2$, respectively.

\end{corollary}

While $@Y_\mathrm{rv}$ and $@Y_\mathrm{fv}$ are complete, in the sense that
they characterize every possible remove and FIFO violation, the
$@Y_\mathrm{lv}$ and $@Y_\mathrm{ev}$ families are incomplete:
Example~\ref{ex:emptyv} demonstrates this for $@Y_\mathrm{ev}$ by exhibiting a
parameterized history $h(n)$ for $n \ge 1$ such that $h(n) \not\in
H(L_\mathrm{queue})$,\footnote{$L_\mathrm{queue}$ denotes an atomic FIFO queue
library with $\<push>$ and $\<pop>$ methods.} yet $h(n) |= \lnot
@Y_\mathrm{ev}$ for all $n > 1$. While it is theoretically possible to define a
family $@Y_{\mathrm{ev},k}'$ which catches all empty violations in histories
with interval length at most $k$ --- a fact that we demonstrate in
Section~\ref{sec:containers:complete} --- we do not currently know whether it
is possible to construct a formula which catches all empty violations in
histories with arbitrarily-large interval length.

\begin{example}
  \label{ex:emptyv}
  
  In the history of Figure~\ref{fig:history:emptyv}, $n$ pairs of {\sf push}
  and {\sf pop} operations ensure that throughout the span of the 
  $\<pop>(\mathtt{EMPTY})$, some element is always present;
  i.e.,~at every time between the call and return of $\<pop>(\mathtt{EMPTY})$,
  there exists some element $v \in \<Vals>$ such that $\<push>(v)$ has
  completed, yet $\<pop> => v$ has not yet begun.
  \begin{figure}[t]
    
\newcommand{\elemspan}[2][10mm] {
  \tikz[]{
    \node[lnode] (a1c) {};
    \node[lnode] (a1r) [right=3mm of a1c] {};
    \node[lnode] (r1c) [right=#1 of a1r] {};
    \node[lnode] (r1r) [right=3mm of r1c] {};  
    \draw[line width=2pt] (a1c) -- node[above] {$\<push>(#2)$} (a1r);
    \draw[dotted] (a1r) -- (r1c);
    \draw[line width=2pt] (r1c) -- node[above] {$\<pop>=>#2$} (r1r);
  }
}

\begin{tikzpicture}
  \scriptsize
  \tikzstyle{lnode}=[circle,draw,minimum size=2pt,inner sep=0pt]
  
  \node at (0,6mm) {\elemspan[8mm]{1}};
  \node at (14mm,12mm) {\elemspan[23mm]{2}};
  \node at (28mm,6mm) {\elemspan[12mm]{3}};
  \node at (34mm,12mm) (n1) {};
  \node (n2) [right=3mm of n1] {};
  \draw[dotted,line width=1pt] (n1) -- (n2);
  \node at (53mm,12mm) {\elemspan[12mm]{n\!-\!1}};
  \node at (61mm,6mm) {\elemspan{n}};

  % \node (x0) at (0,1) {};
  % \node (x1) at (5,1) {};
  % \draw[dotted] (x0) -- (x1);

  \node[lnode] (re1)  at (0mm,0) {};
  \node[lnode] (re2)  [right=60mm of re1] {};
  \draw[line width=2pt] (re1) -- node[below] {$\<pop> => \mathtt{EMPTY}$} (re2); 

\end{tikzpicture}

    \vspace{-1em}
    \caption{A family of empty violations, parameterized by $n \in 
      \protect\<Nats>$.}
    \label{fig:history:emptyv}
  \end{figure}
  It follows that any such history would not be included in the histories
  $H(L_\mathrm{queue})$ of an atomic queue library. The empty-violation family
  $@Y_\mathrm{ev}$ however only captures such violations for $n=1$, i.e.,~with
  a single {\sf push}-{\sf pop} pair spanning $\<pop>(\mathtt{EMPTY})$.

\end{example}

Despite the theoretical possibility for empty violations which only surface for
large $n$, our practical experience suggests small $n$ tends to suffice. The
bug $\text{B}_5$ of our static analysis experiment in
Section~\ref{sec:exp:static} surfaced as an empty violation, and was detected
with approximation $A_1$, implying $n \le 1$, i.e.,~where some unmatched
$\<push>$ operation precedes $\<pop>({\tt EMPTY})$. Similarly, although the
history of Figure~\ref{fig:stacks}(b) satisfies the empty-violation formula
$@Y_\mathrm{ev}$, so does the weaker length-$1$ history of
Figure~\ref{fig:stacks}(c).
Furthermore, we have found that the same bugs which manifest as empty
violations often manifest as order violations as well. The ABA bug of
Figure~\ref{fig:stacks}(a) is such a case: had the first thread executed
another $\<push>(4)$ before $\<push>(1)$, then the final $\<pop>$ of Thread~2
would have returned $4$, yielding the order violation $\<push>(4); \<push>(2);
\<pop> => 4$.


\subsection{Completeness for Bounded Interval Length}
\label{sec:containers:complete}

While Section~\ref{sec:containers:small} demonstrates complete families
$@Y_\mathrm{queue}$ and $@Y_\mathrm{stack}$ of operation counting formul\ae for
the $L_\mathrm{queue}$ and $L_\mathrm{stack}$ objects for histories of interval
length up to $2$ and $3$, respectively, this section exhibits complete
formul\ae for histories of any bounded interval length $k \in \<Nats>$.

For the rest of this section, we fix an interval length $k \in \<Nats>$ and
demonstrate that a bounded enumeration of histories suffices to characterize
every possible violation.

Given a history $h = \tup{O,<,f}$, an \emph{element} $\tup{o_1,o_2} \in O^2$ is
a pair of operations such that $f(o_1) = \<push>(v)$ and $f(o_2) = \<pop> => v$
for some $v \in \<Vals>$. We say that two distinct elements $\tup{o_1,o_2}$ and
$\tup{o_3,o_4}$ are \emph{redundant in $h$} when their operations have the same
pasts and futures,\footnote{Similarly to the definition of $\<past>$ from
Section~\ref{sec:counting:logic}, we define the \emph{future} of an operation
$o$ in an history $\tup{O,<,f}$ by $\<future>(o) = \set{o' \in O : o < o'}$.}
\begin{align*}
  \<past>(o_1) = \<past>(o_3) & & \<future>(o_1) = \<future>(o_3) \\
  \<past>(o_2) = \<past>(o_4) & & \<future>(o_2) = \<future>(o_4)
\end{align*}
(and thus will share the same canonical intervals). We say the element
$\tup{o_1,o_2}$ is \emph{redundant with $h$} when there exists an element of
$h$ with which it is redundant. We say that $h' = \tup{O \u O', <', f'}$
\emph{extends $h = \tup{O,<,f}$ with $O'$} when $o_1 < o_2$ if{f} $o_1 <' o_2$
for all $o_1, o_2 \in O$, and $f(o) = f'(o)$ for all $o \in O$.

\begin{lemma}

  Let $x \in \set{\mathrm{stack}, \mathrm{queue}}$.
  If $h'$ extends $h$ with $\set{o_1,o_2}$ and $\tup{o_1,o_2}$ is
  redundant in $h$, then $h \in H(L_x)$ if{f} $h' \in H(L_x)$.

\end{lemma}

% \begin{proof}
%   TODO FILL IT IN
% \end{proof}

Let $\mathbb{I}_k = \set{ [i,j] : i,j \in \<Nats>,\ 0 \le i \le j \le k }$ be
the finite set of integral intervals up to $k \in \<Nats>$, and suppose a fixed
order $\ll$ over the finite set $\mathbb{I}_k^2$. We define the function $H^1$
which maps any subset $\set{ \tup{I_{1,1}, I_{1,2}}, .., \tup{I_{n,1}, I_{n,2}}
} \subseteq \mathbb{I}_k^2$ to the history $\tup{O,<,f}$ where
\begin{align*}
  O
    & = \set{ \tup{i,1}, \tup{i,2} : 1 \le i \le n } \\
  \tup{i_1,j_1} < \tup{i_2,j_2}
    & \text{ if{f} } I_{i_1,j_1} < I_{i_2,j_2} \\
  f(\tup{i,j}) 
    & = \left\{
    \begin{array}{ll}
      \<push>(i) & \text{ if } j = 1 \\
      \<pop> => i & \text{ if } j = 2
    \end{array}
  \right.
\end{align*}
such that $\tup{I_{1,1}, I_{1,2}} \ll \tup{I_{2,1}, I_{2,2}} \ll .. \ll
\tup{I_{n,1}, I_{n,2}}$. Then we define the finite set $H^1_k = \set{ H^1(X) :
X \subseteq \mathbb{I}_k^2}$. Notice that the histories of $H^1_k$ do not
contain redundant elements. Finally, we define the finite subset $H^1_{x,k} =
H^1_k @\ H(L_x)$ of histories which are not admitted by $L_x$, for $x \in
\set{\mathrm{stack}, \mathrm{queue}}$, and enumerate them to write the formula
characterizing length $k$ violations to $L_x$:
\begin{align*}
  @Y_{x,k} = \bigvee \set{ @Y_h : h \in H^1_{x,k} }
\end{align*}
where $@Y_h$ characterizes the history $h = \tup{O,<,f}$ with operations
$\set{ o_1, .., o_n }$ and elements $@c$, and its extensions:
\begin{align*}
  @Y_h = \exists x_1, .., x_n.
  \!\!\! \bigwedge_{o_i < o_j} \!\!\! \pred{before}_k(x_i,x_j)
  \land \!\!\!\!\!\! \bigwedge_{\langle o_i,o_j \rangle \in @c} \!\!\!\!\!\! \pred{match}(x_i,x_j)
\end{align*}
Note that $@Y_{x,k}$ is polynomial in size for fixed $k \in \<Nats>$, and
has quantifier rank $1$.

\begin{theorem}
  Let $x \in \set{stack, queue}$.
  If $\len h \le k$ and $h$ uses unique values,
  then $h \in H(L_x)$ if{f} $h |= \lnot @Y_{x,k}$.
\end{theorem}

% \begin{proof}
%   TODO FILL THIS IN
% \end{proof}

% Theorems~\ref{lem:cutoff} and~\ref{lem:rep} imply that the $A_3$ approximation
% is guaranteed to discover any stack/queue violation in histories using unique
% values, modulo the empty violations of Example~\ref{ex:emptyv} for $n > 3$.
