%!TEX root = draft.tex
\section{Prioritized Exploration of Simple Clients}
\label{sec:ranking}

In principle, the Sections~\ref{sec:counting}--\ref{sec:reach} outline
a complete technique for detecting violations to observational refinement,
relying on a single ``most general client'' program which exercises every
library history which can possibly be exposed by any program. In applying
this technique however, we have found that existing program analysis tools can
have difficulty reasoning about this most general client, due to the high
degree of nondeterminism employed: this client calls an arbitrary number of
arbitrary methods in an arbitrary order.

In this section we propose another complete exploration scheme which reduces
the program analysis burden by considering, instead of one monolithic and
highly nondeterministic client, a sequence of simpler clients. In particular,
each client is a parallel composition of a fixed number of threads, each of which
calls one fixed method; of course, nondeterminism in the order in which threads
interleave the execution of their methods is still present. The main complication
in proposing such a scheme is in prioritizing the clients which are likely to
induce violations, in order to quickly converge in the case violating
observational refinement is possible. Formally we say a class $\pclass$ of
programs is \emph{complete} for a given specification $S$ when for any library
$L$ with the same methods as $S$ which does not observationally refine $S$, $L$
does not observationally refine $S$ with respect to $\pclass$. We prioritize
client exploration order according to a \emph{ranking function} $R : \pclass ->
\<Nats>$.

In what follows, we develop our clients and ranking functions assuming the very
simple setting without shared memory writes. Extending the following concepts
to allow shared-memory writes is straightforward, by using shared-memory reads
and writes to enforce a particular interval partitioning, and thus specific
counter values --- each client will correspond to a single counter valuation.

Our strategy for prioritizing clients follows two basic principles:
(1)~avoiding clients whose operation counts are included in the specification,
and (2)~preferring clients whose operation counts are ``near'' those included
in the specification. For example, while one $\mathrm{push}[a]$-operation
concurrent with two $\mathrm{pop}[a]$-operations would not be included in the
specification of a stack, the count having only one of each operation would; we
thus assign a low rank to the client concurrently invoking these three
operations. The relevance of this heuristic is based on the hypothesis that
programming errors are more likely to surface within a certain proximity of the
intended behavior. Plus, by prioritizing according to specification proximity,
and client size, we favor small manifestations of violations to observational
refinement for any given programming error.

\begin{definition}
  A \emph{counter client} is a program
  $$P_{\tilde\pi} = C_{f_1} \| C_{f_2} \| .. \| C_{f_k}$$
  for a given multiset $\tilde\pi = \mset{ f_1, f_2, .., f_k }$ of method names,
  where each $C_{f_i}$ simply calls method $f_i$ with arbitrary argument
  values.
\end{definition}

For the rest of this section, we fix a specification $S$, and the class
$\pclass = \set{ P_{\tilde\pi} \mid \tilde\pi : \meths{S} -> \<Nats> }$ of
counter clients for $S$.

We base our client ranking on a measure of similarity to allowed counter
valuations in the specification. Let $<Ops>$ denote the set of symbols $\oper{f}{\cval}{\rval}$,
for all $f\in\meths{S}$, $\cval,\rval\in\<Nats>$. The \emph{distance} $@D(\pi_1,\pi_2)$ between
valuations $\pi_1, \pi_2 : \<Ops> -> \<Nats>$ is the sum of differences:
\begin{align*}
  @D(\pi_1, \pi_2) = 
    \sum_{f[n,m] \in \<Ops>} \bigl| \pi_1(f[n,m]) - \pi_2(f[n,m]) \bigr|.
\end{align*}
A valuation $\pi : \<Ops> -> \<Nats>$ is said to \emph{match} $\tilde\pi :
\meths{S}-> \<Nats>$, written $\pi |= \tilde\pi$, when $\sum_{n,m} \pi(f[n,m])
= \tilde\pi(f)$ for all $f \in \meths{S}$.
The \emph{distance} $@D(\tilde\pi,S)$ between a valuation $\tilde\pi :
\meths{S} -> \<Nats>$ and the specification $S$ is the minimal non-zero
distance $\<min>\set{ @D(\pi,\pi') \mid \pi |= \tilde\pi \text{ and } \pi' \in
S}$ between any $\pi$ matching $\tilde\pi$ and a counter valuation $\pi'$
allowed by $S$, or $0$ if no such non-zero distance count exists.

The \emph{counter client ranking function} $R_S$ assigns each client
the sum of its size and distance from the specification $S$, or $@w$ when that
distance is zero:
\begin{align*}
  R_S( P_{\tilde\pi} ) = \left\{
  \begin{array}{ll}
    @D(\tilde\pi,S) + |\tilde\pi| - 2
    & \text{ when } @D(\tilde\pi,S) > 0 \\
    @w 
    & \text{ otherwise. }
  \end{array}
  \right.
\end{align*}
The simplest violation-inducing clients (\ie size and distance $1$) have rank
$0$. Note that rank-$@w$ clients cannot witness a violation to observational
refinement, since their 0-distance counts are included in the 
specification. Furthermore, all clients are rank @w (\resp their size plus $1$)
when $S$ permits all (\resp none) operation counts.

\vspace{-1eX}
\begin{lemma}
  The set $\set{ P_{\tilde\pi} \in \pclass \mid R_S(P_{\tilde\pi}) < @w }$ of
  finite-rank counter clients is complete.
  \vspace{-1eX}
\end{lemma}

Figure~\ref{fig:client:dist} illustrates a ranking calculation for clients of a
stack specification. Note that we differentiate two $\mathrm{pop}$ operations,
by writing $\mathrm{pop\_OK}$ for those that return some element, and
$\mathrm{pop\_EMP}$ for those which return empty. Taking, for instance, the
multiset $\tilde\pi = \mset{ \mathrm{push}^2, \mathrm{pop\_OK}^2 }$ of two push
operations concurrent with two pop operations, $\tilde\pi$ matches
the zero-distance valuation $\pi = \mset{ \mathrm{push}[d_1]^2,
\mathrm{pop\_OK}[d_1]^2 }$, the distance-one valuation $\mset{
\mathrm{push}[d_1], \mathrm{push}[d_2], \mathrm{pop\_OK}[d_1]^2 }$, and the
distance-two valuation $\mset{ \mathrm{push}[d_1]^2, \mathrm{pop\_OK}[d_2]^2}$,
for data values $d_1 \neq d_2 \in D$. Thus $R_S(P_{\tilde\pi}) =
@D(\tilde\pi,S) + |\tilde\pi| - 2 = 1 + 4 - 2 = 3$. As we discuss in
Section~\ref{sec:exp}, we find that violations to observational refinement
surface in low-rank clients.

% according to our ranking, the first three rounds ..
% \begin{align*}
%   \text{ round 1: } \quad
%   & \mathrm{push}[\ast] \big\| \mathrm{pop}[\ast,\<true>], 
%     \quad \mathrm{pop}[\ast,\<true>] \big\| \mathrm{pop}[\ast,\<false>] \\
%   \text{ round 2: } \quad
%   & \mathrm{pop}[\ast,\<true>]^2,
%     \quad \mathrm{push}[\ast]^2 \big\| \mathrm{pop}[\ast,\<true>],
%     \quad \mathrm{push}[\ast] \big\| \mathrm{pop}[\ast,\<true>]^2, \\
%   & \mathrm{push}[\ast] \big\| \mathrm{pop}[\ast,\<true>] \big\| 
%       \mathrm{pop}[\ast,\<false>],
%     \quad \mathrm{pop}[\ast,\<true>] \big\| \mathrm{pop}[\ast,\<false>]^2 \\
%   \text{ round 3: } \quad
%   & \mathrm{pop}[\ast,\<true>]^2 \big\| \mathrm{pop}[\ast,\<false>],
%   \quad \mathrm{push}[\ast]^3 \big\| \mathrm{pop}[\ast,\<true>], \ldots
% \end{align*}

\begin{figure*}[ht]
  \centering
  \scriptsize
  \begin{tabular}{|l|p{4mm}p{6mm}p{4mm}p{4mm}p{4mm}p{4mm}|}
    \hline
    & \multicolumn{6}{c|}{2 operations} \\
    \hline
    $\#\mathrm{push}$ 
    & 2 & 1 & 1 & 0 & 0 & 0 \\
    $\#\mathrm{pop\_OK}$ 
    & 0 & 1 & 0 & 2 & 1 & 0 \\
    $\#\mathrm{pop\_EMP}$
    & 0 & 0 & 1 & 0 & 1 & 2 \\
    \hline
    % $\Delta$ (1 val) 
    % & 0 & 0 & 0 & 2 & 1 & 0 \\
    $\Delta(\cdot,S)$ % (abs) 
    & 0 & 0,\underline{1} & 0 & 2 & 1 & 0 \\
    $R(\cdot)$
    & @w & 1 & @w & 2 & 1 & @w \\
    \hline
  \end{tabular}
  \begin{tabular}{%
    |p{4mm}p{6mm}p{4mm}p{6mm}p{6mm}p{4mm}p{4mm}p{4mm}p{4mm}p{4mm}|}
    \hline
    \multicolumn{10}{|c|}{3 operations} \\
    \hline
    3 & 2 & 2 & 1 & 1 & 1 & 0 & 0 & 0 & 0 \\
    0 & 1 & 0 & 2 & 1 & 0 & 3 & 2 & 1 & 0 \\
    0 & 0 & 1 & 0 & 1 & 2 & 0 & 1 & 2 & 3 \\
    \hline
    % 0 & 0 & 0 & 1 & 0 & 0 & 3 & 2 & 1 & 0 \\
    0 & 0,\underline{1} & 0 & \underline{1},2 & 0,\underline{1} & 0 & 3 & 2 & 1 & 0 \\
    @w & 2 & @w & 2 & 2 & @w & 4 & 3 & 2 & @w \\
    \hline
  \end{tabular}
  \\[1mm]
  \begin{tabular}{|l|p{4mm}p{6mm}p{4mm}p{8mm}p{6mm}p{4mm}p{6mm}p{6mm}p{6mm}%
    p{4mm}p{4mm}p{4mm}p{4mm}p{4mm}p{4mm}|}
    \hline
    & \multicolumn{15}{c|}{4 operations} \\
    \hline
    $\#\mathrm{push}$ 
    & 4 & 3 & 3 & 2 & 2 & 2 & 1 & 1 & 1 & 1 & 0 & 0 & 0 & 0 & 0 \\
    $\#\mathrm{pop\_OK}$ 
    & 0 & 1 & 0 & 2 & 1 & 0 & 3 & 2 & 1 & 0 & 4 & 3 & 2 & 1 & 0 \\
    $\#\mathrm{pop\_EMP}$
    & 0 & 0 & 1 & 0 & 1 & 2 & 0 & 1 & 2 & 3 & 0 & 1 & 2 & 3 & 4 \\
    \hline
    % $\Delta$ (1 val) 
    % & 0 & 0 & 0 & 0 & 0 & 0 & 2 & 1 & 0 & 0 & 4 & 3 & 2 & 1 & 0 \\
    $\Delta(\cdot,S)$ % (abs) 
    & 0 & 0,\underline{1} & 0 & 0,\underline{1},2 & 0,\underline{1} & 0 & \underline{2},3 & \underline{1},2 & 0,\underline{1} & 0 & 4 & 3 & 2 & 1 & 0 \\
    $R(\cdot)$
    & @w & 3 & @w & 3 & 3 & @w & 4 & 3 & 3 & @w & 6 & 5 & 4 & 3 & @w \\
    \hline
  \end{tabular}
  \caption{The calculation of the ranking function for
    a typical stack specification. % in zero-barrier executions.  
    All distances for 
    matching images are listed, with the minimal non-zero distance underlined.}
  \label{fig:client:dist}
  \vspace{-3eX}
\end{figure*}
