%!TEX root = draft.tex

\section{Motivating Example}
\label{sec:motivation}

Figure~\ref{fig:stacks} lists two implementations of an atomic stack-based data
structure which provides {\tt push} and {\tt pop} methods. The first lock-based
implementation ensures that methods execute atomically by holding a lock for
the duration of each operation, and stores {\tt push}ed elements in a
singly-linked list rooted at {\tt S->Top}. The second implementation of
Treiber~\cite{Treiber'86} stores its elements in the same singly-linked list
structure, but avoids blocking lock acquisitions in favor of non-blocking
compare-and-swap ({\tt CAS}) operations in order to maximize parallelism,
allowing methods to interleave their internal actions; in one atomic step, the
{\tt CAS} operation assigns {\tt S->Top = n} only if {\tt S->Top == t}.

\begin{figure*}[t]
  \lstset{numbers=left, 
          numberstyle=\tiny\tt, 
          stepnumber=1, 
          firstnumber=1,
          % numberfirstline=true,
          numbersep=4pt}
  \footnotesize
  \centering
  \begin{minipage}[b]{38mm}
    \begin{program}
struct node {
  int data;
  struct node *next;
}
struct stack {
  struct node *Top
}
struct stack *S;
    \end{program}
    \bigskip
    \lstset{numbers=none}
    \begin{minipage}[b]{17mm}
      \textbf{Program} \\[1em]
      \textbf{Thread 1}
      \begin{program}
push(1);
x = pop();
      \end{program}
    \end{minipage}
    \begin{minipage}[b]{16mm}
      \textbf{Thread 2}
      \begin{program}
y = pop();
push(2);
push(3);
z = pop();
      \end{program}
    \end{minipage}
  \end{minipage}
  \begin{minipage}[b]{44mm}
    \textbf{Lock-based stack}
    \begin{program}
void push(int v) {
  lock();
  struct node *n;
  n = malloc(sizeof( *x));
  n->data = v;
  n->next = S->Top;
  S->Top = n;
  unlock();
}

int pop() {
  lock()
  struct node *t = S->Top;
  if (t==NULL)
    return EMPTY;
  S->Top = t->next;
  unlock()
  return t-> data;
}
    \end{program}
  \end{minipage}
  \begin{minipage}[b]{50mm}
    \textbf{Treiber's stack}
    \begin{program}
void push(int v) {
  struct node *n,*t;
  n = malloc(sizeof( *n));
  n->data = v;
  do {
    struct node *t = S->Top;
    n->next = t;
  } while (! CAS (&S->Top, t, n))
}

int pop() {
  struct node *n,*t;
  do {
    *t = S->Top;
    if (t==NULL)
      return EMPTY;
    n = t->next;
  } while (! CAS (&S->Top, t, n))
  int result = t->data;
  free(t);
  return result;
}
    \end{program}
  \end{minipage}
  \begin{minipage}[b]{43mm}
    %
%\hspace{.2cm}
%\begin{tikzpicture}[node distance=.7cm]
%\tikzstyle{node}=[minimum size=10pt]
%\node[node] (x)  [] at (1, 0) {${\tt n} = {\tt 0xFFFF 0000}$};
%\node[node] (y)  [below of=x] {}; %label=above:$x$
%\draw[->] (x) -- (y); %node[draw=none,right] {$h$}
%\end{tikzpicture}
%\hspace{0.7cm}
%\begin{tikzpicture}[node distance=.7cm]
%\tikzstyle{node}=[minimum size=10pt]
%\node[node] (x)  [] at (1, 0) {{\tt free (0xFFFF0000)}};
%\node[node] (y)  [below of=x] {}; %label=above:$x$
%\draw[->] (x) -- (y); %node[draw=none,right] {$h$}
%\end{tikzpicture}
%\hspace{2.8cm}
%\begin{tikzpicture}[node distance=.7cm]
%\tikzstyle{node}=[minimum size=10pt]
%\node[node] (x)  [] at (1, 0) {${\tt n} = {\tt 0xFFFF 0000}$};
%\node[node] (y)  [below of=x] {}; %label=above:$x$
%\draw[->] (x) -- (y); %node[draw=none,right] {$h$}
%\end{tikzpicture}
%
%\vspace{-1cm}
%\[
%\underbrace{\<push>(1)_1\ldots \<ret>_1\ \<pop>_2\ldots ^{8}}_{\tt {\bf Thread 1}} \ddagger\
%\underbrace{\<pop>()_3\ldots \<ret>(1)_3\ z=1\ \<push>(2)_4\ldots \<ret>_4\ \<push>(3)_5\ldots \<ret>_5}_{\tt {\bf Thread 2}} \ddagger\ 
%\underbrace{^{8}\ldots  \<ret>(3)_2\ x=3\ \<pop>()_6\ldots \<ret>({\tt EMPTY})_6\ y={\tt EMPTY}}_{\tt {\bf Thread 1}}
%\]

\begin{tikzpicture}[node distance=.5cm]
\tikzstyle{node}=[minimum size=0pt]
\tikzstyle{nnode}=[minimum size=0pt,inner sep=0pt]
\tikzstyle{lnode}=[circle,draw,minimum size=2pt,inner sep=0pt]
\node[node] (x0)  [] at (0, 0) {{\tt{\bf Thread1}}};
\node[lnode] (x1)  [below of=x0,label=right:$\<push>(1)$] {};
\node[lnode] (x2)  [below of=x1,label=right:$\<ret>$] {}; %label=above:$x$
\node[lnode] (x3)  [below of=x2,label=right:$\<pop>()$] {}; 
\node[nnode] (x4)  [below of=x3,label=left:{\it preempted}] {}; 
\node[nnode] (x41)  [below= 2.5mm of x4,label=left:{\it at line 8}] {}; 

\node[node] (z0)  [below= 2mm of x1] {}; 
\node[node] (z1)  [left= 3mm of x1,yshift=-2mm] {{\tt n=0xFF}}; 

\node[node] (y0)  [right=7mm of x3,yshift=-2mm]  {{\tt{\bf Thread2}}};
\node[lnode] (y1)  [right = 15mm of x4,yshift=-2mm,label=left:$\<pop>()$] {}; 
\node[lnode] (y2)  [below of=y1,label=left:$\<ret>(1)$] {}; 
\node[lnode] (y3)  [below of=y2,label=left:${\tt y=1}$] {}; 
\node[lnode] (y4)  [below of= y3,label=left:$\<push>(2)$] {}; 
\node[lnode] (y5)  [below of=y4,label=left:$\<ret>$] {}; 
\node[lnode] (y6)  [below of=y5,label=left:$\<push>(3)$] {}; 
\node[lnode] (y7)  [below of=y6,label=left:$\<ret>$] {}; 

\node[node] (u0)  [below=1mm of y1] {}; 
\node[node] (u1)  [left=17mm of y1,yshift=-5mm] {{\tt free(0xFF)}}; 

\node[node] (v0)  [below=2mm of y6] {}; 
\node[node] (v1)  [left=18mm of y6,yshift=-5mm] {{\tt n=0xFF}}; 


\node[nnode] (x5)  [left=15mm of y7,yshift=-2mm] {}; %label=left:$^{8}$
\node[lnode] (x6)  [below of=x5,label=right:$\<ret>(3)$] {}; 
\node[lnode] (x7)  [below of= x6,label=right:${\tt x=3}$] {}; 

\node[lnode] (y8)  [right = 15mm of x7,yshift=-5mm,label=left:$\<pop>()$] {}; 
\node[lnode] (y9)  [below of =y8,label=left:$\<ret>({\tt EMPTY})$] {}; 
\node[lnode] (y10)  [below of= y9,label=left:${\tt z=EMPTY}$] {}; 


\draw[line width=2pt] (x1) -- (x2); %node[draw=none,right] {$h$}
\draw[] (x2) -- (x3);
\draw[line width=2pt] (x3) -- (x4);

\draw[line width=2pt] (y1) -- (y2);
\draw[] (y2) -- (y3);
\draw[] (y3) -- (y4);
\draw[line width=2pt] (y4) -- (y5);
\draw[] (y5) -- (y6);
\draw[line width=2pt] (y6) -- (y7);

\draw[dotted,line width=1pt] (x4) -- (x5); %node[draw=none,right] {$h$}
\draw[line width=2pt] (x5) -- (x6);
\draw[] (x6) -- (x7);

\draw[dotted] (y7) -- (y8);
\draw[line width=2pt] (y8) -- (y9);
\draw[] (y9) -- (y10);

\draw[->] (z0) -- (z1);
\draw[->] (u0) -- (u1);
\draw[->] (v0) -- (v1);

\end{tikzpicture}

  \end{minipage}
  \caption{Two implementations of a concurrent stack object, a stack-using
  program, and an execution using Treiber's stack. The {\tt pop} operation
  returns the {\tt EMPTY} when the stack is empty. The execution depicts calls,
  returns, and assignments, and time progresses downward.}
  \label{fig:stacks}
\end{figure*}

Unfortunately this nonblocking implementation suffers from a subtle concurrency
bug, now commonly known as an ``ABA'' bug~\cite{tr/ibm/Michael04}. This bug
manifests in the program of Figure~\ref{fig:stacks}, via the depicted
execution. Essentially, Thread~1 wrongfully assumes the absence of interference
from other threads on the successful {\tt CAS} operation. Thread~1 is preempted
right before executing its {\tt CAS} in the {\tt pop} method; at that moment,
its {\tt t} variable points to the first element in the list at address {\tt
0xFF} added by {\tt push(1)}, and {\tt n == NULL}. While Thread~2 updates the
list with two additional elements, added by {\tt push(2)} and {\tt push(3)},
the {\tt t} variable of Thread~1 still points to the list's first element at
address {\tt 0xFF}, which was freed by Thread~2's call to {\tt pop}, and
reallocated in the call to {\tt push(3)}. When Thread~1 resumes, its {\tt CAS}
succeeds, effectively removing two elements from the list instead of one. The
final {\tt pop} of Thread~2 thus erroneously returns {\tt EMPTY}. Intuitively,
this is a problem because the {\tt EMPTY} value should not have been returned
since more elements have been pushed than popped prior to Thread~2's final {\tt
pop} operation.

This bug exposes the fact that our {\tt CAS}-based implementation does not
conform to programmers' expectations of a stack object whose operations execute
atomically. In particular, the assignment {\tt z = EMPTY} should never have
been observed in an execution of the given program. This idea of conformance is
rigorously captured by the formal notion of \emph{observational refinement}.
Essentially, an implementation $L_1$ of a concurrent object ``refines'' another
implementation $L_2$ if every observable behavior of a program using $L_1$ is
also observable using $L_2$. This property clearly does not hold between the
{\tt CAS}-based and lock-based implementations of Figure~\ref{fig:stacks},
since {\tt y = 1; x = 3; z = EMPTY} is observable using the {\tt CAS}-based
implementation, yet not using the lock-based implementation.
