%!TEX root = draft.tex

\section{Motivating Example}
\label{sec:motivation}

Figure~\ref{fig:treiber} lists the implementation of the non-blocking Treiber's
stack~\cite{Treiber'86}, which provides {\tt push} and {\tt pop} methods. This
implementation stores {\tt push}ed elements in a singly-linked list rooted at
{\tt Top}, but avoids blocking lock acquisitions in favor of non-blocking
compare-and-swap ({\tt CAS}) operations in order to maximize parallelism,
allowing methods to interleave their internal actions; in one atomic step, the
{\tt CAS} operation assigns {\tt Top = n} only if {\tt Top == t}.

% The first lock-based implementation ensures that methods execute atomically by
% holding a lock for the duration of each operation, and stores {\tt push}ed
% elements in a singly-linked list rooted at {\tt S->Top}. The second
% implementation of Treiber~\cite{Treiber'86} stores its elements in the same
% singly-linked list structure,

\begin{figure}[t]
  \scriptsize
  \begin{minipage}[t]{45mm}
    \begin{program}
void push(int v) {
  struct node *n,*t;
  n = malloc(sizeof( *n));
  n->data = v;
  do {
    struct node *t = Top;
    n->next = t;
  } while (! CAS (&Top, t, n))
}

int pop() {
  struct node *n,*t;
  do {
    *t = Top;
    if (t==NULL) return EMPTY;
    n = t->next;
  } while (! CAS (&Top, t, n))
  int result = t->data;
  free(t);
  return result;
}
    \end{program}
  \end{minipage}
  \begin{minipage}[t]{35mm}
    \begin{program}
struct node {
  int data;
  struct node *next;
} *Top;  

void thread1() {
  push(1);
  int x = pop();
}

void thread2() {
  int y = pop();
  push(2);
  push(3);
  int z = pop();
}
    \end{program}
  \end{minipage}

  \caption{An implementation of Treiber's stack. The {\tt pop} operation
  returns the {\tt EMPTY} when the stack is empty.}
  \label{fig:treiber}
\end{figure}

\begin{figure}[t]
  \footnotesize
  \centering
  %!TEX root = ../draft.tex
\begin{tikzpicture}[node distance=.7cm]
  \scriptsize

  \tikzstyle{node}=[minimum size=0pt]
  \tikzstyle{nnode}=[minimum size=0pt,inner sep=0pt]
  \tikzstyle{lnode}=[circle,draw,minimum size=2pt,inner sep=0pt]
  \node[node] (x0)  [] at (0, 0) {{\tt{\bf Thread1}}};
  \node[lnode] (x1)  [right= .3cm of x0,label={[rotate=60,above=1mm]right:$\<push>(1)$}] {};
  \node[lnode] (x2)  [right of=x1,label={[rotate=60,above=1mm]right:$\<ret>$}] {}; %label=above:$x$
  \node[lnode] (x3)  [right of=x2,label={[rotate=60,above=1mm]right:$\<pop>()$}] {}; 
  \node[nnode] (x4)  [right of=x3] {}; 
  \node[nnode] (x41) [right= 2.5mm of x4,label={[above=1mm,xshift=15mm]above:{\it preemption at line 18}}] {};

  \node[node] (z0)  [right=3mm of x1] {}; 
  \node[node] (z1)  [below=3mm of x1,xshift=5mm] {{\tt n=0xFF}}; 

  \node[node] (y0)  [below=8mm of x3]  {{\tt{\bf Thread2}}};
  \node[lnode] (y1)  [below=10mm of x4,xshift=2mm,label={[rotate=60,above=1mm]right:$\<pop>()$}] {}; 
  \node[lnode] (y2)  [right of=y1,label={[rotate=60,above=1mm]right:$\<ret>(1)$}] {}; 
  \node[lnode] (y3)  [right of=y2,label={[rotate=60,above=1mm]right:${\tt y=1}$}] {}; 
  \node[lnode] (y4)  [right of= y3,label={[rotate=60,above=1mm]right:$\<push>(2)$}] {}; 
  \node[lnode] (y5)  [right of=y4,label={[rotate=60,above=1mm]right:$\<ret>$}] {}; 
  \node[lnode] (y6)  [right of=y5,label={[rotate=60,above=1mm]right:$\<push>(3)$}] {}; 
  \node[lnode] (y7)  [right of=y6,label={[rotate=60,above=1mm]right:$\<ret>$}] {}; 

  \node[node] (u0)  [right=1mm of y1] {}; 
  \node[node] (u1)  [below=3mm of y1,xshift=5mm] {{\tt free(0xFF)}}; 

  \node[node] (v0)  [right=2mm of y6] {}; 
  \node[node] (v1)  [below=3mm of y6,xshift=5mm] {{\tt n=0xFF}}; 

  \node[nnode] (x5)  [above=10mm of y7,xshift=2mm] {}; %label=left:$^{8}$
  \node[lnode] (x6)  [right of=x5,label={[rotate=60,above=1mm]right:$\<ret>(3)$}] {}; 
  \node[lnode] (x7)  [right of= x6,label={[rotate=60,above=1mm]right:${\tt x=3}$}] {}; 

  \node[lnode] (y8)  [below=10mm of x7,xshift=3mm,label={[rotate=60,above=1mm]right:$\<pop>()$}] {}; 
  \node[lnode] (y9)  [right of =y8,label={[rotate=60,above=1mm]right:$\<ret>({\tt EMPTY})$}] {}; 
  \node[lnode] (y10)  [right of= y9,label={[rotate=60,above=1mm]right:${\tt z=EMPTY}$}] {}; 

  \draw[line width=2pt] (x1) -- (x2); %node[draw=none,right] {$h$}
  \draw[] (x2) -- (x3);
  \draw[line width=2pt] (x3) -- (x4);

  \draw[line width=2pt] (y1) -- (y2);
  \draw[] (y2) -- (y3);
  \draw[] (y3) -- (y4);
  \draw[line width=2pt] (y4) -- (y5);
  \draw[] (y5) -- (y6);
  \draw[line width=2pt] (y6) -- (y7);

  \draw[dotted,line width=1pt] (x4) -- (x5); %node[draw=none,right] {$h$}
  \draw[line width=2pt] (x5) -- (x6);
  \draw[] (x6) -- (x7);

  \draw[dotted] (y7) -- (y8);
  \draw[line width=2pt] (y8) -- (y9);
  \draw[] (y9) -- (y10);

  \draw[->] (z0) -- (z1);
  \draw[->] (u0) -- (u1);
  \draw[->] (v0) -- (v1);

\end{tikzpicture}
 \\
  \parbox{0.8\linewidth}{(a) An execution $e$ of the program; it depicts calls,
  returns, and assignments, and time progresses from left to right.}

  \bigskip
  \begin{minipage}{43mm}
    %!TEX root = ../draft.tex
\begin{tikzpicture}[node distance=4mm]
  \scriptsize

  \tikzstyle{node}=[minimum size=0pt]
  \tikzstyle{nnode}=[minimum size=0pt,inner sep=0pt]
  \tikzstyle{lnode}=[circle,draw,minimum size=4pt,inner sep=0pt,fill]

  \node[lnode] (x1)  [label=left:$\<push>(1)$] at (0,0) {};
  \node[lnode] (x2)  [right=4mm of x1,yshift=4mm,label={[yshift=2mm,xshift=4mm] left:$\<pop> \Rightarrow 3$}] {}; 
  \node[lnode] (x3)  [right of=x1,yshift=-4mm,label={[yshift=-1mm,xshift=.5mm] left:$\<pop> \Rightarrow 1$}] {}; 

  \node[lnode] (x4)  [right=4mm of x3,label=below:$\<push>(2)$] {};
  \node[lnode] (x5)  [right=4mm of x4,label={[yshift=-1mm,xshift=-.5mm] right:$\<push>(3)$}] {};

  \node[lnode] (x6)  [right of=x5,yshift=6mm,label=above:$\<pop> \Rightarrow {\tt EMPTY}$] {};

  \draw[->,>=stealth',thick] (x1) -- (x2);
  \draw[->,>=stealth',thick] (x1) -- (x3);
  \draw[->,>=stealth',thick] (x3) -- (x4);
  \draw[->,>=stealth',thick] (x4) -- (x5);
  \draw[->,>=stealth',thick] (x5) -- (x6);
  \draw[->,>=stealth',thick] (x2) -- (x6);

\end{tikzpicture}
 \\
    (b) The history $h$ of the execution $e$.
  \end{minipage}
  \begin{minipage}{40mm}
    %!TEX root = ../draft.tex
\begin{tikzpicture}[node distance=.4cm]

\tikzstyle{node}=[minimum size=0pt]
\tikzstyle{nnode}=[minimum size=0pt,inner sep=0pt]
\tikzstyle{lnode}=[circle,draw,minimum size=4pt,inner sep=0pt,fill]
%\node[nnode] (x0) [] at (0,0) {};
\node[lnode] (x1)  [label=left:${\tt push}(1)$] at (0,0) {}; %right=4cm of x0,
\node[lnode] (x2)  [below of=x1,label=left:${\tt pop} \Rightarrow 3$] {}; 
\node[lnode] (x3)  [below of=x2,label=left:${\tt pop} \Rightarrow 1$] {}; 

\node[lnode] (x4)  [below of=x3,label=left:${\tt push}(2)$] {};
\node[lnode] (x5)  [below of=x4,label=left:${\tt push}(3)$] {};

\node[lnode] (x6)  [right=1.5 cm of x3,label=right:${\tt pop}\Rightarrow {\tt Empty}$] {};

\draw[->,>=stealth',thick] (x1) -- (x6); 
\draw[->,>=stealth',thick] (x2) -- (x6); 
\draw[->,>=stealth',thick] (x3) -- (x6); 
\draw[->,>=stealth',thick] (x4) -- (x6); 
\draw[->,>=stealth',thick] (x5) -- (x6); 


\end{tikzpicture}
 \\
    (c) A history weaker than $h$.
  \end{minipage}

  \bigskip
  %!TEX root = ../draft.tex
\begin{tikzpicture}[node distance=4mm]
  \scriptsize

  \tikzstyle{node}=[minimum size=0pt]
  \tikzstyle{nnode}=[minimum size=0pt,inner sep=0pt]
  \tikzstyle{lnode}=[circle,draw,minimum size=2pt,inner sep=0pt]
  \tikzstyle{opline}=[line width=2pt]

  \node[nnode] (t0) [label=below:{\tt 0}] {};
  \node[nnode] (t1) [right of=t0,label=below:1] {};
  \node[nnode] (t2) [right of=t1,label=below:2] {};
  \node[nnode] (t3) [right of=t2,label=below:3] {};
  \node[nnode] (t4) [right of=t3,label=below:4] {};
  \node[nnode] (t5) [right of=t4,label=below:5] {};
  \node[nnode] (t6) [right of=t5,label=below:6] {};
  \node[nnode] (t7) [right of=t6,label=below:7] {};
  \node[nnode] (t8) [right of=t7,label=below:8] {};
  \node[nnode] (t9) [right of=t8,label=below:9] {};
  \node[nnode] (t10) [right of=t9,label=below:10] {};
  \node[nnode] (t11) [right of=t10,label=below:11] {};
  \node[nnode] (t12) [right of=t11,label=below:12] {};
  \node[nnode] (t13) [right of=t12,label=below:13] {};
  \node[nnode] (t14) [right of=t13] {};

  \node[lnode] (x1)  [above=12mm of t0] {};
  \node[lnode] (x2)  [above=12mm of t1] {}; 
  \node[lnode] (x3)  [above=12mm of t2] {}; 
  \node[lnode] (x6)  [above=12mm of t10] {};

  \node[lnode] (y1)  [above=5mm of t3] {}; 
  \node[lnode] (y2)  [above=5mm of t4] {}; 
  \node[lnode] (y4)  [above=5mm of t6] {}; 
  \node[lnode] (y5)  [above=5mm of t7] {}; 
  \node[lnode] (y6)  [above=5mm of t8] {}; 
  \node[lnode] (y7)  [above=5mm of t9] {};

  \node[lnode] (y8)  [above=5mm of t12] {};
  \node[lnode] (y9)  [above=5mm of t13] {}; 

  \draw[->,>=stealth'] (t0) -- (t14);
  \draw[dotted] (x1) -- (t0);
  \draw[dotted] (x2) -- (t1);
  \draw[dotted] (x3) -- (t2);
  \draw[dotted] (x6) -- (t10);

  \draw[dotted] (y1) -- (t3);
  \draw[dotted] (y2) -- (t4);
  \draw[dotted] (y4) -- (t6);
  \draw[dotted] (y5) -- (t7);
  \draw[dotted] (y6) -- (t8);
  \draw[dotted] (y7) -- (t9);
  \draw[dotted] (y8) -- (t12);
  \draw[dotted] (y9) -- (t13);

  \draw[opline] (x1) -- node[draw=none,above] {$\<push>(1)$} (x2); 
  \draw[opline] (x3) -- node[draw=none,above] {$\<pop>=>3$} (x6);

  \draw[opline] (y1) -- node[draw=none,above] {$\<pop>=>1$} (y2);
  \draw[opline] (y4) -- node[draw=none,above,xshift=-1mm] {$\<push>(2)$} (y5);
  \draw[opline] (y6) -- node[draw=none,above,xshift=1mm] {$\<push>(3)$} (y7);
  \draw[opline] (y8) -- node[draw=none,above] {$\<pop>=>{\tt EMPTY}$} (y9);

\end{tikzpicture}
 \\
  (d) The history $h$ as an interval order.

  \bigskip
  %!TEX root = ../draft.tex
\begin{tikzpicture}[node distance=12mm]
  \scriptsize

  \tikzstyle{node}=[minimum size=0pt]
  \tikzstyle{nnode}=[minimum size=0pt,inner sep=0pt]
  \tikzstyle{lnode}=[circle,draw,minimum size=2pt,inner sep=0pt]
  \tikzstyle{opline}=[line width=2pt]

  \node[nnode] (t0) [label=below:{\tt 0}] {};
  \node[nnode] (t1) [right of=t0,label=below:1] {};
  \node[nnode] (t2) [right of=t1,label=below:2] {};
  \node[nnode] (t3) [right of=t2,label=below:3] {};
  \node[nnode] (t4) [right of=t3,label=below:4] {};
  \node[nnode] (t5) [right of=t4] {};

  \node[lnode] (x1)  [above=8mm of t0] {};
  \node[lnode] (x2)  [above=8mm of t1] {};
  \node[lnode] (x3)  [above=8mm of t3] {};

  \node[lnode] (y1)  [above=2mm of t1] {}; 
  \node[lnode] (y2)  [above=2mm of t2] {}; 
  \node[lnode] (y3)  [above=2mm of t3] {};
  \node[lnode] (y4)  [above=2mm of t4] {};

  \draw[->,>=stealth'] (t0) -- (t14);
  \draw[dotted] (x1) -- (t0);
  \draw[dotted] (x2) -- (t1);
  \draw[dotted] (y2) -- (t2);
  \draw[dotted] (x3) -- (t3);
  \draw[dotted] (y4) -- (t4);

  \draw[opline] (x1) -- node[draw=none,above] {$\<push>(1)$} (x1); 
  \draw[opline] (x2) -- node[draw=none,above] {$\<pop>=>3$} (x3);

  \draw[opline] (y1) -- node[draw=none,above] {$\<pop>=>1$} (y1);
  \draw[opline] (y2) -- node[draw=none,above] {$\<push>(2)$} (y2);
  \draw[opline] (y3) -- node[draw=none,above] {$\<push>(3)$} (y3);
  \draw[opline] (y4) -- node[draw=none,above,xshift=2mm] {$\<pop>=>{\tt EMPTY}$} (y4);

\end{tikzpicture}
 \\
  (e) The canonical representation of $h$.

  \caption{An execution and its history.}
  \label{fig:stacks}
\end{figure}

Unfortunately this nonblocking implementation suffers from a subtle concurrency
bug, now commonly known as an ``ABA'' bug~\cite{tr/ibm/Michael04}. This bug
manifests via the execution depicted in Figure~\ref{fig:stacks}(a).
Essentially, Thread~1 wrongfully assumes the absence of interference from other
threads on the successful {\tt CAS} operation. Thread~1 is preempted right
before executing its {\tt CAS} in the {\tt pop} method; at that moment, its
{\tt t} variable points to the first element in the list at address {\tt 0xFF}
added by {\tt push(1)}, and {\tt n == NULL}. While Thread~2 updates the list
with two additional elements, added by {\tt push(2)} and {\tt push(3)}, the
{\tt t} variable of Thread~1 still points to the list's first element at
address {\tt 0xFF}, which was freed by Thread~2's call to {\tt pop}, and
reallocated in the call to {\tt push(3)}. When Thread~1 resumes, its {\tt CAS}
succeeds, effectively removing two elements from the list instead of one. The
final {\tt pop} of Thread~2 thus erroneously returns {\tt EMPTY}. Intuitively,
this is a problem because the {\tt EMPTY} value should not have been returned
since more elements have been pushed than popped prior to Thread~2's final {\tt
pop} operation.

This bug exposes the fact that our {\tt CAS}-based implementation does not
conform to programmers' expectations of a stack object whose operations execute
atomically, e.g., by holding a lock for the duration of each operation.
In particular, the assignment {\tt z = EMPTY} should never have
been observed in an execution of the given program. 

This idea of conformance is rigorously captured by the formal notion of
\emph{observational refinement}. Essentially, an implementation $L_1$ of a
concurrent object ``refines'' another implementation $L_2$ if every observable
behavior of a program using $L_1$ is also observable using $L_2$. This property
clearly does not hold between the {\tt CAS}-based implementation of
Figure~\ref{fig:treiber} and a lock-based implementations, since {\tt y = 1; x
= 3; z = EMPTY} is observable using the {\tt CAS}-based implementation, yet not
using the lock-based implementation.
