%!TEX root = draft.tex
\section{Introduction}

\paragraph{Motivation}

Efficient implementations of (shared-memory) concurrent objects such as
semaphores, locks, atomic registers, and data structures (like sets, stacks,
and queues) are essential to modern computing. Several libraries implementing
operations for the manipulation of such objects are available (e.g., \cite{}).
Clients (or users) of these libraries assume that the latter are \emph{conform}
to \emph{reference implementations} where, typically operations (or methods) are
\emph{atomic}, as it helps apprehending the library behaviors. However, in order
to minimize synchronization overhead between concurrent object invocations,
implementors of concurrent objects need to relax atomicity, allowing operations
to be concurrently intertwined. Still, implementors must ensure that this
relaxation is fully transparent to the client, that is, the interactions of the
library with the client should indeed be conform to his expectations
(corresponding to the reference implementation). This is a notoriously hard and
error prone task. The necessary intricacy of these implementations is indeed a
breeding ground for insidious and difficult-to-diagnose bugs. Accordingly,
algorithmic methods for detecting conformance violation between implementations
of concurrent objets is in high demand. Conformance between libraries is
formally captured by the concept of \emph{observational refinement}: Given two
libraries $L_1$ and $L_2$, each of them implementing the operations of some
concurrent object, $L_1$ \emph{refines} $L_2$ if and only if for every
(library-client) program $P$, every computation of $P$ invoking $L_1$ is also
possible when $P$ invokes $L_2$ instead. The challenge we address in this paper
is to provide an efficient algorithmic approach for automatic detection of
refinement violations.

\paragraph{Observational refinement vs history inclusion}

Naturally, automating the verification of observational refinement is quite
challenging. The most immediate obstacle arises from the quantification over
the infinitely many possible library-client programs. The first contribution of
this paper is to provide a \emph{precise} characterization of the observational
refinement problem between two libraries $L_1$ and $L_2$ as an inclusion
problem between two sets of (happen-before) partial orders on their operations,
defined independently from their execution contexts.

More precisely, we associate to each execution a partial order on its
operations, called \emph{history}, where an operation $o_1$ is considered to
happen before an operation $o_2$ if $o_1$ returns (terminates) before $o_2$ is
called (starts). We prove that a library $L_1$ refines another one $L_2$ if and
only if the set $H(L_1)$ of histories (associated with the executions) of $L_1$
is included in the set $H(L_2)$ of histories of $L_2$.

\paragraph{History inclusion vs. linearizability}

The characterization above of observational refinement is a fundamental result
that offers a fresh view for reasoning about this problem. Indeed, the
principal approach for tackling the problem of checking observation refinement
in the literature is based on checking \emph{linearizability} \cite{}, i.e.,
that every execution in $L_1$ can be reordered into an execution of $L_2$ while
preserving the order between return and call actions. While linearizability
implies observational refinement \cite{}, we show that the converse does not
hold in general. In order to shed light on the subtle relation between these
concepts, we investigate the links between history inclusion and
linearizability. We prove that, interestingly, when $L_2$ is atomic (which is
typically the case for reference implementations as mentioned above), history
inclusion (and therefore observational refinement) between $L_1$ and $L_2$ is
equivalent to linearizability.

As we will see in the sequel, besides being a useful semantical means for
reasoning about observational refinement, our characterization of this problem
as a history inclusion problem leads to an efficient and powerful approach for
detecting refinement violations.

\paragraph{Complexity obstacles}

In fact, the problem of checking observational refinement is intrinsically
hard. Even for a single execution (of some library $L_1$), checking that its
history belongs to the set $H(L_2)$, for some fixed library $L_2$, is an
NP-hard problem \cite{}. As mentioned earlier, existing approaches for finding
refinement violations are based on finding linearizability bugs, and they do
that basically by enumerating all the (exponentially many) possible
linearizations of executions, which can only be applied for a limited number of
operation invocations. Moreover, from our result in \cite{}, observational
refinement is in general undecidable. (The proof there concerns
linearizability, but uses a specification that can be seen as an atomic
reference implementation.) To overcome these decidability and complexity
obstacles, we adopt an approach based on parameterized under-approximations.

\paragraph{Parameterized approximation schema}

The approach we introduce in this paper is based on fundamental properties of
library executions and their histories. The basic idea is to consider a notion
of \emph{weakening pre-order} $\preceq$ on histories as a means for
approximation: A history $h_1$ is \emph{weaker} than another history $h_2$,
written $h_1 \preceq h_2$, if $h_1$ is obtained from $h_2$ by relaxing some of
its order constraints. An important fact that makes this idea exploitable is
that, for every library $L$ (in some formally well defined sense), the set of
its histories $H(L)$ is \emph{downward closed} under $\preceq$, that is, if
$H(L)$ contains a history $h$, than it contains also all weaker histories than
$h$. Indeed, this fact leads to the principle of considering for histories $h
\in H(L_1)$ approximations $h' \preceq h$, such that checking $h' \in H(L_2)$
is \emph{tractable}. If $h' \not\in H(L_2)$, then we also have $h \not\in
H(L_2)$ since $H(L_2)$ is $\preceq$-downward closed, and therefore a refinement
violation is found.

So, the challenge we undertake is to provide an approximation schema based on
defining a series of functions $A_k$, parameterized with $k \in \mathbb{N}$,
such that for every $h$, we have (1) $A_k (h) \preceq h$, and (2) the test
$A_k(h) \in H(L_2)$ is decidable in \emph{polynomial time} (w.r.t. size of $h$).
Moreover, the schema should be \emph{complete} in the sense that there must be a
$k$ such that $h \preceq A_k(h)$, which means that if a violation exists, it
will be captured for a large enough parameter. Finally, and quite importantly,
we seek for an approximation schema that is easy to implement, and which is
able to catch refinement violations with small parameter values.

\paragraph{Bounded interval-length histories}

In the paper, we provide such an approximation schema, exploiting a fundamental
property of the executions of \emph{shared-memory} libraries. In fact, we show
that histories of such executions are not arbitrary orders, but particular
orders called \emph{interval orders} \cite{}. The main property of these orders
is that they admit a \emph{canonical representation} where each element $o$ is
mapped to an integer-bounded interval $I(o)$ such that for every two elements
$o_1$ and $o_2$, $o_1$ is before $o_2$ if and only if $I(o_1)$ ends before
$I(o_2)$ starts \cite{}. Also, based on this, interval orders have a (known)
notion of \emph{length} which corresponds to the maximal upper-bound of an
interval in their canonical representation \cite{}.

This leads us to the definition of parameterized weakening-based approximation
schema where the parameter $k$ corresponds naturally to the notion of length
mentioned above: For each $k \in \mathbb{N}$, the function $A_k$ maps each
history $h$ to a ($\preceq$-)weaker history $h'$ of interval-length $k$. In
this paper, we consider approximation functions that keep precise the last $k$
interval bounds, and abstract all the previous ones with equality.

\paragraph{Reduction to reachability using counting representations}

We show in the paper that interval-length bounding is a tractable approach for
refinement checking, and that it can be implemented efficiently. A key idea for
that is to use \emph{counting representations} for bounded interval-length
histories: each interval is represented by a counter corresponding to the
number of elements mapped to this interval in the canonical representation.
(Notice that there might be an unbounded number of elements mapped to a same
interval.) Indeed, representing histories as vectors of integers opens the door
to symbolic manipulation of sets of histories using arithmetical constraints.
In fact, we introduce for that purpose a simple logic, called OCL (for
Operation Counting Logic), that is suitable for reasoning about libraries
implementing common concurrent objects, and for which checking if a given
history satisfies a formula can be done in \emph{polynomial time}.

Moreover, counting representations provide a simple way for implementing a
monitor $P_k$ that tracks histories of interval-length $k$. In fact, we define
a \emph{polynomial time} algorithm that, given an execution, builds a
($k$-bounded interval-length) approximation history of it. Therefore, bounded
interval-length refinement checking between $L_1$ and $L_2$ can be reduced to a
\emph{reachability (or invariant) checking} problem in $P_k$ composed with
$L_1$, provided that the set of histories $A_k(H(L_2))$ is effectively defined
in OCL. We show that this is the case for reference implementations of the
usual concurrent objects such as collection objects (like stacks or queues),
semaphores, locks, etc. In fact, we show that it is possible for a wide class
of objects to construct systematically a formula characterizing precisely their
sets of histories.

\paragraph{Demonstrating feasibility}

The reduction of interval-bounded refinement checking to reachability checking
can be exploited in several ways. We demonstrate in the paper that our approach
is feasible both with a dynamic and a static state space exploration strategy.

In the dynamic case, we show experimentally two remarkable facts: First, our
approach is \emph{scalable}: its overhead is low and does not increase with the
length of computations, whereas the overhead of the standard approach (used in
existing tools such as Line-Up \cite{}) that is based on enumerating
linearizations of executions explodes exponentially. Second, our approach is
\emph{efficient} and well suited for catching refinement violations: most of
these violations in practice are detected with small bounds, i.e., for $k$
ranging from 0 to $2$, and only few (marginal) cases need greater bounds such
as 3 or 4. On this issue, we were actually able to prove that for the common
data structures of stacks and queues, refinement checking for order constraints
can actually be reduced (without loss of completeness) to bounded refinement
checking with small cut-off bounds, namely 3 and 2 respectively.

In the static case, using the fact that sets of histories can be represented
using arithmetical constraints, we can, for the first time check the existence
of refinement violations using existing tools for reachability analysis of
concurrent programs such as CSeq \cite{} (with backend CBMC \cite{}) and SMACK
\cite{} (with backend Corral \cite{}) that are based on efficient symbolic
encodings of sets of computations and the use of SAT/SMT solvers.

\paragraph{Summary}

To summarize, the paper presents the following main contributions:

\begin{itemize}

  \item A characterization of observational refinement as a history inclusion
  problem.

  \item The definition of general parameterized under-approximation schema for
  detecting observational refinement violations based on a weakening pre-order
  on histories.

  \item An instantiation of this schema based on the fact that histories of
  shared-memory libraries are interval-orders: The provided approximation
  schema is based on considering weaker histories of bounded interval-length.

  \item An efficient implementation of this schema using counting
  representations, and its use for reducing bounded refinement checking to a
  reachability problem, using symbolic representations of sets of
  interval-length bounded histories.

  \item Use of this reduction for defining dynamic and static analysis
  algorithms for detecting refinement violations, and showing the efficiency
  and the scalability of the approach in practice.

\end{itemize}
