%!TEX root = draft.tex

\section{Histories}

A \emph{history} $h = \tup{N,f,\ll}$ is a set $N \subset \<Nats>$ of operation
identifiers, along with a function $f: N -> (M \x \<Bools>)$ and partial order
$\ll$ on $N$. The \emph{history} $H(e)$ of a well-formed execution $e \in @S^*$
is given by
\begin{itemize}

  \item $N = \set{ \<id>(e(i)) : 0 \le i < |e| \text{ and } e(i) \in C }$,

  \item $f(i) = \left\{
  \begin{array}{ll}
    \tup{m,\<true>} & \text{ if } \tup{m,i} \in e \text{ and } \tup{i,m} \in e \\
    \tup{m,\<false>} & \text{ if } \tup{m,i} \in e \text{ and } \tup{i,m} \not\in e
  \end{array}
  \right.$

  \item $\<id>(e(i)) \ll \<id>(e(j))$ iff $i < j$, $e(i) \in R$, and $e(j) \in C$.

\end{itemize}
The \emph{histories} of a library $L$ are those of its admitted executions:
$H(L) = \set{ H(e) : e \in E(L) }$.

\begin{lemma}

  The history $H(e) = \tup{N,f,\ll}$ of a well-formed execution $e$ forms an
  interval order: $i_1 \ll j_1$  and $i_2 \ll j_2$ implies either
  $i_1 \ll j_2$ or $i_2 \ll j_1$.

\end{lemma}

A history $h_1 = \tup{N_1, f_1, \ll_1}$ is \emph{stricter than} a history $h_2
= \tup{N_2, f_2, \ll_2}$ (alternatively, $h_2$ is \emph{weaker than} $h_1$)
written $h_1 \preceq h_2$, when there exists a bijection $g : N_1 -> N_2$ such
that
\begin{itemize}

  \item $f_1(i) \preceq f_2(g(i))$ for each $i \in N_1$, and
  
  \item $i \ll_1 j$ if $g(i) \ll_2 g(j)$ for each $i,j \in N_1$.

\end{itemize}
where $\tup{m_1,b_1} \preceq \tup{m_2,b_2}$ iff $m_1 = m_2$ and $b_2
\Rightarrow b_1$. Two histories $h_1$ and $h_2$ are equivalent, denoted by $h_1
\equiv h_2$ iff $h_1 \preceq h_2$ and $h_2 \preceq h_1$.

\begin{lemma}
  
  The histories $H(L)$ of a library $L$ upward closed under $\preceq$.

\end{lemma}

\begin{proof}

  From the closure properties on $E(L)$.

\end{proof}

\section{Observational refinement is equivalent to history inclusion}

\begin{theorem}
For all libraries $L_1$ and $L_2$, 
$L_1 \leq L_2$ iff $H(L_1) \subseteq H(L_2)$.
\end{theorem}
\begin{proof}
($\Rightarrow$) Let $h\in H(L_1)$. As in Filipovic et al., construct a program $P_h$ (this LTS can be described as a program with shared variables) where we use shared variables to enforce the order constraints in $h$. Because $ReachStates_1(P_h\times L_1) \subseteq ReachStates_1(P_h\times L_2)$, there exists an execution $e$ of $P_h\times L_2$ s.t. $H(w)=h'$ and $h\preceq h'$. By the closure property on $L_2$, if $h'\in H(L_2)$ then $h\in H(L_2)$. 

($\Leftarrow$) Let $P$ be a program and let $e$ be an execution in $P\times L_1$. By hypothesis, $h=H(e)\in H(L_2)$. Let $w'\in L_2$ s.t. $H(w')=h$. Note that the projection of $e$ on library actions may be different than $w'$. We must show that by the closure properties on $P$, $E(P)$ contains the interleaving of $w'$ and the projection of $e$ on client actions.
\end{proof}

\section{Counter-based representations for interval orders}

Let $(A,<)$ be an interval order.
Given $a\in A$, let 
\[
pre^*(a)=\{b < a\mid b\in A\}\mbox{ and }post^*(a)=\{a < b\mid b\in A\}.
\]

\begin{lemma}[\cite{Rabinovitch197850}]
Let $(A,<)$ be an interval order. Then, the possible values of $pre^*(a)$ are linearly ordered by set inclusion. Similarly, for the possible values of $post^*(a)$.
\end{lemma}

Let $\sim$ be an equivalence relation over $A$ defined by $a\sim b$ iff $pre^*(a)=pre^*(b)$ and $post^*(a)=post^*(b)$.

The quotient of $(A,<)$ w.r.t. $\sim$ is denoted by $(A/_\sim,<_\sim)$.

Two labeled interval orders are $\sim$-equivalent if there exists an isomorphism between their quotients s.t. two equivalence classes are related by the isomorphism if they contain the same multiset of synopses. 

We represent an equivalence class w.r.t. the $\sim$-equivalence (i.e., a set of labeled interval orders) by integer maps $\iota:\<Nats>\times \<Nats>\times \Gamma\rightarrow \<Nats>$. First, given a labeled interval order $(A,\ell,<)$, we define a notation for equivalence classes of $\sim$ as follows:
\begin{itemize}
	\item let $pre^*(a_0)\subseteq pre^*(a_1)\subseteq \ldots\subseteq pre^*(a_n)$ be the set of different values of $pre^*(a)$. For any $a\in A$, let $low(a)=i$ s.t. $pre^*(a)=pre^*(a_i)$.
	\item let $post^*(b_1)\supseteq post^*(b_1)\supseteq \ldots\supseteq post^*(b_n)$ be the set of different values of $post^*(b)$. For any $a\in A$, let $high(a)=j$ s.t. $post^*(a)=post^*(a_j)$.
	\item the equivalence class $[a]$ is denoted by $[low(a),high(a)]$.
\end{itemize}

Given $h=(A,\ell,<)$, the integer map associated to $h$, $\Pi(h)$ is defined by: $\Pi(h)(low(a),high(a))(s)=$ the number of operations in $[a]$ having the synopsis $s$.

\begin{lemma}
For any two $\sim$-equivalent labeled interval orders $h$ and $h'$, $\Pi(h)=\Pi(h')$.
\end{lemma}

\section{Checking history inclusion using counting}

Given a library $L$, $\Pi(L)=\{\Pi(h)\mid h\in H(L)\}$.

\begin{lemma}
Given $L$ and $L'$ which are thread-independent, $H(L)\subseteq H(L')$ iff
\[
\Pi(L) \subseteq \Pi(L')
\]
\end{lemma}
 
\section{Reducing history inclusion to a state inclusion problem}

Define the monitor that counts.
 
\section{Bounding}

Bounding the set of input/output values and the size of the $\sim$-quotients ... Bounding the number of elements of the $\sim$-quotient corresponds to bounding the number of barriers.

%Bounding
%--------
%1. idea: bound "k" on Values & size of history quotients (or separately)
%2. this bounds the G and dom(\pi) in representations (G,\pi) of (f,<)
%3. decidability of H[k](L) \subseteq H[k](S) via \Pi[k](L) \subseteq \Pi[k](S),
%   when L and S use only finite domain variables
%4. better to compute invariant \varphi of \Pi[k](S), and \Pi[k](L) |= G \varphi
%5. better yet to fix complete enumeration G1, G2, .. of bounded barrier graphs
%   -- now \varphi is in Presburger arithmetic
%6. then demonstrate small "k" is usually adequate in theory and practice
