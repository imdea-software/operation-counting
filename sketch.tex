%!TEX root = draft.tex


\section{Reducing history inclusion to a state inclusion problem}

Define the monitor that counts.
 
\section{Bounding}

Bounding the set of input/output values and the size of the $\sim$-quotients ... Bounding the number of elements of the $\sim$-quotient corresponds to bounding the number of barriers.

%Bounding
%--------
%1. idea: bound "k" on Values & size of history quotients (or separately)
%2. this bounds the G and dom(\pi) in representations (G,\pi) of (f,<)
%3. decidability of H[k](L) \subseteq H[k](S) via \Pi[k](L) \subseteq \Pi[k](S),
%   when L and S use only finite domain variables
%4. better to compute invariant \varphi of \Pi[k](S), and \Pi[k](L) |= G \varphi
%5. better yet to fix complete enumeration G1, G2, .. of bounded barrier graphs
%   -- now \varphi is in Presburger arithmetic
%6. then demonstrate small "k" is usually adequate in theory and practice

TODO THIS DISCUSSION OUGHT TO GO ALONG WITH THE DISCUSSION ABOUT BOUNDING,
SINCE THAT IS WHERE WE WOULD LIKE TO KNOW WHAT THE LENGTH "$n$" OF AN INTERVAL
ORDER REPRESENTS.

Let $(A,<)$ be an interval order.
Given $a\in A$, let 
\[
pre^*(a)=\{b < a\mid b\in A\}\mbox{ and }post^*(a)=\{a < b\mid b\in A\}.
\]

\begin{lemma}[Rabinovitch~\cite{Rabinovitch197850}]

  Let $(A,<)$ be an interval order. Then, the possible values of $pre^*(a)$ are
  linearly ordered by set inclusion. Similarly, for the possible values of
  $post^*(a)$.

\end{lemma}

Let $\sim$ be an equivalence relation over $A$ defined by $a\sim b$ iff
$pre^*(a)=pre^*(b)$ and $post^*(a)=post^*(b)$.

The quotient of $(A,<)$ w.r.t. $\sim$ is denoted by $(A/_\sim,<_\sim)$.

Two labeled interval orders are $\sim$-equivalent if there exists an
isomorphism between their quotients s.t. two equivalence classes are related by
the isomorphism if they contain the same multiset of synopses.

We represent an equivalence class w.r.t. the $\sim$-equivalence (i.e., a set of
labeled interval orders) by integer maps $\iota:\<Nats>\times \<Nats>\times
\Gamma\rightarrow \<Nats>$. First, given a labeled interval order $(A,\ell,<)$,
we define a notation for equivalence classes of $\sim$ as follows:

\begin{itemize}

	\item let $pre^*(a_0)\subseteq pre^*(a_1)\subseteq \ldots\subseteq
pre^*(a_n)$ be the set of different values of $pre^*(a)$. For any $a\in A$, let
$low(a)=i$ s.t. $pre^*(a)=pre^*(a_i)$.

	\item let $post^*(b_1)\supseteq post^*(b_1)\supseteq \ldots\supseteq
post^*(b_n)$ be the set of different values of $post^*(b)$. For any $a\in A$,
let $high(a)=j$ s.t. $post^*(a)=post^*(a_j)$.

	\item the equivalence class $[a]$ is denoted by $[low(a),high(a)]$.

\end{itemize}

Given $h=(A,\ell,<)$, the integer map associated to $h$, $\Pi(h)$ is defined
by: $\Pi(h)(low(a),high(a))(s)=$ the number of operations in $[a]$ having the
synopsis $s$.

\begin{lemma}

  For any two $\sim$-equivalent labeled interval orders $h$ and $h'$,
  $\Pi(h)=\Pi(h')$.

\end{lemma}

Given a library $L$, $\Pi(L)=\{\Pi(h)\mid h\in H(L)\}$.

\begin{lemma}

  Given $L$ and $L'$ which are thread-independent, $H(L)\subseteq H(L')$ iff
  \[
    \Pi(L) \subseteq \Pi(L')
  \]

\end{lemma}
