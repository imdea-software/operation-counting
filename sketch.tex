%!TEX root = draft.tex


\section{Reducing history inclusion to a state inclusion problem}

Define the monitor that counts.
 
\section{Bounding}

Bounding the set of input/output values and the size of the $\sim$-quotients ... Bounding the number of elements of the $\sim$-quotient corresponds to bounding the number of barriers.

%Bounding
%--------
%1. idea: bound "k" on Values & size of history quotients (or separately)
%2. this bounds the G and dom(\pi) in representations (G,\pi) of (f,<)
%3. decidability of H[k](L) \subseteq H[k](S) via \Pi[k](L) \subseteq \Pi[k](S),
%   when L and S use only finite domain variables
%4. better to compute invariant \varphi of \Pi[k](S), and \Pi[k](L) |= G \varphi
%5. better yet to fix complete enumeration G1, G2, .. of bounded barrier graphs
%   -- now \varphi is in Presburger arithmetic
%6. then demonstrate small "k" is usually adequate in theory and practice
