%!TEX root = draft.tex
\section{Histories}

%THE POINT: REDUCING REFINEMENT TO SOMETHING WITHOUT A UNIVERSALLY QUANTIFIED
%PROGRAM: INCLUSION OF LIBRARY HISTORIES.

While observational refinement is defined over program executions, in the
following we introduce an alternative mathematical structure called
\emph{histories} which concisely summarize the executions admitted by a
library. Histories also provide a convenient foundation in which to formulate
our technical development. More precisely, we show that observational refinement
is equivalent to a plain inclusion between library histories. This result is important because
it reduces observational refinement to a property that doesn't contain the universal quantifier
over programs. TODO AWKWARD PHRASING

A \emph{history} $h = \tup{O,<,f}$ is a partial order $<$ on a set $O \subseteq
\<Ops>$ of operation identifiers labeled by $f: O -> (\<Methods> \x \<Bools>)$
for which $f(o) = \tup{\_,\<false>}$ implies $o$ is maximal in $<$.
The \emph{history} $H(e)$ of a well-formed execution $e \in @S^*$ labels each
operation with a method and completed status, and orders the non-overlapping
operations, according to:
\begin{itemize}

  \item $O = \set{ \<op>(e_i) : 0 \le i < |e| \text{ and } e_i \in C }$,

  \item $\<op>(e_i) < \<op>(e_j)$ iff $i < j$, $e_i \in R$, and $e_j \in C$.

  \item $f(o) = \left\{
  \begin{array}{ll}
    \tup{m,\<true>} & \text{ if } \tup{m,o} \in e \text{ and } \tup{o,m} \in e \\
    \tup{m,\<false>} & \text{ if } \tup{m,o} \in e \text{ and } \tup{o,m} \not\in e
  \end{array}
  \right.$

\end{itemize}
The \emph{histories} of a library $L$ are those of its admitted executions:
$H(L) = \set{ H(e) : e \in E(L) }$.

The \emph{histories} of a program $P$ are the histories of its admitted executions projected on call and return actions:
$H(P) = \set{ H(e|(C\cup R)) : e \in E(P) }$.

\begin{example}
  \label{ex:histories}

  TODO DRAW THE HISTORY/IES CORRESPONDING TO A/SOME PREVIOUS EXECUTION/S
  \todo{Show that histories are an abstraction of executions, i.e., there are
  different executions that have the same history}

\end{example}


A history $h_1 = \tup{O_1,<_1,f_1}$ is \emph{weaker than} another history $h_2
= \tup{O_2,<_2,f_2}$ written $h_1 \preceq h_2$, when there exists a bijection
$g : O_1' -> O_2$, where $O_1'$ is a subset of $O_1$ that contains all the completed
operations in $O_1$, i.e., 
$O_1=O_1\cup\set{o : f_1(o)=\tup{m,\<false>}\mbox{ with $m\in\<Methods>$}}$, such that
TODO WHY MUST WE ALLOW OMITTING OPERATIONS??
\begin{itemize}
  
  \item $o_1 <_1 o_2$ implies $g(o_1) <_2 g(o_2)$ for each $o_1, o_2 \in O_1'$, and
  
  \item $f_1(o) \preceq f_2(g(o))$ for each $o \in O_1'$.

\end{itemize}
where $\tup{m_1,b_1} \preceq \tup{m_2,b_2}$ iff $m_1 = m_2$ and $b_1 => b_2$.
Two histories $h_1$ and $h_2$ are \emph{equivalent}, denoted by $h_1 \equiv h_2$ iff
$h_1 \preceq h_2$ and $h_2 \preceq h_1$.

%A history $h_1 = \tup{O_1,<_1,f_1}$ is \emph{quasi-equivalent} to another history $h_2
%= \tup{O_2,<_2,f_2}$ written $h_1 \sim h_2$, when there exists a bijection
%$g : O_1 -> O_2$ such that
%\begin{itemize}
%  
%  \item $o_1 <_1 o_2$ implies $g(o_1) <_2 g(o_2)$ for each $o_1, o_2 \in O_1$, and
%
%  \item $o_1 <_2 o_2$ implies $g^{-1}(o_1) <_1 g^{-1}(o_2)$ for each $o_1, o_2 \in O_2$, and
%  
%  \item $f_1(o) \preceq f_2(g(o))$ for each $o \in O_1$.
%
%\end{itemize}
%%where $\tup{m_1,b_1} \preceq \tup{m_2,b_2}$ iff $m_1 = m_2$ and $b_1 => b_2$.


\begin{lemma}\label{lemma:lib_closure}
  
The set of histories $H(L)$ of a library $L$ is downward closed under $\preceq$, i.e., whenever some history $h$ belongs to $H(L)$ all the histories $h'$ with $h'\preceq h$ belong also to $H(L)$. %Moreover, if a history $h$ belongs to $H(L)$ then all the histories $h'=\tup{O,<,f}$ obtained from $h$ by adding pending operations, i.e., operations $o$ with $f(o)=\<false>$, belong also to $H(L)$.

\end{lemma}

\begin{proof}

  From the closure properties on $E(L)$.

\end{proof}

\begin{lemma}\label{lemma:lib_exec}
  
  $E(L)$ contains all executions with histories in $H(L)$:
  \begin{align*}
    h \in H(L) \text{ if and only if } \set{ e \in @S^* : H(e) = h } \subseteq E(L)
  \end{align*}

\end{lemma}

\begin{proof}

  From the closure properties on $E(L)$.

\end{proof}

To avoid an overload of notation, for any two sets of histories $H$ and $H'$,
$H\subseteq H'$ denotes the fact that for every history $h\in H$ there exists a
history $h'\in H'$ s.t. $h$ and $h'$ are equivalent.

\begin{theorem}
  \label{th:equiv}

  $L_1 \leq L_2$ iff $H(L_1) \subseteq H(L_2)$.

\end{theorem}

\begin{proof}
  
  ($=>$) Let $h = \tup{O,<,f} \in H(L_1)$; we show $h \in H(L_2)$ by
  constructing a program $P_h$ which only admits executions with histories
  stronger than $h$:
  \begin{align*}
    \forall e \in E(P_h).\ |(e|@S)| = n => h \preceq H(e) \text{,}
  \end{align*}
  where $n = |\set{ o \in O : f(o) = \tup{\_,\<true>}}|$ is the number of
  completed operations in $h$.
  Given such a program $P_h$, taking any execution $e_1 \in E(P_h \x L_1)$
  with $|(e_1|@S)| = n$, we must also have an execution $e_2 \in E(P_h \x L_2)$
  such that $(e_2|@S) = (e_1|@S)$ by definition of $L_1 <= L_2$. Since
  $|(e_2|@S)| = n$ and $e_2 \in E(P_h)$,\footnote{Technically, $e_2$ may contain
  internal actions of $P_h$ and $L_2$ both, so by writing $e_2 \in E(P_h)$,
  resp., $e_2 \in E(L_2)$, we mean the projection of $e_2$ to $P_h$'s actions,
  resp., $L_2$'s actions.}
  we also know that $h \preceq H(e_2)$, and since $e_2 \in E(L_2)$, we have
  $H(e_2) \in H(L_2)$, along with any history weaker than $H(e_2)$, namely $h$.

  We construct $P_h = \tup{Q,@S,q_0,@d}$ over alphabet $@S = C \u R \u \set{a}$
  whose states $Q = 2^O$ map operations to a ``completed'' status. The initial
  state is $q_0 = \set{ o |-> \<false> : o \in O}$. Transitions are given by,
  \begin{align*}
    & \text{for each } q,o,m \text{ such that } \lnot q(o)
      \text{, and } q(o') \text{ for all } o' < o \text{:} \\
    & \qquad q \xrightarrow{\tup{m,o}} q \\
    & \text{for each } q,o,m \text{ such that } f(o) = \tup{m,\<true>} \text{:} \\
    & \qquad q[o |-> \<false>] \xrightarrow{\tup{o,m}; a} q[o |-> \<true>]
  \end{align*}
  In any execution $e \in E(P_h)$, the call action $\tup{o,\_}$ of any
  operation $o$ must come after the return action $\tup{\_,o'}$ of each $o'<o$.
  Furthermore, all completed operations of $h$ are completed in $e$ if and
  only if $(e|@S) = a^n$. It follows that $|(e|@S)| = n => h \preceq H(e)$.

  ($\Leftarrow$) TODO MIKE IS NOT CONVINCED OF THIS DIRECTION...
  Let $P$ be a program and let $e$ be an execution in the
  composition of $P$ with $L_1$, $P\times L_1$. We show that $e$ is also an
  execution of $P\times L_2$, which implies that the projection of $e$ on
  program actions $e|@S_P$ belongs to $E(P \x L_2)|@S_P$.

  By hypothesis, the history of $e$ is also a history of the library $L_2$,
  i.e., $H(e)\in H(L_2)$. Let $l$ be the projection of $e$ on library actions.

  % and $l'$ an execution of $L_2$ whose history is $h$.   
 
  By Lemma~\ref{lemma:lib_exec}, $l$ is also an execution of $L_2$. Therefore,
  $e$ is also an execution of $P\times L_2$. \hfill $\Box$

  % Note that the projection of $e$ on library actions may be different than
  % $w'$. We must show that by the closure properties on $P$, $E(P)$ contains 
  % the interleaving of $w'$ and the projection of $e$ on client actions.

\end{proof}
