\section{Operation Histories}

While observational refinement is defined over program executions, in the
following we introduce an alternative mathematical structure called
\emph{histories} which concisely summarize the executions admitted by a
library. Histories also provide a convenient foundation in which to formulate
our technical development.

A \emph{history} $h = \tup{O,f,<}$ is a partial order $<$ on a set $O \subseteq
\<Ops>$ of operation identifiers labeled by $f: O -> (\<Methods> \x \<Bools>)$.
The \emph{history} $H(e)$ of a well-formed execution $e \in @S^*$ labels each
operation with a method and completed status, and orders the non-overlapping
operations, according to:
\begin{itemize}

  \item $O = \set{ \<op>(e_i) : 0 \le i < |e| \text{ and } e_i \in C }$,

  \item $f(o) = \left\{
  \begin{array}{ll}
    \tup{m,\<true>} & \text{ if } \tup{m,o} \in e \text{ and } \tup{o,m} \in e \\
    \tup{m,\<false>} & \text{ if } \tup{m,o} \in e \text{ and } \tup{o,m} \not\in e
  \end{array}
  \right.$

  \item $\<op>(e_i) < \<op>(e_j)$ iff $i < j$, $e_i \in R$, and $e_j \in C$.

\end{itemize}
The \emph{histories} of a library $L$ are those of its admitted executions:
$H(L) = \set{ H(e) : e \in E(L) }$.

\begin{example}
  \label{ex:histories}

  TODO DRAW THE HISTORY/IES CORRESPONDING TO A/SOME PREVIOUS EXECUTION/S
  
\end{example}

A history $h_1 = \tup{O_1,f_1,<_1}$ is \emph{weaker than} another history $h_2
= \tup{O_2,f_2,<_2}$ written $h_1 \preceq h_2$, when there exists a bijection
$g : O_1 -> O_2$ such that
\begin{itemize}

  \item $f_1(o) \preceq f_2(g(o))$ for each $o \in O_1$, and
  
  \item $o_1 <_1 o_2$ implies $g(o_1) <_2 g(o_2)$ for each $o_1, o_2 \in O_1$.

\end{itemize}
where $\tup{m_1,b_1} \preceq \tup{m_2,b_2}$ iff $m_1 = m_2$ and $b_1 => b_2$.
Two histories $h_1$ and $h_2$ are equivalent, denoted by $h_1 \equiv h_2$ iff
$h_1 \preceq h_2$ and $h_2 \preceq h_1$.

\begin{lemma}
  
  The histories $H(L)$ of a library $L$ downward closed under $\preceq$.

\end{lemma}

\begin{proof}

  From the closure properties on $E(L)$.

\end{proof}

\begin{theorem}

  $L_1 \leq L_2$ iff $H(L_1) \subseteq H(L_2)$.

\end{theorem}

\begin{proof}

  ($\Rightarrow$) Let $h\in H(L_1)$. As in Filipovic et al., construct a
  program $P_h$ (this LTS can be described as a program with shared variables)
  where we use shared variables to enforce the order constraints in $h$.
  Because $ReachStates_1(P_h\times L_1) \subseteq ReachStates_1(P_h\times
  L_2)$, there exists an execution $e$ of $P_h\times L_2$ s.t. $H(w)=h'$ and
  $h\preceq h'$. By the closure property on $L_2$, if $h'\in H(L_2)$ then $h\in
  H(L_2)$.

  ($\Leftarrow$) Let $P$ be a program and let $e$ be an execution in $P\times
  L_1$. By hypothesis, $h=H(e)\in H(L_2)$. Let $w'\in L_2$ s.t. $H(w')=h$. Note
  that the projection of $e$ on library actions may be different than $w'$. We
  must show that by the closure properties on $P$, $E(P)$ contains the
  interleaving of $w'$ and the projection of $e$ on client actions.

\end{proof}
