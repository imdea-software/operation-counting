%!TEX root = draft.tex
\section{History Inclusion}
\label{sec:histories}

Though we seek to develop automated techniques to check observational
refinement between libraries, the definition of Section~\ref{sec:refinement}
does not suggest any practical means; it only suggests enumerating every
possible execution of every possible program. In this section we introduce an
equivalent notion based on concise abstractions of program executions called
\emph{histories}. Besides being independent of programs, this equivalent notion
helps expose the structure of the refinement problem, and suggests practical
means of automation which we develop in Section~\ref{sec:counting}.

\subsection{Histories}

For given sets $\<Methods>$ and $\<Vals>$ of methods and values, we fix a set
$\<Labels> = \<Methods> \x \<Vals> \x (\<Vals> \u \set{\bot})$ of
\emph{operation labels}, and denote the label $\tup{m,u,v}$ by $m(u) => v$. A
\emph{history} $h = \tup{O,<,f}$ is a partial order $<$ on a set $O \subseteq
\<Ops>$ of operation identifiers labeled by $f: O -> \<Labels>$ for which $f(o)
= m(u) => \bot$ implies $o$ is maximal in $<$. The \emph{history} $H(e)$ of
a well-formed execution $e \in @S^*$ labels each operation with a method-call
summary, and orders non-overlapping operations:
\begin{itemize}

  \item $O = \set{ \<op>(e_i) : 0 \le i < |e| \text{ and } e_i \in C }$,

  \item $\<op>(e_i) < \<op>(e_j)$ if{f} $i < j$, $e_i \in R$, and $e_j \in C$.

  \item $f(o) = \left\{
  \begin{array}{ll}
    m(u) => v     & \text{if } m(u)_o \in e \text{ and } \<ret>(v)_o \in e \\
    m(u) => \bot  & \text{if } m(u)_o \in e \text{ and } \<ret>(\_)_o \not\in e
  \end{array}
  \right.$

\end{itemize}
The histories admitted by $L$ are $H(L) = \set{ H(e) : e \in E(L) }$.

\begin{example}

The history $h$ of the execution $e$ in Figure~\ref{fig:stacks}(a) is pictured in Figure~\ref{fig:stacks}(b).
The order relation is depicted by arrows (transitive orderings are omitted).
Note that a history abstracts away the order between successive call, resp, return, actions. Therefore, an execution obtained from
$e$ by reversing the order between the call action of the $\<pop>=>1$ operation (from the second thread) and the call action
of the $\<pop>=>3$ operation (from the first thread) has the same history $h$. This holds also for an execution
obtained from $e$ by reversing the order between the return action of the $\<push>(3)$ operation and
the return action of the $\<pop>=>3$ operation.

\end{example}

While the general concept of histories allows arbitrary partial orders of
operations, any history $H(e)$ arising from an LTS execution $e$ falls into a
restricted class called \emph{interval orders}. Intuitively, this is because
our execution model assumes that operations share a common notion of global
time: the actions in an execution are linearly ordered.
%the \emph{pasts} of any two operations (formalized in
%Section~\ref{sec:counting} as the set of operations that have completed before
%a given operation starts) cannot be incomparable.

\begin{definition}

  An \emph{interval order} is a partial order $\tup{O,<}$ such that
  $o_1 < o_3$ and $o_2 < o_4$ implies $o_1 < o_4$ or $o_2 < o_3$.

\end{definition}

\begin{lemma}
  \label{lem:intervals}

  The history $H(e) = \tup{O,f,<}$ of a well-formed execution $e$ forms an
  interval order $\tup{O,<}$.

\end{lemma}

\begin{proof}

  Suppose $o_1 < o_3$ and $o_2 < o_4$ in $H(e)$, and fix $i_1, i_2, i_3, i_4$
  such that $e_{i_1}$ and $e_{i_2}$ are the return actions of $o_1$ and $o_2$,
  and $e_{i_3}$ and $e_{i_4}$ are the call actions of $o_3$ and $o_4$; note
  that $i_1 < i_3$ and $i_2 < i_4$. Since $<$ linearly orders $\set{i_1, i_2,
  i_3, i_4}$, either $i_1 < i_4$, in which case $o_1 < o_4$, or $i_4 < i_1$, in
  which case $i_2 < i_4 < i_1 < i_3$, so $o_2 < o_3$.
\end{proof}

\begin{example}
  \label{ex:histories}

 Figure~\ref{fig:stacks}(d) pictures the history $h$ in Figure~\ref{fig:stacks}(b) as an interval order.
 Each operation is represented by an integer-bounded interval on the real number line such that
 $o_1<o_2$ iff the interval associated to $o_1$ finishes before the interval associated to $o_2$.
  
\end{example}

\noindent
In this work we consider only histories of well-formed executions, i.e.,~those
forming interval orders, without explicit qualification.

We define an order relation between histories, which is essential for proving
the main result of this section but also, for defining our approximation for
refinement checking. Basically, a history $h_1$ is \emph{weaker than} another
history $h_2$ if $h_2$ contains all the completed operations in $h_1$ and it
preserves the order between these operations. The pending operations of $h_1$
can be either omitted or completed in $h_2$.

\begin{definition}
Let $h_1 = \tup{O_1,<_1,f_1}$ and $h_2 = \tup{O_2,<_2,f_2}$. We say $h_1$ is
\emph{weaker than} $h_2$, written $h_1 \preceq h_2$, when there exists an
injection $g: O_2 -> O_1$ such that
\begin{itemize}

  \item $o \in \<range>(g)$ when $f_1(o) = m(u) => v$ and $v \neq \bot$,

  \item $g(o_1) <_1 g(o_2)$ implies $o_1 <_2 o_2$ for each $o_1, o_2 \in O_2$,

  \item $f_1(g(o)) \ll f_2(o)$ for each $o \in O_2$.

\end{itemize}
where $(m_1(u_1) => v_1) \ll (m_2(u_2) => v_2)$ if{f} $m_1 = m_2$, $u_1 =
u_2$, and $v_1 \in \set{ v_2, \bot }$. We say $h_1$ and $h_2$ are
\emph{equivalent} when $h_1 \preceq h_2$ and $h_2 \preceq h_1$. 
\end{definition}

\begin{example}

Figure~\ref{fig:stacks}(c) pictures a history $h'$ weaker than the history $h$ in Figure~\ref{fig:stacks}(b).
Note that $h'$ contains two pending $\<pop>=>\bot$ operations, one of them is completed in $h$ 
(it corresponds to the $\<pop>=>3$ operation) and one of them is omitted in $h$.

\end{example}

Throughout this work we do not distinguish between equivalent histories, and we
assume every set $H$ of histories is closed under inclusion of equivalent
histories, i.e.,~if $h_1$ and $h_2$ are equivalent and $h_1 \in H$, then $h_2
\in H$ as well.

\subsection{History Inclusion is Equivalent to Refinement}

We prove that refinement between two libraries is equivalent to containment
between their history sets. Before that, we give two lemmas which characterize
the set of histories $H(L)$ of a library $L$. First, the closure properties on
$E(L)$ imply that whenever $H(L)$ contains some history $h_1$, it also contains
all histories $h_2$ weaker than $h_1$. Thus, if $h_1$ is the history of an
execution $e$ and $h_2\preceq h_1$, then there exists another execution $e'$
such that $e ~> e'$ and $h_2$ is the history of a prefix of $e'$. Being the
history of a prefix of $e'$ allows $h_2$ to have less completed operations than
$h_1$.

\begin{lemma}
  \label{lem:lib:closed}

  If $h_1 \in H(L)$ and $h_2 \preceq h_1$ then $h_2 \in H(L)$.

\end{lemma}

%\begin{proof}
%
%  TODO GIVE A PROPER PROOF
%
%\end{proof}

In general, a history is an abstraction of a set of executions, that can differ only in the order between successive call (resp., return) actions.
For example, $\<push>(0)_1\ \<pop>_2\ \<ret>_1\ \<ret>(0)_2$ and $ \<pop>_2\ \<push>(0)_1\ \<ret>(0)_2\ \<ret>_1$
have the same history. The closure properties on $E(L)$ imply that a library $L$ contains all executions $e$ with $H(e)=h$, whenever
$h\in H(L)$.

%have the same history, i.e., $H(e)=H(e')$. The next lemma shows that $H(L)$ contains some history 
%
%which are
%uninterrupted by return (resp., call) actions
%
%histories abstract away the order between successive call (resp., return) actions, which are
%uninterrupted by return (resp., call) actions, i.e. any two executions

\begin{lemma}
  \label{lem:lib:execs}

  $H(e) \in H(L)$ if{f} $e|(C\u R)^* \in E(L)$.

\end{lemma}

%\begin{proof}
%
%  TODO GIVE A PROPER PROOF
%
%\end{proof}

%TODO SAY WHERE WE APPLY THE CLOSURE

Lemmas~\ref{lem:lib:closed} and~\ref{lem:lib:execs} ultimately imply the
equivalence between refinement and history inclusion. Essentially, any given
history $h$ of a library $L_1$ can be captured by a program $P_h$ whose
observations witness the ordering among $h$'s operations; the refinement $L_1
\leq L_2$ implies that $L_2$ also admits those observations, and ultimately
$h$ \alert{(the observations of $P_h$ may witness a history stronger than $h$
but, if that history belongs to $H(L_2)$, then by Lemma~\ref{lem:lib:closed}, also $h$ does)}. 
For the reverse direction, it follows from Lemma~\ref{lem:lib:execs} that
history inclusion implies inclusion of executions, and ultimately observations.

\begin{theorem}
  \label{thm:equivalence}

  $L_1 \leq L_2$ if{f} $H(L_1) \subseteq H(L_2)$.

\end{theorem}

\begin{proof}
  
  ($\Rightarrow$) Let $h = \tup{O,<,f} \in H(L_1)$; we show $h \in H(L_2)$ by
  constructing a program $P_h$ over actions $@S$ which only admits executions
  with histories stronger than $h$:
  \begin{align*}
    \forall e \in E(P_h).\ |(e|@S)| = n \implies h \preceq H(e) \text{,}
  \end{align*}
  where $n = |\set{ o \in O : f(o) = m(u)=>v\neq \bot}|$ is the number of
  completed operations in $h$. Given such a program $P_h$, taking any execution
  $e_1 \in E(P_h \x L_1)$ with $|(e_1|@S)| = n$, we must also have an execution
  $e_2 \in E(P_h \x L_2)$ such that $(e_2|@S) = (e_1|@S)$ by definition of $L_1
  <= L_2$. Since $|(e_2|@S)| = n$ and $e_2 \in E(P_h)$, we also know that $h
  \preceq H(e_2)$, and since $(e_2 | C \u R) \in E(L_2)$, we have $H(e_2) \in
  H(L_2)$, along with any history weaker than $H(e_2)$ by
  Lemma~\ref{lem:lib:closed}, namely $h$.

  We construct $P_h = \tup{Q,@S,q_0,@d}$ over alphabet $@S = C \u R \u \set{a}$
  whose states $Q : O -> \<Bools>^2$ track operations called/completed status.
  The initial state is $q_0 = \set{ o |-> \tup{\bot,\bot} : o \in O}$.
  Transitions are given by,
  \begin{align*}
    & \text{for each } q \in Q, o \in O, m \in \<Methods>, v \in \<Vals> \\
    & \quad \text{if } f(o) = m(v) => \_
      \text{ and } q(o') \text{ for all } o' < o \text{ then} \\
    & \qquad q[o |-> \bot,\bot] \xrightarrow{m(v)_o} q[o |-> \top,\bot] \\
    & \quad \text{ if } f(o) = m(\_) => v \text{ then} \\
    & \qquad q[o |-> \top,\bot] \xrightarrow{\<ret>(v)_o}
      \cdot \xrightarrow{a} q[o |-> \top,\top] \\
    & \quad \text{ if } f(o) = m(\_) => \bot \text{ then} \\
    & \qquad q[o |-> \top,\bot] \xrightarrow{\<ret>(v)_o} q[o |-> \top,\top]
  \end{align*}
  It is routine to verify that $P_h$ is a program according to
  Definition~\ref{def:programs}.
  Moreover, in any execution $e \in E(P_h)$, the call $m(u)_o$ of any operation $o$ must
  come after the return $\<ret>(v)_{o'}$ of each $o'<o$. Furthermore, all
  completed operations of $h$ are completed in $e$ if and only if $(e|@S) =
  a^n$. Note that $e$ may contain more completed operations than $h$ (because
  of the last transition rule) but less pending operations (because the transition
  rule corresponding to the call of a pending operation may not have been applied in $e$). 
  It follows that $|(e|@S)| = n \implies h \preceq H(e)$.

  ($\Leftarrow$)
  Let $P$ be a program over actions $@S$, and $e \in E(P \x L_1)$; we show that
  $e \in E(P \x L_2)$. Since $(e | C \cup R) \in E(L_1)$, we know $H(e) \in
  H(L_1)$ by definition of $H(L_1)$, and then $H(e) \in H(L_2)$ by hypothesis.
  By Lemma~\ref{lem:lib:execs} we deduce $(e | C \cup R) \in E(L_2)$, and
  thus by definition of LTS composition, $e \in E(P \x L_2)$.
\end{proof}
