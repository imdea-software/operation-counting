%!TEX root = draft.tex
\section{Comparison with Linearizability}
\label{sec:lin}

The linearizability criterion~\cite{journals/toplas/HerlihyW90} provides an
alternative characterization of library conformance which implies observational
refinement~\cite{journals/tcs/FilipovicORY10}. In this section we demonstrate
that linearizability is generally stricter than observational refinement, yet
the two criteria coincide when the reference implementation is atomic.

% \paragraph{Observational refinement and linearizability}

Linearizability~\cite{journals/toplas/HerlihyW90} is defined by an execution
order: $e_1 \sqsubseteq e_2$ if{f} there exists a well-formed execution $e_1'$
obtained from $e_1$ by appending return actions, and deleting call actions,
such that:
\begin{quote}

  $e_2$ is a permutation of $e_1'$ that preserves the order between
  return and call actions, i.e.,~a given return action occurs before a given
  call action in $e_1'$ if{f} the same holds in $e_2$.

\end{quote}
For example, the second and the third executions in Example~\ref{ex:libraries}
are weaker than the first (with sequential calls to $\<push>$ and $\<pop>$). An
execution $e_1$ is \emph{linearizable} w.r.t.~a library $L_2$\footnote{The
original definition of \citet{journals/toplas/HerlihyW90} assumes that $L_2$ is
a set of sequential executions. Here we consider a slight extension adapted to
concurrent executions, faithful to the original intention, where $\ker L_2$ may
be a set of sequential executions.} if{f} there exists a sequential execution
$e_2 \in E(L_2)$, with only
completed operations, such that $e_1 \sqsubseteq e_2$. A library $L_1$
is \emph{linearizable} w.r.t. $L_2$, written $L_1 \sqsubseteq L_2$, if{f}
each execution $e_1 \in E(L_1)$ is linearizable w.r.t. $L_2$.

Since linearizability compares executions of $L_1$, which may contain pending
operations, with executions of $L_2$, in which every operation is completed,
observational refinement need not imply linearizability when $L_2$ contains
non-terminating methods, i.e.,~where the calls to these methods are pending in
all executions.

\begin{example}
  \label{ex:notlin}

  Let $L$ be the library whose kernel contains the single execution $e =
  m(u)_1\ m'(u)_2\ \<ret>(v)_1$, in which the call to $m'$ is pending.
  Although $L$ refines itself, since refinement is reflexive, $L$ is not
  linearizable w.r.t. itself, since $e$ could only be linearizable w.r.t. $L$
  if $E(L)$ were to contain one of the following executions:
  \begin{mathpar}
    m(u)_1\ \<ret>(v)_1 \quad
    m(u)_1\ m'(u)_2\ \<ret>(v)_1\ \<ret>(\_)_2 \\
    m(u)_1\ \<ret>(v)_1\ m'(u)_2\ \<ret>(\_)_2 \quad
    m'(u)_2\ \<ret>(\_)_2m(u)_1\ \<ret>(v)_1 \text{.}
  \end{mathpar}
  Yet $E(L) = \overline{\set{e}}$ clearly contains none of them.
  
\end{example}

% \paragraph{Linearizability w.r.t. atomic libraries}

A typical correctness criterion for concurrent objects is linearizability with
respect to atomic versions of themselves. Despite the negative general result
of Example~\ref{ex:notlin}, when restricted to atomic libraries,
linearizability and history inclusion, and thus observational refinement,
coincide. This relationship essentially follows from the relationship between
the order relations $\sqsubseteq$ and $\preceq$.

\begin{lemma}
  \label{lemma:exec_hist}

  $e_1\sqsubseteq e_2$ if{f} $H(e_1) \preceq H(e_2)$, if $e_2$ is sequential.

\end{lemma}

\begin{proof}
  By definition $e_1 \sqsubseteq e_2$ means that there exists $e_1'$ such that
  $e_2$ is a permutation of $e_1'$ preserving the order between return and call
  actions, thus $e_2 ~> e_1'$, and thus $H(e_1') \preceq H(e_2)$. Furthermore,
  since $e_1'$ is obtained from $e_1$ by appending returns and deleting calls,
  $H(e_1) \preceq H(e_1')$. By transitivity, $H(e_1) \preceq H(e_2)$. The other
  direction is equally straightforward.
\end{proof}

\begin{theorem}

  $L_1 \sqsubseteq L_2$ if{f} $H(L_1) \subseteq H(L_2)$,
  if $L_2$ is atomic.

\end{theorem}

\begin{proof}

  ($\Rightarrow$)
  Let $h\in H(L_1)$. By hypothesis, any execution $e_1$ with $H(e_1)=h$ is
  linearizable w.r.t. $L_2$, i.e., there exists an execution $e_2\in L_2$ with
  only completed operations such that $e_1\sqsubseteq e_2$. By
  Lemma~\ref{lemma:exec_hist}, this implies $H(e_1)\preceq H(e_2)$. By the
  closure property in Lemma~\ref{lem:lib:closed}, if $H(e_2)\in H(L_2)$ then
  $h=H(e_1)\in H(L_2)$.

  ($\Leftarrow$)
  Let $e_1$ be an execution of $L_1$. By hypothesis, $H(e_1)\in H(L_2)$, which
  by Lemma~\ref{lem:lib:execs}, implies $e_1\in E(L_2)$. Since $L_2$ is an
  atomic library, there exists a sequential execution $e_2\in \ker E(L_2)$ with
  only completed operations such that $e_1\sqsubseteq e_2$. Thus, $e_1$ is
  linearizable w.r.t. $L_2$.
\end{proof}
