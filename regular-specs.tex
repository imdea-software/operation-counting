%!TEX root = draft.tex
\section{Atomic Libraries with Context-Free Kernels}
\label{sec:regular}

While the previous section provides operation-counting formulae for particular
classes of atomic objects, here we provide a systematic technique to derive
operation-counting formulae for any atomic library that can be written as a
context-free language, including objects such as (reader-writer) locks and
semaphores, or context-free approximations of arbitrary libraries. For this
construction we require finite sets $\<Methods>$ and $\<Vals>$ of methods and
values. Given an atomic library $L$ such that $\ker E(L)$ is context free, and
an interval bound $k \in \<Nats>$, we generate a formula $@Y_{L,k}$
representing $L$ up to $k$.

Our construction relies on Parikh's Theorem~\cite{journals/jacm/Parikh66}. We
recall that the \emph{Parikh image} of $w \in @S^*$ is the multiset $\Pi(w) :
@S -> \<Nats>$ mapping each symbol $a \in @S$ to its number of occurrences in
$w$, and the \emph{Parikh image} of a language $E \subseteq @S^*$ is the set
$\Pi(E) = \set{ \Pi(w) : w \in L }$ of its words' images. We show that if $\ker
E(L)$ is context free, then the language $E_k = \set{ e \in E(L) : \len H(e)
\le k}$ of $L$'s executions with bounded-length histories has the same Parikh
image as a context-free language, and by Parikh's Theorem, $\Pi(E_k)$ can be
represented as a Presburger formula, from which we derive $@Y_{L,k}$ (the proof
can be found in the accompanying technical report, \S\ref{app:proofs}).

\begin{theorem}
  \label{thm:formula}

  Let $L$ be an atomic library over finite sets $\<Methods>$ and $\<Vals>$ of
  methods and values such that $\ker E(L)$ is a context-free language, and let
  $k \in \<Nats>$. Then there exists an effectively-computable
  operation-counting formula $@Y_{L,k}$ representing $L$ up to $k$.

\end{theorem}

This construction is useful in practice; with it we have derived the formulae
used in our static verification experiments of Section~\ref{sec:exp:static}.

% In the worst case, the size of the formula $@Y_{L,k}$ is exponential in the
% size of
