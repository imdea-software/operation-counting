%!TEX root = draft.tex
\section{Atomic Libraries with Context-Free Kernels}
\label{sec:regular}

We prove that for every atomic library $L$ over a fixed-size set of methods $\<Methods>$ and values $\<Vals>$, 
 there exists an effectively computable
$\ocl$ formula $@Y_L[k]$ representing $L$ up to $k$, for every $k$, provided that the kernel of $L$ is a context-free 
language. This result offers a systematic method for constructing formulas that represent bounded length
histories of atomic libraries implementing locks, read-write locks, semaphores, etc., or approximations
of arbitrary libraries.

More precisely, we show that there exists a context-free regular language $\Gamma_k$, whose Parikh image is exactly 
the set of counting representations of histories in $L$ of length at most $k$. We recall that the \emph{Parikh image} of a 
sequence $@s \in @S$ is the multiset $\Pi(@s) : @S -> \<Nats>$ mapping each symbol $a \in @S$ to the 
number of occurrences of $a$ in $@s$. The \emph{Parikh image} of a language $L \subseteq @S$
are the images $\Pi(L) = \set{\Pi(@s) : @s \in L}$ of sequences in $L$.
The language $\Gamma_k$ is context-free because it is the intersection between a regular language and the inverse of 
an homomorphism from $\Gamma_k$ to some context-free language $\Lambda$, obtained from $\ker E(L)$.
%(here, we use the closure of context-free languages under 
%inverse homomorphisms and intersection with regular languages). 
Then, $@Y_L[k]$ is defined as the quantifier-free Presburger
formula representing the Parikh image of $\Gamma_k$ (which exists by Parikh theorem~\cite{journals/jacm/Parikh66}).

\begin{theorem}

Let $L$ be an atomic library over a set of methods $\<Methods>$ of fixed size such that  
$\ker E(L)$ is a context-free language. Then, for every $k$, there exists an effectively computable 
$\ocl$ formula $@Y_L[k]$ representing $L$ up to $k$.

\end{theorem}

\begin{proof}

Let $@L$ be the context-free language obtained from $\ker E(L)$ by replacing every pair of consecutive actions
$m(u)_o\ \<ret>(v)_o$ by $m(u)=>v$. Note that every sequence $\sigma$ in $@L$
represents a history $h=(O,<,f)\in H(\ker E(L))$ in the sense that the multiset of symbols in $\sigma$ is exactly the
multiset $\mset{ f(o): o\in O}$ of operation labels in $h$ and every two operation labels in $\sigma$ occur
in the order defined by $<$, i.e., for every two operations $o_1,o_2\in O$, if $o_1<o_2$, then $f(o_1)$ 
occurs before $f(o_2)$ in $\sigma$.

Let $h=(O,<,f)\in H(L)$ be a history of length at most $k$ and $I:O -> [k]^2$ its canonical representation. 
Since there exists a sequential history $h'=(O',<',f')\in H(\ker E(L))$ (where the operations are totally ordered) 
such that $h\preceq h'$, $h$ can be written as a sequence
of symbols \tup{f(o),I(o),b} that is consistent with the order between operations defined by $<'$. The boolean $b$
is $\<true>$ iff the operation $o$ belongs to the sequential history $h'$. Formally,
let $\<Lab>$ be the following alphabet:
\[
\<Lab>=\set{\tup{@l,i,j,b}: @l  \in \<Labels>, 0\leq i\leq j\leq k, b\in\<Bools>}.
\]

Let $\lab(h):O -> \<Lab>$ be a mapping associating to each operation $o$ a symbol in $\<Lab>$:
\[
\lab(h)(o)=\left\{\begin{array}{ll}\tup{f(o),I(o),\<true>},\mbox{ if $o\in O'$}\\
						\tup{f(o),I(o),\<false>},\mbox{ otherwise.}
			\end{array}
		\right.
\]

Then, let $\alpha(h)$ be a sequence over $\<Lab>$, that contains exactly the same 
multiset of symbols as the range of $\lab(h)$ and for every $o_1<' o_2$, $\lab(h)(o_1)$ 
occurs before $\lab(h)(o_2)$ in $\alpha(h)$.

%\begin{itemize}
%	\item the multiset of symbols in $\alpha(h)$ is exactly the multiset of symbols in the range of $\lab(h)$, and 
%	\item for every $o_1<' o_2$, $\lab(h)(o_1)$ occurs before $\lab(h)(o_2)$ in $\alpha(h)$.
%\end{itemize}

Since $h\preceq h'$, whenever an operation $o_1$ is ordered before another operation $o_2$ in 
the sequential history $h'$, the interval $I(o_1)$ is either before or it is overlapping with $I(o_2)$.
Therefore, the order between intervals in the sequence $\alpha(h)$ is not arbitrary. More precisely, 
if a symbol $\tup{@l_1,i_1,j_1,b_1}$ occurs before another symbol $\tup{@l_2,i_2,j_2,b_2}$, 
then $i_1\leq j_2$. A sequence $\sigma\in\<Lab>^*$ satisfying this property is called 
\emph{interval-consistent}. 

Now, let $\mu:\<Lab> -> \<Methods>^*$ be an homomorphism defined by
$\mu(\tup{@l,i,j,\<true>})=@l$ and $\mu(\tup{@l,i,j,\<false>})=\epsilon$,
%\[
%\mu(\tup{@l,i,j,b})=\left\{\begin{array}{ll} @l,\mbox{ if $b=\<true>$}\\
%						\epsilon,\mbox{ otherwise}
%			\end{array}
%		\right.
%\]
where $\epsilon$ is the empty sequence.

It follows easily from definitions that $\mu(\alpha(h))\in @L$, for every history $h$. On the other hand, $\mu^{-1}(@L)$ may contain
sequences which do not correspond to valid histories of $L$ because they are not interval-consistent. However,
%if $_k$ is the set of all sequences $\alpha(h)$ with $h$ a history of $\atLib{S}$ of length at most $k$, i.e.,
%$S_k=\set{\alpha(h):h\in \atLib{S}\mbox{ of length $k$}}$, then 
one can prove that
\[
\Gamma_k=\set{\alpha(h):h\in H(L)\mbox{ of length $k$}}=\mu^{-1}(@L)\cap \Delta_k,
\]
where $\Delta_k$ is the language of all interval-consistent sequences with interval bounds at most $k$. 
Since the latter is regular and the context-free languages
are closed under intersection with regular languages and inverse homomorphisms, the language $\Gamma_k$ is context-free.

Hence, by Parikh theorem~\cite{journals/jacm/Parikh66}, there exists a quantifier-free Presburger formula $@Y[k]$ representing the Parikh image of $\Gamma_k$.
W.l.o.g. we assume that the free variables of $@Y[k]$ are $\# a$ with $a\in\<Lab>$ ($\# a$ is the number of 
occurrences of $a$ in a sequence of $\Gamma_k$).

We define $@Y_L[k]$ as the quantifier-free Presburger formula obtained from 
\[
@Y[k]\ \land \hspace{-1mm}\bigwedge_{(@l,i,j)}\#(@l,i,j)=\# \tup{@l,i,j,\<true>} + \# \tup{@l,i,j,\<false>}
\]
by projecting out all variables of $@Y[k]$.
\end{proof}