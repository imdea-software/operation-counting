%!TEX root = draft.tex
\section{Atomic Libraries with Context-Free Kernels}
\label{sec:regular}

While the previous section provides operation-counting formulae for particular
classes of atomic objects, here we provide a systematic technique to derive
operation-counting formulae for any atomic library that can be written as a
context-free language, including objects such as (reader-writer) locks and
semaphores, or context-free approximations of arbitrary libraries. For this
construction we require finite sets $\<Methods>$ and $\<Vals>$ of methods and
values. Given an atomic library $L$ such that $\ker E(L)$ is context free, and
an interval bound $k \in \<Nats>$, we generate a formuala $@Y_{L,k}$
representing $L$ up to $k$.

Our construction relies on Parikh's Theorem~\cite{journals/jacm/Parikh66}. We
recall that the \emph{Parikh image} of $w \in @S^*$ is the multiset $\Pi(w) :
@S -> \<Nats>$ mapping each symbol $a \in @S$ to its number of occurrences in
$w$, and the \emph{Parikh image} of a language $M \subseteq @S^*$ is the set
$\Pi(M) = \set{ \Pi(w) : w \in L }$ of its words' images. We show that if $\ker
E(L)$ is context free, then the language $E_k = \set{ e \in E(L) : \len H(e)
\le k}$ of $L$'s executions with bounded-length histories has the same Parikh image 
as a context-free language, and by Parkikh's Theorem, $\Pi(E_k)$ can be
represented as a Presburger formula, from which we derive $@Y_{L,k}$.

\begin{theorem}
  \label{thm:formula}

  Let $L$ be an atomic library over finite sets $\<Methods>$ and $\<Vals>$ of
  methods and values such that $\ker E(L)$ is a context-free language, and let
  $k \in \<Nats>$. Then there exists an effectively-computable
  operation-counting formula $@Y_{L,k}$ representing $L$ up to $k$.

\end{theorem}

\begin{proof}

Let $\Delta$ be the context-free language obtained from $\ker E(L)$ by replacing
every pair of consecutive actions $m(u)_o\ \<ret>(v)_o$ by $m(u)=>v$. Note that
every sequence $\sigma$ in $\Delta$ represents a history $h=(O,<,f)\in H(\ker
E(L))$ in the sense that the multiset of symbols in $\sigma$ is exactly the
multiset $\mset{ f(o): o\in O}$ of operation labels in $h$ and every two
operation labels in $\sigma$ occur in the order defined by $<$, i.e., for every
two operations $o_1,o_2\in O$, if $o_1<o_2$, then $f(o_1)$ occurs before
$f(o_2)$ in $\sigma$.

Let $h=(O,<,f)\in H(L)$ be a history of length at most $k$ and $I:O -> [k]^2$
its canonical representation. Since there exists a sequential history
$h'=(O',<',f')\in H(\ker E(L))$ (where the operations are totally ordered) such
that $h\preceq h'$, $h$ can be written as a sequence of symbols
\tup{f(o),I(o),b} that is consistent with the order between operations defined
by $<'$. The boolean $b$ is $\<true>$ iff the operation $o$ belongs to the
sequential history $h'$. Formally, let $\<Labels>_{\text I}$ be the following alphabet:
\begin{align*}
  \<Labels>_{\text I} = \set{\tup{@l,i,j,b}: @l  \in \<Labels>, 0\leq i\leq j\leq k, b\in\<Bools>}.
\end{align*}
Let $\lab(h):O -> \<Labels>_{\text I}$ be a mapping associating to each operation $o$ a
symbol in $\<Labels>_{\text I}$:
\begin{align*}
  \lab(h)(o)=\left\{\begin{array}{ll}\tup{f(o),I(o),\<true>},\mbox{ if $o\in O'$}\\
  						\tup{f(o),I(o),\<false>},\mbox{ otherwise.}
  			\end{array}
  		\right.
\end{align*}
Then, let $\<lin>(h)$ be a sequence over $\<Labels>_{\text I}$, that contains exactly the
same multiset of symbols as the range of $\lab(h)$ and for every $o_1<' o_2$,
$\lab(h)(o_1)$ occurs before $\lab(h)(o_2)$ in $\<lin>(h)$.

Since $h\preceq h'$, whenever an operation $o_1$ is ordered before another
operation $o_2$ in the sequential history $h'$, the interval $I(o_1)$ is either
before or it is overlapping with $I(o_2)$. Therefore, the order between
intervals in the sequence $\<lin>(h)$ is not arbitrary. More precisely, if a
symbol $\tup{@l_1,i_1,j_1,b_1}$ occurs before another symbol
$\tup{@l_2,i_2,j_2,b_2}$, then $i_1\leq j_2$. A sequence $\sigma\in\<Labels>_{\text I}^*$
satisfying this property is called \emph{interval-consistent}.

Now, let $\alpha : \<Labels>_{\text I} -> \<Methods>^*$ be an homomorphism defined by
$\alpha(\tup{@l,i,j,\<true>}) = @l$ and $\alpha(\tup{@l,i,j,\<false>}) = @e$, where
$@e$ denotes the empty sequence.

It follows easily from definitions that $\alpha(\<lin>(h))\in \Delta$, for every
history $h$. On the other hand, $\alpha^{-1}(\Delta)$ may contain sequences which do
not correspond to valid histories of $L$ because they are not
interval-consistent. However, one can prove that
\begin{align*}
  \Theta_k=\set{\<lin>(h):h\in H(L)\mbox{ and $\len h \le k$}}=\alpha^{-1}(\Delta)\cap \Upsilon_k,
\end{align*}
where $\Upsilon_k$ is the language of all interval-consistent sequences with
interval bounds at most $k$. Since the latter is regular and the context-free
languages are closed under intersection with regular languages and inverse
homomorphisms, the language $\Theta_k$ is context-free. Note that $\Theta_k$
has the same Parikh image as $E_k = \set{ e \in E(L) : \len H(e) \le k}$.

Hence, by Parikh's theorem~\cite{journals/jacm/Parikh66}, there exists a
quantifier-free Presburger formula $@Y$ representing the Parikh image of
$\Theta_k$. W.l.o.g. we assume that the free variables of $@Y$ are $\set{ \#a :
a \in \<Labels>_{\text I} }$, where $\#a$ represents the number of occurrences of $a$ in a
sequence of $\Theta_k$.

We define $@Y_{L,k}$ as the quantifier-free Presburger formula
\begin{align*}
  @Y \land
  \bigwedge_{(@l,i,j)} \#(@l,i,j) = \# \tup{@l,i,j,\<true>} + \# \tup{@l,i,j,\<false>}
\end{align*}
which projects away the variables of $@Y$.
\end{proof}

This construction is useful in practice; with it we have derived formulae used
in our static verification experiments of Section~\ref{sec:exp:static}.
