%!TEX root = draft.tex
\section{Experimental Evidence}
\label{sec:exp}

As a proof of concept we have implemented our techniques to analyze algorithms
appearing in the research literature; our prototype, composed entirely of
freely-available components, discovers bugs either injected into these
algorithms, or due to ambiguities in their pseudocode listings. Given library
code written in the Boogie language---either automatically translated from the
source C/C++ code, or manually translated---we execute the following workflow:
\vspace{-1eX}
\begin{enumerate}
  \item Add client and instrumentation to library code, given a bound on
    the number of client shared-memory writes, resulting in a closed concurrent 
    program, annotated with assertions.
  \item Perform delay-bounded sequentialization~\citep{ conf/popl/EmmiQR11},
    for a given delay bound, translating the concurrent library and client 
    code to sequential code; in many cases, we manually limit the number of 
    program points where threads can yield control.
  \item Search for assertion violations, using existing sequential program
    analysis engines; in our case, we use Corral~\citep{ conf/cav/LalQL12},
    typically bounding recursion and loop-unrolling.
    \vspace{-1eX}
\end{enumerate}
The Clang-to-Boogie and concurrent-to-sequential translations are available in
the {\tt SMACK}\footnote{The {\tt SMACK} Project :
\url{http://github.com/smackers/smack}} and {\tt c2s}\footnote{The {\tt c2s}
Project : \url{http://github.com/michael-emmi/c2s}} projects; Corral's
sequential program analysis is available in Boogie\footnote{Boogie :
\url{http://boogie.codeplex.com}} using the {\tt\small /stratifiedInline:1}
switch.

Figure~\ref{fig:exps:lin} lists the results of running our methodology on
several textbook and research algorithms implementing concurrent data
structures %and transactional memories 
in which we have injected subtle
programming errors. Generally speaking, we discover bugs with low-ranking
clients (normally $2$--$3$), few shared-memory writes ($0$--$1$), and few
delays ($1$--$4$). In the case of the mod-2 incrementer example, the injected
bug involves a ``lost update,'' where two concurrent increments write the same
counter value. Without shared-memory writes, all counter valuations are
included in the specification, since the concurrent invocations to the ZERO
method can be ordered to occur after an even number of INC methods; to manifest
a violation, an odd number of INC methods must be separated by a shared-memory
write from a subsequent ZERO method.

\begin{figure}[t]
  \centering
  \scriptsize
  \begin{tabular}{|l|l|ccc|l|}
    \hline
    Example / Bug & Client & $R$ & $W$ & $D$ & Time \\
    \hline
    2-Lock Queue / DEQ 
    & 1xENQ, 2xDEQ
    & 2 & 0 & 1 & 2.80s \\
    Lock-cpl. Set / REM
    & 1xADD, 2xREM
    & 2 & 0 & 1 & 3.47s \\
    Mod-2 Incr. / INC
    & 3xINC ; 1xZERO
    & 3 & 1 & 4 & 10.27s \\
    Treiber Stack / POP 
    & 1xPUSH, 2xPOP
    & 2 & 0 & 1 & 2.18s \\
    Treiber Stack / PUSH
    & 2xPUSH, 2xPOP
    & 3 & 0 & 2 & 3.15s \\ % 1xPUSH(v1), 1xPUSH(v2), 2xPOP(v2)
    Treiber Stack / ``ABA''
    & 3xPUSH, 4xPOP
    & 6 & 0 & 3 & 426.82s \\
    \hline
    \hline
    Elim. Stack / ``empty''
    & 3xPUSH, 1xPOP
    & 3 & 1 & 4 & 441s \\
    Elim. Stack / ``brace''
    & 4xPUSH, 2xPOP
    & 5 & 0 & 8 & 1496 s \\
    \hline
  \end{tabular}
  \caption{Discovering observational refinement violations in textbook algorithms; 
    the bug-injected method is noted, with ``ABA'' denoting the
    infamous ABA bug.  In client listings, the semicolon separates method
    intervals. $R$ denotes client rank, $W$ the number of shared-memory writes,
    and $D$ delays.}
  \label{fig:exps:lin}
  \vspace{-3eX}
\end{figure}

\vspace{1mm}
\noindent
{\bf The Elimination Stack.}
Hendler et al.'s Elimination Stack~\cite{conf/spaa/HendlerSY04} augments
Treiber's stack with a ``collision array,'' used in case an optimistic push or
pop detects a conflicting operation; the collision array pairs together
concurrent push and pop operations to ``eliminate'' them without affecting the
underlying data structure. In writing the Elimination Stack pseudocode in
Boogie, we have inadvertently introduced (at least!) two bugs, both leading to
observational refinement violations.

The first bug is due to the modeling of a {\tt cell}~field, which stores either
the value returned by a pop operation, or the special value {\tt Empty}, in the
case of empty stack. Due to the limited expressivity of typing in Boogie, we
have modeled the {\tt cell}~field with two fields: a value-typed {\tt
cell}~field, and a Boolean {\tt empty} field, to be set to true only when {\tt
cell} is to be set to {\tt Empty}. While we had been careful to always
initialize {\tt empty} to false, and to set {\tt empty} to true whenever the
{\tt cell}~field is assigned {\tt Empty}, we had not set {\tt empty} to false
each time the {\tt cell}~field is assigned a non-{\tt Empty} value. This
mistake surfaces as a violation in executions where: (1)~an
initial push operation completes before two more pushes, concurrent with a pop
operation, begin; (2)~the pop operation, detecting a conflicting push, sets its
{\tt cell} field to {\tt Empty}, and is put on the collision array; (3)~the
third push operation pairs with the pop on the collision array, and sets its
{\tt cell} field to the pushed value, and returns. As our Boogie-coded
version would set pop's {\tt empty} to true before entering the collision
array, and never subsequently set {\tt empty} to false, the pop operation is
observed to return empty; an execution with a shared write 
after the first push operation violates the stack specification because it imposes that 
the pop starts after this push operation. We have
detected this violation in $441$ seconds, in a single-{\shwrite} $4$-delay
execution, with the rank $3$ client of the $4$ operations described above.

The second bug is due to a typographical error in the Elimination Stack
pseudocode: a missing closing brace. The indentation suggests three possible
positions; while one leads to a non-termination bug (beyond the scope of this
work), another leads to a bug where a pop operation prematurely returns without
having read a valid pushed value---thus returning an uninitialized value. This
bug is induced by a rank $5$ client with $4$ push operations concurrent with
$2$ pops, in which: (1)~the first push operation succeeds leaving the stack
non-empty; (2)~the second push operation sends a third push, and both pop
operations, to the collision array; (3)~the third push pairs with one pop, both
return; (4)~the remaining pop leaves the collision array, trying again to
perform a stack operation; (5)~the fourth push operation intervenes, making the
pop detect another conflict; (6)~because of the misplaced brace, the pop
operation prematurely returns, rather than repeating the entire process again.
This violation surfaces in a zero-{\shwrite} execution with $8$ delays. % in $1496$ seconds.

\vspace{1mm}
\noindent
{\bf Comparison with testing-based tools.} Comparing our running times with those reported for 
testing-based tools that detect violations to linearizability, we have found that they are in the same 
order of magnitude although our prototype is based on a symbolic analysis and it explores a much larger 
set of program behaviors (without having to fix the methods input values).
For instance, Line-Up takes in average around 40 minutes for checking the linearizability of one 
concrete library client with 9 method invocations (in average, libraries have 900 loc, around 2 times bigger 
than our implementation of Elimination Stack). Our tool takes at most 24 minutes to check 
operation counting refinement for a client (where the method inputs are not fixed) with 6 method invocations.

%randomly samples concrete library clients 
%(finite collections of method invocations with concrete inputs), 
%with at most 3 threads, where each thread executes at most 3 methods. 
%Line-Up takes in average around 40 minutes for checking the linearizability of one concrete library client with 9 method invocations (in average, libraries have 900 loc, around 2 times bigger than our implementation of Elimination Stack). Our tool takes at most 24 minutes to check (bounded-barrier) linearizability of a client (where the method inputs are not fixed) with 6 method invocations. Consequently, checking linearizability of a concrete client in Line-Up is of the same order of magnitude as checking the linearizability of a client in our prototype. But, differently from Line-Up, our prototype finds the input values needed in order to exposed a bug.

