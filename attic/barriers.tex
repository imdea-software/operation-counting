\section{Bounded-Barrier Linearizability}

% \begin{theorem}[Bounded-Barrier Linearizability Characterization]
% Let $\osys_\concr$ be a quiescent object system and $\osys_\abs$ a 
% thread-isomorphic quiescent object system. The following statements are 
% equivalent.
% \begin{enumerate}
% % \item \label{bbl}
% %   $ \forall \hist_\concr \in \maxbar{\osys_\concr}{k}.\ 
% %       \exists \hist_\abs \in \osys_\abs.\ \hist_\concr \lin \hist_\abs$
% % \item \label{bbl-tid}
% %   $ \forall \hist_\concr \in \maxbar{\osys_\concr}{k}.\ 
% %       \exists \hist_\abs \in \osys_\abs.\ \hist_\concr \plin \hist_\abs$
% % \item \label{ebbl}
% %   $ \forall \hist_\concr \in \osys_\concr.\ 
% %       \forall \barriers \subseteq \set{1,\dots,\length{\hist}}.\ 
% %         |\barriers| \leq k \implies
% %           \exists \hist_\abs \in \osys_\abs.\ 
% %             \hist_\concr \linchoice{\barriers} \hist_\abs$
% \item \label{ebbl-tid}
%   $ \forall \hist_\concr \in \osys_\concr.\ 
%       \forall \barriers \subseteq \set{1,\dots,\length{\hist_\concr}}.\ 
%         |\barriers| \leq k \implies
%           \exists \hist_\abs \in \osys_\abs.\ 
%             \hist_\concr \plinchoice{\barriers} \hist_\abs$
% % \item \label{or-one}
% %   $ \osys_\concr$ observationally refines $\osys_\abs$ for clients with 
% %   at most one method per thread, and at most $k$ global writes
% \item \label{or-many}
%   $ \osys_\concr$ observationally refines $\osys_\abs$ for clients with 
%   no local reads, except possibly at the end of each thread, and at most $k$
%   global writes 
% \end{enumerate}
% 
% 
% \end{theorem}


\begin{theorem}[Bounded-Barrier Linearizability Characterization]
Let $\osys_\concr$ be a quiescent object system and $\osys_\abs$ a 
thread-isomorphic quiescent object system. For $k \in \mathbb{N}$, the 
following statements are equivalent:
\begin{enumerate}
% \item \label{bbl}
%   $ \forall \hist_\concr \in \maxbar{\osys_\concr}{k}.\ 
%       \exists \hist_\abs \in \osys_\abs.\ \hist_\concr \lin \hist_\abs$
% \item \label{bbl-tid}
%   $ \forall \hist_\concr \in \maxbar{\osys_\concr}{k}.\ 
%       \exists \hist_\abs \in \osys_\abs.\ \hist_\concr \plin \hist_\abs$
\item \label{ebbl}
  $ \forall \hist_\concr \in \osys_\concr.\ 
      \forall \barriers \subseteq \set{1,\dots,\length{\hist}}.\ 
        |\barriers| \leq k \implies
          \exists \hist_\abs \in \osys_\abs.\ 
            \hist_\concr \linchoice{\barriers} \hist_\abs$
% \item \label{ebbl-tid}
%   $ \forall \hist_\concr \in \osys_\concr.\ 
%       \forall \barriers \subseteq \set{1,\dots,\length{\hist_\concr}}.\ 
%         |\barriers| \leq k \implies
%           \exists \hist_\abs \in \osys_\abs.\ 
%             \hist_\concr \plinchoice{\barriers} \hist_\abs$
\item \label{or-one}
  $ \osys_\concr$ observationally refines $\osys_\abs$ for clients with 
  at most one method per thread, and at most $k$ global writes
% \item \label{or-many}
%   $ \osys_\concr$ observationally refines $\osys_\abs$ for clients with 
%   no local reads, except at the end of each thread, and at most $k$ global 
%   writes 
\end{enumerate}
\end{theorem}

\begin{proof}
\newcommand{\npairs}{{p}}
\newcommand{\nbars}{k'}
\newcommand{\ticker}{b}

$(\ref{or-one} \Rightarrow \ref{ebbl})$ Let 
$\hist_\concr \in \osys_\concr$, 
$\barriers \subseteq \set{1,\dots,\length{\hist_\concr}}$ with 
$|\barriers| \leq k$. Let $\barrier_1,\dots,\barrier_k$ be the values of 
$\barriers$ sorted in increasing order. We define a program 
$\prog_{\hist_\concr,\barriers}$ with one method per thread and at most $k$ 
global writes. Let $(c_1,r_1),\dots,\ab (c_\npairs,r_\npairs)$ the pairs 
of matching call and return actions from $\hist_\concr$. 

Let $j_{c_i}$ (\resp $j_{r_i}$) be the largest integer such that 
$\barrier_{j_{c_i}}$ is smaller or equal than the index of $c_i$ 
(\resp $r_i$) in $\hist_\concr$, if it exists, or $0$ otherwise.
Let $\callAct{\tid}{\obj_i}{\meth_i}{\cval_i} = c_i$ and 
$\retAct{\tid}{\obj_i}{\meth_i}{\rval_i} = r_i$.

Let $\commands_i$ be the following sequence of commands:
\begin{flalign*}
&x_i \colonequals \cval_i;\text{  \comment{// local write}}&\\
&\assumeCom{\ticker = \barrier_{c_i}};\text{  \comment{// global read}} 
&\\
&y_i \colonequals \callCom{\obj_i}{\meth_i}{x_i};
  \text{  \comment{// local write}}&\\
&\assumeCom{\ticker = \barrier_{r_i}};\text{  \comment{// global read}}&\\
&\assumeCom{y_i = \rval_i};\text{  \comment{// local read}}&
\end{flalign*}

The program $\prog_{\hist_\concr,\barriers}$ then executes the commands
$\commands_1,\dots,\commands_\npairs$ in a parallel composition.
We also add a new thread in $\prog_{\hist_\concr,\barriers}$ 
which has $k$ increments of the global variable $\ticker$.
In $\prog_{\hist_\concr,\barriers}$, we have at most one method call per 
thread, as well as (at most) $k$ global writes.

Let $n_0$ be a value which doesn't appear in $\hist_\concr$ and $\state_0$ be 
an initial state where all the local variables are initialized to $n_0$ and 
the global variable $\ticker$ is initialized to $0$.

There is a trace $\trace_\concr$ in $\semantic{\prog_{\hist_\concr,\barriers}}$
whose history is $\hist_\concr$ and which reaches a state $\state_f$ where 
$x_i = \cval_i$ and $y_i = \rval_i$ for all $\tid$ and 
$i \in \set{1,\dots,\npairs}$. Since we have $\state_f \in 
\comp{\prog_{\hist_\concr,\barriers}}{\osys_\concr}(\state_0)$ and 
$\osys_\concr$ observationally refines $\osys_\abs$, we also have
$\state_f \in \comp{\prog_{\hist_\concr,\barriers}}{\osys_\abs}(\state_0)$.
This entails the existence of a trace $\trace_\abs$ in 
$\semantic{\prog_{\hist_\concr,\barriers}}$ and such that 
$\state_f \in \eval{\trace_\abs}{\state_0}$. Let 
$\hist_\abs = \getHist{\trace_\abs}$.

We can now show $\hist_\concr \linchoice{\barriers} \hist_\abs$. First, we 
define the bijection $\bijection$ as expected. For each thread $\tid$, 
$\bijection$ maps (the index of) $c_i$ (\resp $r_i$) to (the index of) the call
action (\resp the return action) in $\hist_\abs$ corresponding to the command 
$C_i$.

Let $i,j \in \set{1,\dots,\length{\hist_\concr}}$ and 
$\barrier_{i,j} \in \barriers$, such that 
$i \separated{\barrier_{i,j}}{\hist_\concr} j$. By construction of 
$\prog_{\hist_\concr,\barriers}$, action ${\hist_\abs}_{\bijection(i)}$ is 
followed by a return action (or is itself a return action) which is followed 
by $\assumeCom{b = b'}$ in $\trace_\abs$ for some 
$\barrier' < \barrier_{i,j}$, while ${\hist_\abs}_{\bijection(j)}$ follows a
call action (or is itself a call action) which follows
$\assumeCom{b = b''}$ in $\trace_\abs$ for some 
$\barrier'' \geq \barrier_{i,j}$. This shows that 
${\hist_\abs}_{\bijection(i)}$ comes before ${\hist_\abs}_{\bijection(j)}$
in $\hist_\abs$, and thus that $\bijection(i) < \bijection(j)$, which ends the 
proof.

$(\ref{ebbl} \Rightarrow \ref{or-one})$
Let $\trace_\concr \in \semantic{\programs}$ be a trace with at most one 
method per thread, at most $k$ global writes and 
whose history $\hist_\concr$ is in $\osys_\concr$. Let $\state,\state'$ be two
states such that $\state' \in \eval{\trace_\concr}{\state}$.

We decompose $\trace_\concr$ into 
$\trace_\concr^0 w_0 \trace_\concr^1 \dots w_{k-1} \trace_\concr^{k}$ where 
$w_0,\dots,w_{k-1}$ are the (at most) $k$ global writes of $\trace_\concr$.

This induces a splitting
$\hist_\concr = \hist_\concr^0 \dots \hist_\concr^k$ where 
$\getHist{\trace_\concr^i} = \hist_\concr^i$. For $\i \in \set{1,\dots,k}$
let $\barrier_i$ be the index of the first action of $\hist_\concr^k$ in 
$\hist_\concr$, and let 
$\barriers = \set{\barrier_i\ |\ 1 \leq i \leq k}$.

By applying~\ref{ebbl}, there is a history $\hist_\abs \in \osys_\abs$
such that $\hist_\concr \plinchoice{\barriers} \hist_\abs$. We now derive a 
trace $\trace_\abs \in \semantic{\programs}$ from $\trace_\concr$, whose 
history is $\hist_\abs$ and such that $\state' \in \eval{\trace_\abs}{\state}$.

\end{proof}

% \begin{proof}
% \newcommand{\npairs}{{p_\tid}}
% \newcommand{\nbars}{k'}
% \newcommand{\ticker}{b}
% 
% $(\ref{or-one} \Rightarrow \ref{ebbl})$ Let 
% $\hist_\concr \in \osys_\concr$, 
% $\barriers \subseteq \set{1,\dots,\length{\hist_\concr}}$ with 
% $|\barriers| \leq k$. We define a program 
% $\prog_{\hist_\concr,\barriers}$ with no local reads excepted at the end of
% each thread, and at most $k$ global writes.
% Let $t$ be a thread-id appearing in $\hist_\concr$.
% Let $(c_1^\tid,r_1^\tid),\dots,\ab (c_\npairs^\tid,r_\npairs^\tid)$ the pairs 
% of matching call and return actions from $\hist_\concr$. 
% 
% Let $j_{c_i^\tid}$ and $j_{r_i^\tid}$ be the indexes of $c_i^\tid$ and 
% $r_i^\tid$ in $\hist_\concr$. Let $\barrier_{c_i^\tid}$ be the largest element 
% of $\barriers$ such that $\barrier_{c_i^\tid} \leq j_{c_i^\tid}$ if it exists,
% and $0$ otherwise. 
% We define $\barrier_{r_i^\tid}$ similarly. Let 
% $\callAct{\tid}{\obj_i^\tid}{\meth_i^\tid}{\cval_i^\tid} = c_i^\tid$ and 
% $\retAct{\tid}{\obj_i^\tid}{\meth_i^\tid}{\rval_i^\tid} = r_i^\tid$.
% 
% Let $C_i^\tid$ be the following sequence of commands:
% \begin{flalign*}
% &x_i^\tid \colonequals \cval_i^\tid;\text{  \comment{// local write}}&\\
% &\assumeCom{\ticker = \barrier_{c_i^\tid}};\text{  \comment{// global read}} 
% &\\
% &y_i^\tid \colonequals \callCom{\obj_i^\tid}{\meth_i^\tid}{x_i^\tid};
%   \text{  \comment{// local write}}&\\
% &\assumeCom{\ticker = \barrier_{r_i^\tid}};\text{  \comment{// global read}}&
% \end{flalign*}
% and $D_i^\tid$ be the command:
% \begin{flalign*}
% &\assumeCom{y_i^\tid = \rval_i^\tid};\text{  \comment{// local read}}&
% \end{flalign*}
% 
% The program of thread $t$ in $\prog_{\hist_\concr,\barriers}$ is then
% $C_1^\tid;\dots;C_\npairs^\tid;D_1^\tid;\dots;D_\npairs^\tid$. All the local 
% reads are executed at the end, and there are no global writes.
% We also add a new thread $\tid_\ticker$ in $\prog_{\hist_\concr,\barriers}$ 
% which has $k$ increments of the global variable $\ticker$.
% 
% Let $n_0$ be a value which doesn't appear in $\hist_\concr$ and $\state_0$ be 
% an initial state where all the local variables are initialized to $n_0$ and 
% the global variable $\ticker$ is initialized to $0$.
% 
% There is a trace $\trace_\concr$ in $\semantic{\prog_{\hist_\concr,\barriers}}$
% whose history is $\hist_\concr$ and which reaches a state $\state_f$ where 
% $x_i^\tid = \cval_i^\tid$ and $y_i^\tid = \rval_i^\tid$ for all $\tid$ and 
% $i \in \set{1,\dots,\npairs}$. Since we have $\state_f \in 
% \comp{\prog_{\hist_\concr,\barriers}}{\osys_\concr}(\state_0)$ and 
% $\osys_\concr$ observationally refines $\osys_\abs$, we also have
% $\state_f \in \comp{\prog_{\hist_\concr,\barriers}}{\osys_\abs}(\state_0)$.
% This entails the existence of a trace $\trace_\abs$ in 
% $\semantic{\prog_{\hist_\concr,\barriers}}$ and such that 
% $\state_f \in \eval{\trace_\abs}{\state_0}$. Let 
% $\hist_\abs = \getHist{\trace_\abs}$. Up to thread renaming, and without loss
% of generality, we can assume that thread $\tid$ in $\trace_\abs$ executed the
% command sequence $C_1^\tid;\dots;C_\npairs^\tid;D_1^\tid;\dots;D_\npairs^\tid$.
% 
% We can now show $\hist_\concr \plinchoice{\barriers} \hist_\abs$. First, we 
% define the bijection $\bijection$ as expected. For each thread $\tid$, 
% $\bijection$ maps $c_i^\tid$ (\resp $r_i^\tid$) to the call action (\resp the
% return action) in $\hist_\abs$ corresponding to the command $C_i^\tid$.
% 
% Let $i,j \in 
% \set{1,\dots,\length{\hist_\concr}}$ and $\barrier_{i,j} \in \barriers$, 
% such that $i \separated{\barrier_{i,j}} j$.
% By construction of $\prog_{\hist_\concr,\barriers}$, action 
% ${\hist_\abs}_\bijection{i}$ is followed by $\assumeCom{b = b'}$ in 
% $\trace_\abs$ for some $\barrier' < \barrier_{i,j}$, while 
% ${\hist_\abs}_\bijection{j}$ follows $\assumeCom{b = b''}$ in $\trace_\abs$
% for some $\barrier'' \geq \barrier_{i,j}$. This shows that 
% ${\hist_\abs}_\bijection{i}$ comes before ${\hist_\abs}_\bijection{j}$
% in $\hist_\abs$, and thus that $\bijection{i} < \bijection{j}$, which ends the 
% proof.
% 
% $(\ref{ebbl-tid} \Rightarrow \ref{or-many})$ 
% Let $\trace_\concr \in \semantic{\programs}$ be a trace with no local reads, 
% except possibly at the end of each thread, and at most $k$ global writes and 
% whose history $\hist_\concr$ is in $\osys_\concr$. Let $\state,\state'$ be
% states such that $\state' \in \eval{\trace_\concr}{\state}$.
% 
% We decompose $\trace_\concr$ into 
% $\trace_\concr^0 w_0 \trace_\concr^1 \dots w_{k-1} \trace_\concr^{k}$ where 
% $w_1,\dots,w_{k-1}$ are the (at most) $k$ global writes of $\trace_\concr$.
% 
% This induces a splitting
% $\hist_\concr = \hist_\concr^0 \dots \hist_\concr^k$ where 
% $\getHist{\trace_\concr^i} = \hist_\concr^i$. For $\i \in \set{1,\dots,k}$
% let $\barrier_i$ be the index of the first action of $\hist_\concr^k$ in 
% $\hist_\concr$, and let 
% $\barriers = \set{\barrier_i\ |\ 1 \leq i \leq k}$.
% 
% By applying~\ref{ebbl-tid}, there is a history $\hist_\abs \in \osys_\abs$
% such that $\hist_\concr \plinchoice{\barriers} \hist_\abs$. We now derive a 
% trace $\trace_\abs \in \semantic{\programs}$ from $\trace_\concr$, whose 
% history is $\hist_\abs$ and such that $\state' \in \eval{\trace_\abs}{\state}$.
% 
% \end{proof}
% 


