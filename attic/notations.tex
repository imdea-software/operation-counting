%!TEX root = draft.tex
\section{Preliminaries}
\label{sec:prelim}

We define in this section the formal context of our work\footnote{Some of the notations are borrowed from \citet{journals/tcs/FilipovicORY10}.}. We consider concurrent programs with an arbitrary number of threads, using a finite number of variables that can be either local (to a thread) or shared between threads, and calling methods (and waiting for their return values) provided by some external library.

To simplify the presentation, we consider that all the variables manipulated by programs, as well as parameters and return values of methods are of the same type of natural numbers (as we are going to add counters to programs later). All the results and the techniques presented in this paper are independent from the considered data type of program variables and method parameters. We also assume that methods have precisely one input and one output parameter, but the 
generalization to an arbitrary number of input/output parameters is straightforward. 




%In this section, we define the formal setting in which we prove our results. % (basically, we follow the notations introduced in \citep{FilipovicORY10}.}
%A library is abstractly defined as a set of histories, which are sequences of call and return actions.
%%Then, we define an abstract programming language  
%We consider a very restricted set of programs for the library clients without most of the constructs that we can find in real programming languages, e.g., conditionals or loops.  Our intention is to consider the simplest possible setting for our results -- yet without losing any generality (see Section~\ref{sec:obs_ref}).
%%and that this simple setting is indeed without loss of generality.
%%The programming language is defined by an unspecified set of primitive commands, commands for method calls, the sequential and the parallel composition. 
%%For the verification problems we consider in this paper, it is not necessary to consider other constructs such as non-deterministic choice or loops .

\subsection{Libraries} 

Libraries of methods are abstractly defined through their interface with users, i.e., by considering the set of all possible sequences of method calls and returns, with their associated input and output values, that can be observed when they are running in parallel with any program. Let us define this more formally.

Let $\mathbb{M}$ be a set of method names. 
A \emph{library action} is a 
\emph {call action} $\callAct{\tid}{\meth}{\cval}$
or a \emph{return action} $\retAct{\tid}{\meth}{\rval}$,
%\[ 
%  \libAct \coloncolonequals %(\tid,\act)\ |\
%  \callAct{\tid}{\meth}{\cval}\ |\
%  \retAct{\tid}{\meth}{\rval},
%\]
where $\tid$ is a thread id (\ie a natural number), $\meth$ is a method name in $\mathbb{M}$, $\cval$ is the parameter value,
and $\rval$ is the return value. 
%For simplicity, we suppose that all methods have exactly one input parameter and one output parameter
%and also, that the parameters and the return values are natural numbers. %belong to some data domain $\<Dom>$.
%A \emph{trace} is a sequence of client actions.
%An \emph{object action} $\objAct$ can be a call action or a return action.

A \emph{sequential history} is a sequence of library actions such that every call action $\callAct{\tid}{\meth}{\cval}$ is immediately followed by a matching return action 
$\retAct{\tid}{\meth}{\rval}$, %, for some $\rval$, 
except possibly for the last one.
Intuitively, in a sequential history, all actions are executed atomically.
%
In general, a \emph{history} is a sequence of library actions $\hist$ such that %every return matches a previous call action and 
for every thread $t$, the projection of $\hist$ on the actions of the thread $t$ is a sequential history. 
% \ie every $\retAct{\tid}{\lib}{\meth}{\rval}$ is
%preceded by an action 


An $\oper{\meth}{\cval}{\rval}$-operation (or more simply, operation) of a history $\hist$ is a pair of two actions $\callAct{\tid}{\meth}{\cval}$ and $\retAct{\tid}{\meth}{\rval}$, which are successive in the projection of $\hist$ over the actions of thread $\tid$. An $\oper{\meth}{\cval}{*}$-operation of a history $\hist$ is a call action $\callAct{\tid}{\meth}{\cval}$, which is not followed by a matching return. An $\oper{\meth}{\cval}{\rval}$-operation is called a \emph{completed} operation while an $\oper{\meth}{\cval}{*}$-operation is called \emph{pending}.


A history $\hist'$ is called a \emph{quiescent permutation} of a history $\hist$ if it is obtained from $\hist$ by shifting a call action to left or by shifting a return action to the right (without crossing actions of the same thread).
Formally, $\hist'$ is a quiescent permutation of $\hist$ if 
\begin{itemize}
	\item %if $\hist$ is a history in $L$, then any history $\hist'$ obtained from $\hist$ by moving one call action to the left belongs also to $L$. Formally, if 
	$\hist=\hist_1\cdot \hist_2\cdot \callAct{\tid}{\meth}{\cval}\cdot \hist_3$, where $\hist_2$ doesn't contain actions of thread $\tid$, and $\hist'=\hist_1\cdot \callAct{\tid}{\meth}{\cval}\cdot \hist_2\cdot  \hist_3$ ($\cdot$ denotes the concatenation of sequences), or
	\item %if $\hist$ is a history in $L$, then any history $\hist'$ obtained from $\hist$ by moving one return action to the right belongs also to $L$. Formally, if 
	$\hist=\hist_1\cdot \retAct{\tid}{\meth}{\rval}\cdot\hist_2\cdot \hist_3$, where $\hist_2$ doesn't contain actions of thread $\tid$, and $\hist'=\hist_1\cdot \hist_2\cdot \retAct{\tid}{\meth}{\rval}\cdot \hist_3$.
\end{itemize}
A set of histories $\lib$ is called \emph{quiescent permutation closed} iff for every history $\hist\in \lib$, $\lib$ contains all the quiescent permutations of $\hist$.

For example, consider the following history ($\_$ is a special value denoting the fact that the method has no input or return value):
{\small
\[
\hist=\callAct{\tid_1}{push}{1}\cdot \callAct{\tid_2}{pop}{\_}\cdot \retAct{\tid_2}{pop}{1}\cdot \retAct{\tid_1}{push}{\_}.
\]}
Then, the following two histories are quiescent permutations of $\hist$:
{\small
\[
\begin{array}{l}
\hist_1=\callAct{\tid_2}{pop}{\_}\cdot \callAct{\tid_1}{push}{1}\cdot \retAct{\tid_2}{pop}{1}\cdot \retAct{\tid_1}{push}{\_},\\[.5mm]
\hist_2=\callAct{\tid_1}{push}{1}\cdot \callAct{\tid_2}{pop}{\_}\cdot \retAct{\tid_1}{push}{\_} \cdot \retAct{\tid_2}{pop}{1}.
\end{array}
\]}

The idea behind quiescent permutation closure is that, from the point of view of the user of a library (1) it is always possible to execute earlier a call  event of a method along a computation provided that the order of actions of each of thread is respected, because the execution of internal operations of the concerned method can stay at their original positions, and (2) similarly  it is always possible to delay the return event of a method since the returned value can be kept internally by the method in some local variable. More precisely, consider the examples of histories given above.
%
%We will define a library as a set of histories, which is quiescent permutation closed. We use the two examples above to motivate this fact.
The history $\hist_1$ is obtained by shifting $\callAct{\tid_2}{pop}{\_}$ to the left. We consider that a library %n implementation of a concurrent stack (with $push$ and $pop$ methods) 
which allows $\hist$ should also allow $\hist_1$. If there exists a concrete execution of these two methods, where the $push$ method is invoked before $pop$, then there also exists an execution where $pop$ is invoked before $push$ (with the same parameter and return values). In the latter execution, the method $pop$ is invoked first but executes its first action at the same moment as in the former execution.
%the implementations of its methods can not constrain the order in which the call actions are issued by the clients of the library.
Then, the history $\hist_2$ is obtained by shifting $\retAct{\tid_2}{pop}{1}$ to the right. If we assume w.l.o.g. that all the return statements in a library are of the form {\tt return} $x$, where $x$ is a method-local variable, then a library which allows $\hist$ should also allow $\hist_2$. This is because the value of the method-local variable $x$ is not affected by concurrently executing methods.

\begin{definition}[Library]
A \emph{library} $\lib$ is a set of histories, which is prefix closed and quiescent permutation closed.
\end{definition}

%\todo{See if the condition that call actions can be moved to the left is still required}

The set of all method names appearing in $L$ is denoted by $\meths{L}$.
%
%An $\oper{\meth}{\cval}{\rval}$-operation (or more simply, operation) of a history $\hist$ is a pair of two actions $\callAct{\tid}{\meth}{\cval}$ and $\retAct{\tid}{\meth}{\rval}$, which are successive in the projection of $\hist$ over the actions of thread $\tid$. An $\oper{\meth}{\cval}{*}$-operation of a history $\hist$ is a call action $\callAct{\tid}{\meth}{\cval}$, which is not followed by a matching return. An $\oper{\meth}{\cval}{\rval}$-operation is called a \emph{completed} operation while an $\oper{\meth}{\cval}{*}$-operation is called \emph{pending}.
%
Many of the results in this paper are established for  \emph{thread-independent} libraries. Intuitively, these are libraries whose behaviors do not depend on the thread ids. 

\begin{definition}[Thread-independent library]
A library $L$ is \emph{thread-independent} iff for every history $h$ in $\sem{L}$, $\sem{L}$ contains all the histories $h'$ obtained 
from $h$ by replacing the thread id $\tid$ of an operation with some arbitrary value $\tid'$.
% pair of matching call and return actions $\callAct{\tid}{\meth}{\cval}$ and $\retAct{\tid}{\meth}{\rval}$ (resp., an unmatched call action $\callAct{\tid}{\meth}{\cval}$) with some arbitrary value $\tid'$.
%every thread id $t$ with $\alpha(t)$ belongs also to $\sem{L}$.
\end{definition}
%A library $L$ is \emph{data-independent} iff for any history $h$ in $\sem{L}$ and any injective function $\beta:\<Dom>->\<Dom>$, the history obtained from $h$ by replacing every actual parameter or return value $d$ with $\beta(d)$ belongs also to $\sem{L}$.

\subsection{Clients}

We consider that client programs can be any parallel composition of an arbitrary number of threads $t_1, t_2, ...$, sharing a finite number of variables $x, y, ...$ and calling methods $f, g, ...$ in some external library $L$. 
%
However, since our goal is to investigate observational refinement  between libraries, it is actually enough to reason about a restricted class of programs without conditional branches, nondeterminism, nor loops. Indeed, it is possible to reduce the problem of checking observational refinement for the general class of programs to the one for this particular class of programs (that can be seen as bounded unfoldings of general programs). This issue is discussed later in Section~\ref{sec:obs_ref}.

Therefore, we consider the following syntax for {\em sequential commands}
$\commands$ and {\em client programs} $\programs$ (called simply programs):
\begin{align*}
\commands \coloncolonequals\ & c\ |\
%x := E\ |\ 
y \colonequals \callCom{\meth}{x}\ |\ 
%\assume B \ |\
%\assert B \ |\ 
%\yield \ |\
%\\
%&
\commands; \commands\ %|\
%\commands + \commands\ |\ 
%\commands^*\
\\
\programs \coloncolonequals & \commands \| \dots \| \commands
%\spawn \commands \ |\ \programs;\programs\ |\ \programs+\programs\ |\  \programs^* %\commands \| \dots \| \commands
\end{align*}
where $c$ ranges over a set $\pcommands$ of {\em primitive commands} that we let unspecified, though
%and $x$, $y$ are variables interpreted as natural numbers.
we assume that $\pcommands$ includes (atomic sequences of) assignments of the form $x:=E$, where $E$ is an arithmetic expression, 
and $\assume\ B$ statements (that are executable only when the boolean expressions $B$ evaluate to true). %and an $\assert\ B$ statement, which results in an error when $B$ is not true. 
%Also, in order to simplify the presentation, we assume that there exists a primitive command $\yield$, which allows the current thread to give up control to the thread scheduler. 
%Therefore, a sequential command with no $\yield$ is executed atomically.
%the sequential commands interleavings 

Given a program $P=\commands_1 \| \dots \| \commands_k$, a variable in $P$ is called \emph{shared} if it appears in at least two sequential commands $C_i$ and $C_j$ with $i\neq j$ and \emph{local} otherwise.

%$E$ ranges over an unspecified set $\expr$ of expressions interpreted over an infinite data domain $\<Dom>$, which contains the nullary non-deterministic choice operator $\star$.

Let us now give a formal semantics of this class of programs. We are going to define the notion of traces of a program $P$ when it is executed together with an external library $L$. Such traces are sequences of actions that are either corresponding to internal operations of $P$, or call/return actions corresponding to the execution of operations in $L$. 

A \emph{client (internal) action} is a pair $(t,a)$, where $a$ is an atomic client operation taken from a set $\clientop$.
%(corresponding to the commands in $\pcommands$). 
We assume that every primitive command $c$ in $\pcommands$ is interpreted as a set of atomic client actions, denoted by $\sem{c}$. 
%Let $\prog$ be a program.
A client action $(t,a)$ is called a \emph{{\shwrite} for $\prog$} if $a$ modifies the value of some shared variable of $\prog$ and 
%belongs to $\Writes{x}$, for some shared variable $x$ of $P$.
a primitive command $c$ of $\prog$ is called a \emph{\shwrite} if its interpretation $\sem{c}$ contains a {\shwrite} for $\prog$. % client action.



An \emph{action} is a client action or a library action.
% action is a client action $(t,a)$, where $a$ is an internal action, % belonging to some unspecified set  that assigns the value $n\in\<Dom>$ to the variable $x$, 
%or a library action.
A \emph{trace} is a sequence of actions such that 
for every thread $t$, the projection over the library actions of the thread $t$ is sequential.
The history $\getHist{\trace}$ of a trace $\trace$ is the projection of $\trace$ on library actions.
The notion of operation is extended to traces straightforwardly. 

%We assume that every primitive command $c$ is interpreted as a set of atomic client actions, denoted by $\sem{c}$. 
Given a program $\prog$, its set of traces, denoted by $\toTraceFinal{\prog}$, is defined in Figure~\ref{fig:traces}.
For simplicity, we assume that a program state is a valuation for its variables; the set of all program states is denoted by $\States$.
The set of states reachable by trace $\trace$ from state $\state$, denoted by $\eval{\trace}{\state}$, is defined in Figure~\ref{fig:eval}. 
Note that the effect of a command $y \colonequals \callCom{\meth}{x}$ over the client's state is encoded in the assignment $y:=\rval$
from the trace corresponding to this command. Therefore, the library actions themselves do not affect the state of the client.
%Since a state represents a valuation for the client's variables, it is not affected by call and return actions.

\begin{figure}[t]
\begin{align*}
\toTrace{c}{\tid}&= \{(\tid,a)\mid a\in\sem{c}\}\\ %(\tid,a_1)\cdot\ldots\cdot (\tid,a_k)\mid a_1\cdot\ldots\cdot a_k\in\sem{c}
\toTrace{y\colonequals \callCom{\meth}{x}}{\tid}&
= \{ \trace_1\cdot\callAct{\tid}{\meth}{\cval}\cdot
\retAct{\tid}{\meth}{\rval}\cdot \trace_2\ |\
\cval,\rval \in \<Nats>, \\
& \hspace{5mm}\trace_1 \in \toTrace{\assume\ x=n}{\tid}, \trace_2 \in \toTrace{y:=m}{\tid} \}\\
\toTrace{\commands_1;\commands_2}{\tid}& = 
  \set{\trace_1\cdot\trace_2\ |\ \trace_i \in \toTrace{\commands_i}{\tid},\mbox{ for all $1\leq i\leq 2$}}\\
%\semantic{\tid,\commands_1 + \commands_2}& = 
%  \semantic{\tid,\commands_1} \cup \semantic{\tid,\commands_2}\\
%\semantic{\tid,\commands^*}& = \semantic{\tid,\commands}^*\\
\toTraceFinal{\commands_1\|\dots\|\commands_n}& =
  \{\trace|\ \trace\mbox{ is a prefix of a sequence in }\trace_1\shuffle \dots\shuffle\trace_n\\
  &\hspace{5mm}\mbox{with }  \trace_i \in \toTrace{\commands_i}{i}, 
\mbox{for all $1\leq i\leq n$}\}
% T (C1+C2 )t = T (C1 )t ∪ T (C2)t
% T (C )t = (T (C)t)
% T (C1 · · · Cn) =
%  { interleave(τ1, ..., τn) | τi ∈ T (Ci )i ∧ 1 ≤ i ≤ n }
\end{align*}
\caption{The set of traces of a program. 
Here, $\shuffle$ denotes the shuffling (interleaving) operator over sequences:
for any two sequences $\sigma_1$ and $\sigma_2$, $\sigma_1\shuffle\sigma_2$ is the set of 
sequences $w_1\cdot v_1\cdot w_2\cdot v_2\cdot\ldots \cdot w_i\cdot v_j$, for all 
decompositions $\sigma_1=w_1\cdot w_2\cdot\ldots\cdot w_i$ and $\sigma_2=v_1\cdot v_2\cdot\ldots\cdot v_j$.}
\label{fig:traces}
\end{figure}
%
\begin{figure}[h]
\begin{align*}
%\evalSymb &: \States \times \wfTraces \rightarrow \mathcal{P}(\States)\\
\eval{\trace\cdot \callAct{\tid}{\meth}{\cval}}{\state} &= 
  \eval{\trace}{\state}\\
\eval{\trace\cdot \retAct{\tid}{\meth}{\rval}}{\state} &= 
  \eval{\trace}{\state}\\
\eval{\trace\cdot (\tid,\act)}{\state} &=
	\set{\state''\ |\  (\state',\state'') \in \sem{\act}\mbox{ and }\state' \in \eval{\trace}{\state}}\\
\eval{\epsilon}{\state} &= \set{\state}
% evaleval(s, τ ; (t, call o.f (n)))eval(s, τ ; (t, ret(n) o.f ))eval(s, τ ; (t, a))eval(s, ):===
% =States × WTraces → P(States)
% eval(s, τ )
% eval(s, τ )
% s ∈eval(s,τ ) {s | (s , s ) ∈ [[a]]}
% {s}
\end{align*}
\caption{Trace evaluation. We assume that every atomic client action is interpreted as a binary relation over states, denoted by $\sem{a}$.}
\label{fig:eval}
\end{figure}


We define \emph{the set of traces} in the composition of a program $\prog$ with a library $\lib$, denoted by $\traces{\programs}{\lib}$, as follows:
\[
%\comp{\programs}{\lib} &: \States \rightarrow 2^{\States}\\
\traces{\prog}{\lib} =
\set{\trace\ |\ \trace \in \sem{\prog} \mbox{ and }
  \getHist{\trace} \in \lib}.
\]
Also, \emph{the set of reachable states} in the composition of $\prog$ with $\lib$ starting from state $\state_0$, denoted by $\reach{\programs}{\lib}{\state_0}$ is defined by:
\[
%\comp{\programs}{\lib} &: \States \rightarrow 2^{\States}\\
\reach{\prog}{\lib}{\state_0} =
\set{\eval{\trace}{\state_0}\ |\ \trace \in \traces{\prog}{\lib}}.
\]

A trace $\trace$ is called \emph{executable} iff there exists some initial state $\state_0$ such that $\eval{\trace}{\state_0}$ is non-empty. The set of executable traces in the composition of a program $\prog$ with a library $\lib$ is denoted by $\tracesExec{\programs}{\lib}$, \ie
\[
%\comp{\programs}{\lib} &: \States \rightarrow 2^{\States}\\
\tracesExec{\prog}{\lib} =
\set{\trace\ |\ \trace \in \traces{\prog}{\lib} \mbox{ and }
  \exists \state_0\in\States.\ \eval{\trace}{\state_0}\neq\emptyset}.
\]

Two actions $\alpha$ and $\alpha'$ are \emph{independent} iff they contain different thread-ids and $\eval{\alpha\cdot\alpha'}{\state}=\eval{\alpha'\cdot\alpha}{\state}$ for all states $\state$. Otherwise, they are \emph{dependent}.

Two traces $\trace$ and $\trace'$ are \emph{equivalent}, denoted by $\trace\equiv\trace'$, iff $\trace'$ is a permutation of $\trace$ that preserves the order between \emph{dependent} actions.
Formally, $\trace\equiv\trace'$ iff there exists a bijection $\bijection:\{1,\ldots,|\trace|\}->\{1,\ldots,|\trace'|\}$ such that, for every $i$, the action on position $i$ in $\trace$ is the same as the action on position $\bijection(i)$ in $\trace'$ and for every two dependent actions of $\trace$ occurring on the positions $i$ and $j$, with $i<j$, we have that $\bijection(i)<\bijection(j)$.


%For any variable $x$, the set of atomic client operations that modify the value of $x$ is denoted by $\Writes{x}$.
%Let $\prog$ be a program.
%A client action $(t,a)$ is called a \emph{{\shwrite} for $\prog$} if $a$ modifies the value of some shared variable of $\prog$ and 
%%belongs to $\Writes{x}$, for some shared variable $x$ of $P$.
%a primitive command $c$ of $\prog$ is called a \emph{\shwrite} if its interpretation $\sem{c}$ contains a {\shwrite} for $\prog$. % client action.


%Given a trace $\trace$, a \emph{phase} of $\trace$ is a continuous maximal subsequence $\phase$ of $\trace$ delimited by two consecutive {\shwrites}, \ie $\trace=\trace_0 \cdot\phase\cdot\trace_1$, where $\trace_0$ ends in a {\shwrite}, $\trace_1$ starts with a {\shwrite}, and $\phase$ doesn't contain any {\shwrite}.


 

%The projections of the states in $\reach{\prog}{\lib}{\state_0}$ over the set of variables $X$ is denoted by $\projReach{\prog}{\lib}{\state_0}{X}$.

\subsection{Observational Refinement}\label{sec:obs_ref}

\begin{definition}[\bf Observational refinement]
Let $\lib$ and $\spec$ be two libraries, and $\pclass$ a class of programs. % (where programs are defined as in the previous section). 
We say that \emph{$\lib$ observationally refines $\spec$ w.r.t $\pclass$} iff
\[
\forall \prog\in\pclass,\ \state_0\in\States.\ \reach{\prog}{\lib}{\state_0}\subseteq \reach{\prog}{\spec}{\state_0}.
%this definition is equivalent to the one in \citet{FilipovicORY10}.}
\]

When $\pclass$ contains all programs, we simply say that $\lib$ observationally refines $\spec$.
\end{definition}

Let us discuss now our definition of programs in the context of checking observational refinement. In fact, there are definitions of observational refinement, \eg the one in \citet{journals/tcs/FilipovicORY10}, where a program may contain non-deterministic choices (conditionals) and loops. 
However, because of the universal quantification over all the programs, the effect of these constructs is already captured.
%it is not necessary to consider programs with such constructions. %sufficient to consider only programs 
For instance, for $\prog_1+\prog_2$ that represents the non-deterministic choice between two programs 
$\prog_1$ and $\prog_2$ defined according to our syntax, 
\[
\forall\state_0\in\States.\ \reach{\prog_1+\prog_2}{\lib}{\state_0}\subseteq \reach{\prog_1+\prog_2}{\spec}{\state_0}
\]
is implied by 
\[
\forall \state_0\in\States.\ \reach{\prog_i}{\lib}{\state_0}\subseteq \reach{\prog_i}{\spec}{\state_0},\mbox{ for all $i\in\{1,2\}$}
\]
(as usual $\reach{\prog_1+\prog_2}{\lib}{\state_0}=\reach{\prog_1}{\lib}{\state_0}\cup \reach{\prog_2}{\lib}{\state_0}$).
%
Similarly, loops are captured by quantifying over all their bounded unfoldings. 

%which are parallel compositions of a fixed number of sequential commands, each sequential command being a sequential composition of boundedly-many primitive commands.
%with sequential compositions of boundedly-many primitive commands and parallel compositions of boundedly-many sequential commands.

%\begin{remark}
%If $\lib$ observationally refines $\spec$, then $\meths{\lib}\subseteq \meths{\spec}$. To simplify the presentation, we will also assume that we 
%\end{remark}

