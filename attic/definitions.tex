\section{Definitions} 


Given a history $\hist$, we denote by $\pred{\hist}$ the \emph{precedence
relation} over object action indexes of $\hist$ defined as follows:
\begin{align*}
i \pred{\hist} j \Leftrightarrow 
&\exists i',j'.\ i \leq i' < j' \leq j \land
\isRet{\hist_{i'}} \land \isCall{\hist_{j'}} \land \\
& \getTid{\hist_i} = \getTid{\hist_{i'}} \land 
\getTid{\hist_j} = \getTid{\hist_{j'}}
\end{align*}

Let $\hist$ and $\hist'$ be two histories. We say that $\hist$ is 
\emph{\linearizable} with regard to $\hist'$, noted $\hist \lin \hist'$
if there is a bijection 
$\bijection: \set{1,\dots,\length{\hist}} \rightarrow
\set{1,\dots,\length{\hist'}}$ such that:
\[
(\forall i. \pos{\hist}{i} \equivSymb \pos{\hist'}{\bijection(i)}) \land
(\forall i,j.\ i \pred{\hist} j \implies
              \bijection(i) \pred{\hist} \bijection(j)) 
\]

It is said to be \emph{\precisely} with regard to $\hist'$, noted 
$\hist \plin \hist'$ if it satisfies the
same constraint but with $=$ instead of $\equivSymb$, that is, the 
bijection needs to respect the thread-ids.

Given a history $\hist$, we denote by $\separated{\barrier}{\hist}$ the 
relation over object action indexes of $\hist$ defined as follows:
\begin{align*}
i \separated{\barrier}{\hist} j \Leftrightarrow 
&\exists i',j'.\ i \leq i' < \barrier \leq j' \leq j \land
\isRet{\hist_{i'}} \land \isCall{\hist_{j'}} \land \\
& \getTid{\hist_i} = \getTid{\hist_{i'}} \land 
\getTid{\hist_j} = \getTid{\hist_{j'}}
\end{align*}

Given a set of integers $\barriers$, the relation 
$\separated{\barriers}{\hist}$ is the union of the relations 
$\separated{\barrier}{\hist}$ for $\barrier \in \barriers$.
% $\barrier \in \barriers$ such that $i \separated{\barrier} j$.

We say that $\hist$ is \emph{\linearizable{} for $\barriers$} with regard to 
$\hist'$, noted $\hist \linchoice{\barriers} \hist'$
if there is a bijection $\bijection: \set{1,\dots,\length{\hist}} \rightarrow
\set{1,\dots,\length{\hist'}}$ such that:
\[
(\forall i. \pos{\hist}{i} \equivSymb \pos{\hist'}{\bijection(i)}) \land
(\forall i,j.\ i \separated{\barriers}{\hist} j \implies
              \bijection(i) < \bijection(j))
\]
and likewise for \emph{\precisely{} for $\barriers$}, noted 
$\hist \plinchoice{\barriers} \hist'$.

In a history, a \emph{barrier} is the index of a call action which follows a 
return action. We denote by $\maxbar{\osys}{k}$ the subset of histories with $k$ 
barriers or less.

We use the same syntax as \citet{concurrentobjets} for sequential commands 
$\commands$ and complete programs $\programs$:
\begin{align*}
\commands \coloncolonequals &\command\ |\ 
x \colonequals \callCom{\obj}{\meth}{y}\ |\ 
\commands; \commands\ |\
\commands + \commands\ |\ 
\commands^*\\
\programs \coloncolonequals &\commands \| \dots \| \commands
\end{align*}
where $\command$ ranges over an unspecified set $\pcommands$ of primitive 
commands.
\begin{figure} 
\begin{align*}
\semantic{\tid,\command}&\text{ is an action }(\tid,\act)\\
\semantic{\tid,x \colonequals \callCom{\obj}{\meth}{y}}&
= \set{ \callAct{\tid}{\obj}{\meth}{y}; 
\retAct{\tid}{\obj}{\meth}{\rval}; \trace\ |\
\rval \in \domain \land \trace \in \semantic{\tid,x \colonequals \rval}}\\
\semantic{\tid,\commands_1;\commands_2}& = 
  \set{\trace_1\trace_2\ |\ \trace_i \in \semantic{\tid,\commands_i}}\\
\semantic{\tid,\commands_1 + \commands_2}& = 
  \semantic{\tid,\commands_1} \cup \semantic{\tid,\commands_2}\\
\semantic{\tid,\commands^*}& = \semantic{\tid,\commands}^*\\
\semantic{\tid,\commands_1\|\dots\|\commands_n}& =
  \set{\interleave(\trace_1,\dots,\trace_n\ |\ 
    \trace_i \in \semantic{\tid,\commands_i}}
% T (C1+C2 )t = T (C1 )t ∪ T (C2)t
% T (C )t = (T (C)t)
% T (C1 · · · Cn) =
%  { interleave(τ1, ..., τn) | τi ∈ T (Ci )i ∧ 1 ≤ i ≤ n }
\end{align*}
\caption{Action Trade Model}
\end{figure}

\begin{figure}
\begin{align*}
\evalSymb &: \States \times \wfTraces \rightarrow \mathcal{P}(\States)\\
\eval{\trace;\callAct{\tid}{\obj}{\meth}{y}}{\state} &= 
  \eval{\trace}{\state}\\
\eval{\trace;\retAct{\tid}{\obj}{\meth}{\rval}}{\state} &= 
  \eval{\trace}{\state}\\
\eval{\trace;(\tid,\act)}{\state} &=
  \cup_{\state' \in \eval{\trace}{\state}} \set{\state''\ |\
    (\state',\state'') \in \means{\act}}\\
\eval{\epsilon}{\state} &= \set{\state}
% evaleval(s, τ ; (t, call o.f (n)))eval(s, τ ; (t, ret(n) o.f ))eval(s, τ ; (t, a))eval(s, ):===
% =States × WTraces → P(States)
% eval(s, τ )
% eval(s, τ )
% s ∈eval(s,τ ) {s | (s , s ) ∈ [[a]]}
% {s}
\end{align*}
\caption{Trace evaluation}
\end{figure}
 
We define the composition $\comp{\programs}{\osys}$ of a program 
$\programs$ with 
an object system $\osys$ as follows:
\begin{align*}
\comp{\programs}{\osys} &: \States \rightarrow \mathcal{P}(\States)\\
\comp{\programs}{\osys}{\state} &=
 \cup \set{\eval{\trace}{\state}\ |\ \trace \in \semantic{\programs} \land
  \getHist{\trace} \in \osys}
\end{align*}