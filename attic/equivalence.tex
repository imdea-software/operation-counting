%!TEX root = draft.tex
\section{Observational refinement is equivalent to history inclusion}\label{sec:equiv}

We show that the observational refinement between two libraries can be simply expressed as the inclusion between 
their sets of histories.This result reduces the burden of reasoning about the universal quantifier over programs and it is 
similar to the relation between observational refinement and linearizability~\cite{journals/tcs/FilipovicORY10}: linearizability implies
observational refinement and observational refinement implies linearizability but only for 
executions where every operation is completed 
(see the discussion below). To explain
the differences we first recall the notion of linearizability. Thus, a library execution $e_1$ is linearizable w.r.t. a library $L_2$
iff there exists an execution $e_1'$ obtained from $e_1$ by appending return actions (for unmatched call actions) or deleting call actions
and an execution $e_2$ of $L_2$ such that $e_1'$ doesn't contain pending operations, $e_1'$ and $e_2$ contain exactly the same set of actions and $e_2$ is a 
permutation of $e_1'$ that preserves the order between 
return and call actions (i.e., if a return action $r\in R$ occurs before a call action $c\in C$ in $e_1'$ then the same holds in $e_2$).
Then, a library $L_1$ is linearizable w.r.t. a library $L_2$ iff every execution $e_1\in L_1$ is linearizable w.r.t. $L_2$.
%

If we consider that libraries are defined as arbitrary sets of well-formed sequences of call and return actions with possibly 
pending operations, then observational refinement does not imply linearizability. 
For example, this holds for two libraries $L_1$ and $L_2$ containing a method $m$ that never finishes. 
Intuitively, a program can observe that it called the method $m$ with both libraries but the library execution of $L_1$
that allowed this observation is not linearizable. More precisely, let $L_1$ and $L_2$ be two libraries that contain 
a single execution $\tup{m,o}$ and $P$ a program with a single execution 
$\print{\tup{m,o}} \cdot \tup{m,o}$, where $\print{\tup{m,o}}$ is a program action. The observation of $P$ is $\print{\tup{m,o}}$
and it is possible when composing $P$ with both $L_1$ and $L_2$ but the library execution $\tup{m,o}$ is not linearizable 
w.r.t. $L_2$ because $L_2$ doesn't contain an execution where $m$ finishes.

When considering two real libraries $L_1$ and $L_2$, which must satisfy the closure 
properties in Section~\ref{sec:obsref}, we show that observational refinement between $L_1$ and $L_2$ is actually 
\emph{equivalent} to a plain inclusion between histories. Compared to the
result in \citet{journals/tcs/FilipovicORY10}, we give a property that is equivalent to observation refinement irrespectively of the
presence of pending operations, we consider a comparison between histories, which are more abstract 
than library executions (see Example~\ref{}), and we consider a universal model of computation for programs 
(the result in \citet{journals/tcs/FilipovicORY10} is proved only for programs with shared variables). 

Our result can be explained by the following relation between linearizability and the ``weaker than'' comparison between histories.

\begin{proposition}\label{prop:lin}
An execution $e_1$ is linearizable w.r.t. a library $L_2$ iff there exists an execution $e_2\in L_2$, where every operation is completed, 
such that $H(e_1)\preceq H(e_2)$.
\end{proposition}


%Then, linearizability between an execution $e_1$ and a library $L_2$ is equivalent to the fact that the history of $e_1$, where
%some pending operations may be deleted, is weaker than a history of $L_2$ with no pending operations. Linearizability
%between two libraries $L_1$ and $L_2$ means that every execution of $L_1$ is linearizable w.r.t. $L_2$.

Thus, if an execution $e_1$ is linearizable w.r.t. a library $L_2$, hence the comparison between their histories 
stated in Proposition~\ref{prop:lin} holds, then by Lemma~\ref{lemma:lib_closure}, 
there exists a history of $L_2$ which is actually equivalent to the history of $e_1$.


\paragraph{Linearizability w.r.t. sequential specifications.} Concurrent implementations of data structures are usually considered to be correct if they are linearizable w.r.t. the sequential version of the same data structure, e.g., a concurrent stack should be linearizable w.r.t. the sequential stack. In our framework, we consider only concurrent libraries, therefore, this correctness condition translates to the fact that the concurrent implementation is an observational refinement of the atomic implementation, where, intuitively, method bodies execute in one single atomic step but, method calls and returns can still overlap. 

Given a \emph{sequential} library $L_{seq}$, i.e., whose executions are all sequential, the set of executions allowed by the \emph{atomic} concurrent library $L_{at}$ where method bodies are exactly the same as in $L_{seq}$ (modulo the fact that they are surrounded by synchronization actions, e.g., locks, in order to ensure that they are executed atomically) can be defined as the least set of executions that contains all the executions of $L_{seq}$ and that satisfies the closure properties in Section~\ref{sec:obsref}. Then, the following result is a direct consequence of the definitions.

\begin{proposition}\label{prop:atom}
An execution $e_1$ is linearizable w.r.t. a sequential library $L_{seq}$ iff $H(e_1)\in H(L_{at})$, where $L_{at}$ is the atomic concurrent library corresponding to $L_{seq}$.
\end{proposition}

The extension of Proposition~\ref{prop:atom} to libraries is straightforward, i.e., a library $L$ is linearizable w.r.t. a sequential library $L_{seq}$ iff $H(L)\subseteq H(L_{at})$, where $L_{at}$ is as above (or equivalently, by Theorem~\ref{th:equiv}, $L$ is an observational refinement of $L_{at}$).


